%%%%%%%%%%%%%%%%%%%%%%%%%%%%%%%%%%%%%%%%%%
\section{Input Parameters}
\label{sec:input_parameters}
%%%%%%%%%%%%%%%%%%%%%%%%%%%%%%%%%%%%%%%%%%
\subsection{Model Options}

\begin{tabular}{llp{4in}}
  \textbf{Option}&\textbf{Default}&\textbf{Description}\\
  \hline
  \texttt{numvar} & 3   & MHD model. 1: 2-field;  2: 4-Field;  3: 6-Field.\\
  \texttt{linear} & 0   & 1: linear (perturbation terms only, no matrix
  recalculation)\\
  \texttt{eqsubtract}& 0& 1: remove equilibrium terms from equations\\
  \texttt{icsubtract}& 0& set to 1 if PF coils are in the domain.  These are
  defined in the files ``coils.dat'' and ``current.dat''\\
  \texttt{extsubtract} & 0 & 1: subtract fields from non-axisymmetric coils \\
  \texttt{idens}  & 1   & 1: include density equation\\
  \texttt{ipres}  & 0   & 1: include electron pressure equation\\
  \texttt{ipressplit} & 0 & 1: seperate pressure solve from the magnetic field
  solves when isplitstep=1.  (ipressplit must be 0 for isplitstep=0) \\
  \texttt{itemp} & 0 & 1: Advance temperatures rather than pressures (for ipressplit=1 only) \\
  \texttt{gyro}   & 0   & 1: include Braginskii gyroviscous term.  (note:
  needs db to be nonzero also) \\
  \texttt{igauge} & 0   & 0: loop voltage applied to boundary psi only \\
  \texttt{inertia} & 1  & 1: Include $\u \cdot \grad{\u}$ terms\\
  \texttt{itwofluid}& 1 & 1: Include $\j\times\B$ and
  $\grad{p_e}$ terms in Ohm's law (electron form).  2: ion form (not
  recommended)  3: parallel pressure gradient in Ohm's law only
  (not recommended) \\
  \texttt{ibootstrap} & 0 & 1: Using bootstrap coefficients as a function of $\psi$; \\
                      &   & 2: Using bootstrap coefficients as a function of $T_e$ \\
                      &   & 3: Using bootstrap coefficients as a function of $\hat{T}=1-T_e/\max{T_e}$, coefficients are updated within M3D-C1 \\
  \texttt{ibootstrap\_model} & 0 & 1: Sauter (with eqsubtract = 0/1); \\
                             &   & 2: Redl (with eqsubtract = 0/1); \\
                             &   & 3: Sauter Model (with eqsubtract = 0); \\
                             &   & 4: Redl Model (with eqsubtract = 0) \\
  \texttt{ibootstrap\_regular} & 1e-8 & regularization term used in bootstrap current calculations \\
  \texttt{bootstrap\_alpha} & 1 & Amplification factor for bootstrap current \\
  \texttt{imp\_bf} & 0 & 1: include implicit equation for f (recommended for
  3D and 2D complex) \\
  \texttt{nosig} & 0 & 1: drop sigma terms from momentum equation \\
  \texttt{itor}   & 0   & 0: cartesian; 1: cylindrical\\
  \texttt{istatic}& 0   & 1: Do not advance velocity\\
  \texttt{iestatic}&0   & 1: Do not advance magnetic fields\\
  \texttt{chiiner} & 1. & factor to multiply the chi equation inertial terms \\
  \texttt{ieq\_bdotgradt} & 1. & 1: include equilibrium parallel T gradient \\
  \texttt{no\_vdg\_T} & 0 & 1: do notinclude V dot grad T in Temp equation (debug) \\
  \texttt{iwall\_is\_limiter} & 1 & 1: wall acts as limiter \\
  \texttt{kinetic} & 0 & 1: Use kinetic PIC for hot pressure, 
                         2: Incompressible CGL,
                         3. Full CGL  
\end{tabular}

\begin{tabular}{llp{4in}}
  \textbf{Option}&\textbf{Default}&\textbf{Description}\\
  \hline
  \texttt{irunaway} & 0 & 1:  include runaway electron model \\
  \texttt{cre} & 0 & runaway speed \\
  \texttt{imp\_temp} & 0 & 0: compute temperatures for isplitstep=0, itemp=0 \\
  \texttt{iohmic\_heating} & 1 & 1: Include Ohmic heating term\\
  \texttt{irad\_heating} & 1 & 1: Include radiation heat sink \\
  \texttt{gravr} & 0 & gravitational acceleration in R-direction \\
  \texttt{gravz} & 0 & gravitational acceleration in Z-direction 
\end{tabular}

\subsection{Initial Conditions Options}

\begin{tabular}{llp{4in}}
  \textbf{Option}&\textbf{Default}&\textbf{Description}\\
  \hline
  \texttt{itaylor} & 0 & \begin{minipage}[t]{2.5in}
    Pre-defined initial conditions.\\
 {\bf for itor=1 (toroidal geometry)} \\ 
    0: Tilting cylinder \\
    1: Calls Grad-Shafranov solver \\
    2: magneto-rotational equilibrium \\
    3: rotational instability \\   
    40: Fixed boundary stellarator \\
    41: Free boundary stellarator \\
 {\bf for itor=0 (slab geometry) } \\
    0: Tilting cylinder\\
    1: Taylor Reconnection\\
    2: Force-Free equilibrium (Taylor state) \\
    3: GEM Reconnection\\
    4: Wave Propagation\\
    5: Gravitational Instability\\
    6: Strauss equilibrium \\
    7: circular field init \\
    8,9:  biharmonic \\
    10,11,12,13:: analytic RWM test problem \\
    14: 3D wave test \\
    15: 3D diffusion test \\
    16:  FRS cylindrical equilibrium \\
    17:  ftz init \\
    18:  eigen init \\
    19:  ASDEX profiles similar to YU's \\
    20: kstar profiles with multiple q=1 surfaces \\
    21,22: fixed q(r) and p(r) profiles \\
    23:  Startsev equilibrium with $ J = (1/R_0q_0)(1 - r^2)$ \\
    27:  cylindrical test problem \\
    29:  basicJ profiles \\
  \end{minipage}\\
  \texttt{iupstream} & 0 & 1: addsdiffusion term to convection-like upstream differencing \\
  \texttt{magus}  & 5.e-2 & magnitude of the upstream diffusion term \\
  \texttt{iflip}    &  0 & 1: Flip coordinate system handedness\\
  \texttt{iflip\_b} &  0 & 1: Flip sign of toroidal field\\
  \texttt{iflip\_j} &  0 & 1: Flip sign of toroidal current\\
  \texttt{iflip\_v} &  0 & 1: Flip sign of toroidal velocity\\
  \texttt{iflip\_z} &  0 & 1: Flip equilibrium across z=0 plane \\
\end{tabular}

\begin{tabular}{llp{4in}}
  \textbf{Option}&\textbf{Default}&\textbf{Description}\\
  \hline
  \texttt{icsym}    &  0 &  
    \begin{minipage}[t]{2.5in}
    Symmetry of random perturbations \\
    0: No symmetry\\
    1: Odd up-down symmetry (in $U$)\\
    2: Even up-down symmetry (in $U$)
  \end{minipage}\\

  \texttt{bzero} & 1      & $B_\tor$ at \texttt{rzero}\\
  \texttt{bx0}  & 0 & Initial field in x-direction for some test problems \\
  \texttt{vzero} & 0 & Initial toroidal velocity for some test problems \\
  \texttt{phizero} & 0 & Initial velocity stream function for some test problems \\
  \texttt{v0\_cyl} & 0 & Central toroidal velocity for some test problems \\
  \texttt{v1\_cyl} & 0 & VZ=v0\_cyl + v1\_cyl*psin**beta \\
 \texttt{idevice}    &  0 &
    \begin{minipage}[t]{2.5in}
    define coils for a particular device \\
    -1: reads coil.dat file \\
    0: generic dipole configuration \\
    1: CDX-U \\
    2: NSTX \\
    3: ITER \\
    4: DIII-D
  \end{minipage}\\
  \texttt{iwave} & 0 & defines what wave to initialize in wave propagation test \\ 
  \texttt{eps}      &  0.01 & Size of random perturbation\\
  \texttt{maxn}     &  200 & Maximum wavenumber of initial random noise\\
  \texttt{verzero}  & 0 & magnitude of initial vertical velocity \\
   \texttt{irmp}    &  0 &
    \begin{minipage}[t]{2.5in}
    1: apply nonaxisymmetric fields throughout plasma.  
       reads rmp\_coil.dat for (R,Z) of window pane coils.  
       reads rmp\_current.dat for (+-) currents in kA and phases in degrees.  
       toroidal mode number of current specified by ntor.
       requires \texttt{type\_ext\_field} = 0.  \\
    2: apply nonaxisymmetric fields only at boundaries. 
   \end{minipage}\\
   \texttt{type\_ext\_field}    &  -1 & External field type
    \begin{minipage}[t]{2.5in}
    0: RMP or error field for tokamak geometry.
    1: For free boundary stellarator only: FIELDLINES or MGRID.
   \end{minipage}\\

 \texttt{rmp\_atten}  &  0  & additional exponential decay of RMP field from r=1 for irmp=2 \\
 \texttt{iread\_ext\_field} & 0 & 1: read external field \\
 \texttt{beta}  & 0 & parameter used in some model equilibrium initializations \\
 \texttt{ln} & 0 & length scale parameter used in some model equilibrium \\
 \texttt{elongation} & 1 & elongation used in Solovev equilibrium

\end{tabular}

\begin{tabular}{llp{4in}}
  \textbf{Option}&\textbf{Default}&\textbf{Description}\\
  \hline

 \texttt{isample\_ext\_field} & 1 & factor to down sample external field data toroidally \\
 \texttt{isample\_ext\_field\_pol} & 1 & factor do down sample external field data poloidally \\
 \texttt{scale\_ext\_field} & 1 & factor to scale external field \\
 \texttt{shift\_ext\_field} & 0 & toroidal shift (in deg) of external fields \\
 \texttt{ibasicj\_solvep} & 0 & 0: uniform p, solve for F; 1: uniformF, solve for p \\
 \texttt{basicj\_nu} &1 & exponent in basicj equilibrium \\
 \texttt{basicj\_j0} & 1 & On-axis current density in basicj equilibrium \\
 \texttt{basicj\_voff} & 1 & Radial extent of flat toroidal rotation in basicj equilibrium \\
 \texttt{basicj\_vdelt} & 1 & Width of velocity drop-off, as fraction of ln, in basicj equilibrium \\
 \texttt{basicj\_dexp} & 1 & parameter for basicj equilibrium \\
 \texttt{basicj\_dvac} & 1 & parameter for basicj equilibrium \\
 \texttt{basicj\_q0} & 0 &   parameter for basicj equilibrium \\
 \texttt{basicj\_qa}  & 0 &   parameter for basicj equilibrium \\
 \texttt{pf\_shift} & 0 & (array) horizontal shift of PF coil \\
 \texttt{pf\_shift\_angle} & 0 & (array) direction of PF shift in degrees \\
 \texttt{pf\_tilt} & 0 & (array) Angle of PF from vertical in degrees \\
 \texttt{pf\_tilt\_angle} & 0 & (array) Axis of rotation for PF tilt in degrees \\
 \texttt{tf\_shift} & 0 & horizontal shift of TF coils \\
 \texttt{tf\_shift\_angle} & 0 & direction of TF shift in degrees \\
 \texttt{tf\_tilt} & 0 & angle of TF from vertical in degrees \\
 \texttt{tf\_tilt\_angle} & 0 & axis of rotation for TF tilt in degrees 

\end{tabular}

\subsection{Grad-Shafranov Solver Options}
\begin{tabular}{lcp{4in}}
  \textbf{Option}&\textbf{Default}&\textbf{Description}\\
  \hline
  \texttt{inumgs}& 0      & 1: Use numerical def. of p and g from profile-p and profile-g files\\
 \texttt{igs}   & 80     & Max number of Picard iterations\\
  \texttt{eta\_gs} & 1000.& factor for smoothing nonaxisymmetries in psi in 3D GS solve \\
  \texttt{igs\_pp\_ffp\_rescale} & 0 & 1: rescale p' and FF' to match p and F \\
  \texttt{nv1equ}& 0 & 1:use numvar =1 equilibrium for numvar .GT. 1 \\
  \texttt{tcuro} & 1	  & (scaled) plasma current in GS equilibrium\\
  \texttt{xmag}  & 1      & $R$-coordinate of magnetic axis\\
  \texttt{zmag}  & 0      & $Z$-coordinate of magnetic axis\\
  \texttt{xmag0} & 0      &  if nonzero, target magnetic axis $R$ for feedback\\
  \texttt{zmag0} & 0      &  if nonzero, target magnetic axis $Z$ for feecback\\
  \texttt{xlim}  & 0      & $R$-coordinate of limiter\\
  \texttt{zlim}  & 0      & $Z$-coordinate of limiter\\
  \texttt{xlim2}  & 0      & $R$-coordinate of limiter \#2\\
  \texttt{zlim2}  & 0	   & $Z$-coordinate of limiter \#2\\
  \texttt{rzero}  & 1      & nominal major radius of device for itor=1 \\
  \texttt{libetap}& 1.2    & approximate value of $l_i/2 + \beta_P$ for free-boundary equ \\
  \texttt{p0}    & 0.01   & Pressure at magnetic axis\\
  \texttt{pi0}   & 0.005  & Ion pressure at magnetic axis\\
  \texttt{p1}    & 0     & $p^{\prime}(\Psi)$ at magnetic axis\\
  \texttt{p2}    & 0     & $p^{\prime \prime}(\Psi)$ at magnetic axis\\
  \texttt{pedge} & 0	  & Pressure in vacuum region\\
  \texttt{tedge} & 0     & temperature in vacuum region (if .GT. 0).  Only
                           used in GS solve.   Boundary value of electron temp
                           is $twall = pedge \times pefac/den\_edge $ \\
 \texttt{tiedge} & 0     & ion temperature in vacuum region  \\
 \texttt{expn}  & 0 & \parbox[t]{4in}{Fraction of pressure gradient due to
    density gradient: $n = p^\mathtt{expn}$.}\\
 \texttt{q0}    & 1	  & Safety factor at magnetic axis\\
 \texttt{djdpsi}& 0	  & $J_\tor'(\Psi)$ at magnetic axis\\
 \texttt{th\_gs}& 0.8     & implicitness of GS Picard iterations\\
 \texttt{tol\_gs}& $10^{-8}$  & convergence criteria for GS iteration \\
  \texttt{pscale}  & 1.       & factor multiplying pressure profile \\
  \texttt{bscale}      &  1.0 & Factor multipying toroidal field\\
  \texttt{bpscale}     &  1.0 & Factor multiplying F' (keeping F0 constant) \\
  \texttt{vscale}      &  1.0 & Factor multiplying toroidal rotation profile \\
  \texttt{iread\_bscale}&  0   & 1: read profile\_bscale for factor to scale F \\
  \texttt{iread\_pscale} & 0   & 1: read profile\_pscale for factor to scale $p$ and $p^{\prime} $ \\

\end{tabular}

\begin{tabular}{llp{4in}}
  \textbf{Option}&\textbf{Default}&\textbf{Description}\\
  \hline
  \texttt{batemanscale} &  1   & Bateman scale the TF, keeping curent profile fixed \\
  \texttt{irot}         &  0   & 1: include toroidal rotation in equilibrium calculatin \\
  \texttt{iscale\_rot\_by\_p} & 1 & see below \\
  \texttt{alpha0}       &  0   & $\alpha_0$ in analytic rotation profile \\
  \texttt{alpha1}      	&  0   & $\alpha_1$ in analytic	rotation profile \\
  \texttt{alpha2}      	&  0   & $\alpha_2$ in analytic	rotation profile \\
  \texttt{alpha3}      	&  0   & $\alpha_3$ in analytic	rotation profile \\
\end{tabular}

For iread\_omega=0, the function $\alpha(\psi)$ is parameterized by:
 \[ \tilde{\alpha} = \alpha_{0} + \alpha_{1} s + \alpha_{2} s^{2} + \alpha_{3} s^{3} \]
For iscale\_rot\_by\_p = 0:  $\alpha = \tilde{\alpha} \times n(\psi) / p(\psi)$ . \\
For iscale\_rot\_by\_p = 1:  $\alpha = \tilde{\alpha} $. \\
For iscale\_rot\_by\_p = 2:  $\alpha = \left[ \alpha_{0} + \alpha_{1} e^{-\left[ \left( \psi - \alpha_{2} \right) / \alpha_{3} 
                                          \right]^{2} }       \right] \times n(\psi) /p(\psi) $ \\
In all cases, the angular velocity is then determined by:
\[      \omega = \left[  \frac{2 \alpha p(\psi)}{R_0^2 n(\psi)} \right]^{\frac{1}{2}} \]
\begin{tabular}{llp{4in}}
  \texttt{idenfunc}         & 0   &
  \begin{minipage}[t]{4.0in}
    0: $ n = \mbox{den0} \times (p/p0)^{\mbox{expn}} + \mbox{denedge} $ \\
    1: $ n = \mbox{den0} \times \frac{1}{2} \times 
       \left[1 + \tanh \left(\frac{\psi - (\psi_B + n_O (\psi_B - \psi_M))}
                                  {\Delta \times (\psi_B - \psi_M)        } \right)    \right] $ \\
    2: $ n = \mbox{den0} + \frac{1}{2}  \left( \mbox{den\_edge} - \mbox{den0} \right)
                  \times  \left[1 + \tanh \left( \frac{ \tilde{\psi} - n_O}
                                               {      \Delta            } \right)  \right] $\\
    3: if $\tilde{\psi}$ .LT. $n_O$ and $(\psi - \psi_M) \times \left[d \psi /dx (x - x_{MA}) + d \psi /dz (z - z_{MA} )           
                \right] $ .GT. 0, then $n$ = den0.     Else, $n$ = den\_edge.  \\
    ( $\psi_B = \mbox{psibound}, \psi_M = \mbox{psimin}, \tilde{\psi} = (\psi - \psi_M)/(\psi_B - \psi_M) $ )
  \end{minipage}    \\

  \texttt{den\_edge}        & 0.0 & edge density.  If 0, set to den0*(pedge/p0)**expn \\
  \texttt{den0}             & 1.0 & (scaled) central density\\
  \texttt{denoff}           & 1.0 & $n_O$: offset for idenfunc=1,2,3 \\
  \texttt{dendelt}          & 0.1 & $\Delta$: width of transition region for idenfunc=1,2 \\

  \texttt{divertors} & 0  & Number of divertors (0--2)\\
  \texttt{divcur}& 0.1    & Divertor current(s), as fraction of tcuro\\
  \texttt{xdiv}  & 0      & $r$-coordinate of divertor current(s)\\
  \texttt{zdiv}  & 0      & \parbox[t]{4in}{$z$-coordinate of
    divertor 
    current.  If $\mathtt{divertors} = 2$, the second divertor has 
    $z = -\mathtt{zdiv}$.}\\
  \texttt{xnull}     & 0 & Guess for $r$-coordinate of x-point\\
  \texttt{znull}     & 0 & Guess for $z$-coordinate of x-point\\
  \texttt{mod\_null\_rs} & 0 & if 1: you can reset xnull and znull from C1input \\
  \texttt{xnull0}  &  0  & Target R-Coordinate of x-point for feedback \\
  \texttt{znull0}  &  0  & Target Z-Coordinate of x-point for feedback \\
  \texttt{xnull2}     & 0 & Guess for $r$-coordinate of inactive x-point\\
  \texttt{znull2}     & 0 & Guess for $z$-coordinate of inactive x-point\\
  \texttt{mod\_null\_rs2} & 0 & if1: you can reset xnull2 and znull2 from C1input \\
  \end{tabular}

  \begin{tabular}{llp{4.0in}}
  \textbf{Option}&\textbf{Default}&\textbf{Description}\\
  \hline
 \texttt{gs\_pf\_psi\_width}            & 0 & width of psi smoothing into provate flux region \\
 \texttt{gs\_vertical\_feedback}        & 0 & proportional feedback of each coil to (zmag-zmag0) (array) \\ 
 \texttt{gs\_vertical\_feedback\_i}     & 0 & integral feedback of each coil to (zmag-zmag0) (array) \\
 \texttt{gs\_vertical\_feedback\_x}     & 0 & proportional feedback of each coil to (znull-znull0) (array) \\
 \texttt{gs\_vertical\_feedback\_x\_i} & 0 & integral feedback of each coil to (znull-znull0) (array) \\
 \texttt{gs\_radial\_feedback}          & 0 & proportional feedback of each coil to (xmag-xmag0) (array) \\
 \texttt{gs\_radial\_feedback\_i}       & 0 & integral feedback of each coil to (xmag-xmag0) (array) \\
 \texttt{gs\_radial\_feedback\_x}       & 0 & proportional feedback of each coil to (xnull-xnull0) (array) \\
 \texttt{gs\_radial\_feedback\_x\_i}    & 0 & integral feedback of each coil to (xnull-xnull0) (array) \\
 \texttt{igs\_extend\_p}                & 0 & 1: extend p past pls=1 using ne and Te profiles  \\
 \texttt{igs\_feedfac}                  & 1 & proportionality factor for external field feedback \\
 \texttt{igs\_forcefree\_lcfs}          & -1 & 1: ensure that GS solution is force free at LCFS \\
 \texttt{igs\_start\_xpoint\_search}    & 0 &  number of GS iterations before searching for x-point \\
 \texttt{sigma0}                        & 0 &  width of Gaussian for initial current distribution for GS iteration \\
 \texttt{igs\_extend\_diagmag}          & 1 &  1: extend diamagnetic rotation past psi=1 \\
 
\texttt{adapt\_qs} & 0 & Safety factor values to pack around (array) \\
\texttt{adapt\_zlow} & 0 & Z-coordinate below which SOL adaption is coarse \\
\texttt{adapt\_zup}  & 0 & Z-coordinate above which SOL adaptation is coarse


\end{tabular}




\subsection{Transport Coefficients}

\begin{tabular}{llp{4.0in}}
  \textbf{Option}&\textbf{Default}&\textbf{Description}\\
  \hline
  \texttt{ivisfunc} & 0 & select viscosity function \\
                    &   & 0: $ \mbox{visc} = \mbox{amu} $ \\
                    &   & 1: $ \mbox{visc} = \mbox{amu} + \frac{1}{2} \mbox{amu\_edge} \times
                 \left[ 1. + \tanh \left[   \frac{\psi - \left( \psi_l+\nu_0(\psi_l - \psi_0)\right)}
                                                 {\nu_{\Delta} (\psi_l - \psi_0)                   }\right] \right] $  \\
                    &   & 2:  $ \mbox{visc} = \mbox{amu} + \frac{1}{2} \mbox{amu\_edge} \times
                 \left[ 1. + \tanh \left[   \frac{\tilde{\psi} - \nu_0}
                                                 {\nu_{\Delta}}  \right] \right] $  \\
                    &   & or, if amuoff2 .ne. 0 and amudelt2.ne.0) \\ 
                    &   &    $ \mbox{visc} = \mbox{amu} + \frac{1}{4} \mbox{amu\_edge} \times
                     \left[ 2. + \tanh \left[   \frac{\tilde{\psi} - \nu_0}
                                                 {\nu_{\Delta}}  \right]                     
                               + \tanh \left[   \frac{\tilde{\psi} - \nu_{02} }
                                                 {\nu_{\Delta2}}  \right]    \right] $  \\
                    &   & 3:  visc = amu or amu\_edge depending on criteria in define\_fields \\

  \texttt{amu}       & 0 & core viscosity for ivisfunc =0,..,3 \\
  \texttt{amu\_edge} & 0 & edge viscosity for ivisfunc = 1,..,3 \\
  \texttt{amuoff}    & 0 & $\nu_0$ in ivisfunc = 1,2 \\
  \texttt{amuoff2}   &   & $\nu_{02}$ in ivisfunc = 1,2 \\
  \texttt{amudelt}   & 0 & $\nu_{\Delta}$ in ivisfunc = 1,2 \\
  \texttt{amudelt2}  & 0 & $\nu_{\Delta2}$ in ivisfunc = 1,2 \\
  \texttt{amuc}   & 0 & Compressional viscosity coefficient\\
  \texttt{amupar} & 0 & Parallel viscosity coefficient \\
  \texttt{amue}  & 0 & bootstrap viscosity coefficient \\
  \hline
  \texttt{iresfunc} & 0 & select resistivity function  \\
                    &   & 0: eta = etar + eta0/Te**(3/2) \\
                    &   & 1: $ \mbox{eta} = \mbox{etar} + \frac{1}{2} \mbox{eta0} \times
                 \left[ 1. + \tanh \left[   \frac{\psi - \left( \psi_l+\mbox{etaoff} \times (\psi_l - \psi_0)\right)}
                                                 {\mbox{etadelt}\times (\psi_l - \psi_0)                   }\right] \right] $  \\
                    &   & 2:  $ \mbox{eta} = \mbox{etar} + \frac{1}{2} \mbox{eta0} \times
                 \left[ 1. + \tanh \left[   \frac{\tilde{\psi} - \mbox{etaoff}}
                                                 {\mbox{etadelt}}  \right] \right] $  \\
                    &   & The following two options are applied in a way that they \\ 
                    &   & should not have negative values...even if the idl plots \\
                    &   & indicate otherwise \\
                    &   & 3: eta = etar for $\tilde{\psi} < \mbox{etaoff}$ othrwise eta0  \\
                    &   & 4:  Spitzer resistivity with offset. \\
                    &   &     Define $T_{wall}$ = pedge*pefac/den\_edge \\
                    &   & $\mbox{for} T_e > T_{wall} - T_e^{off}, \eta = (T_e-T_e^{off})^{-3/2} $ \\
                    &   & $\mbox{for} T_e < T_{wall} - T_e^{off}, \eta = (T_{wall}-T_e^{off})^{-3/2} $ \\
                    &   & can be increased by inputing eta\_fac $>$ 1.  \\
                    &   & 5:  Simple neoclassical model:  \\
                    &   & $\eta = \mbox{eta0} \times (n_e/p_e)^{3/2} / (1. - 1.46 (r/R)^{1/2})   $ \\
 \texttt{etar}    & 0 & see description of iresfunc \\
  \texttt{eta0}   & 0 & see description of iresfunc \\
  \texttt{etaoff} & 0 & see description of iresfunc \\
  \texttt{etadelt} & 0 & see description of iresfunc \\
  \texttt{eta\_te\_offset} & 0 & $T_e^{off}$  for iresfunc=4 \\
  \texttt{ikprad\_te\_offset} & 0 & if 1, $T_e^{off}$ also used in kprad and ablation \\
  \texttt{eta\_fac} & 0 & for iresfunc=4, resistivity multiplied by eta\_fac \\
  \texttt{eta\_mod} & 0 & if 1: remove d/dphi terms in resistivity \\
\end{tabular}

\begin{tabular}{llp{4.0in}}
  \textbf{Option}&\textbf{Default}&\textbf{Description}\\
  \hline
  \texttt{eta\_max} & 0 & maximum resistivity in plasma (defaults to etavac) \\
  \texttt{eta\_min} & 0 & minimum resistivity in plasma  \\
  \hline
  \texttt{ikappafunc} & 0 &  select thermal conductivity function \\
    & & 0: $\kappa = \mbox{kappat} + \mbox{kappa0} \times * (n^3/p)^{1/2}  $\\
    & & 1: $\kappa = \mbox{kappa0} \times \frac{1}{2}
                    \left[1 + \tanh \left[\frac{\psi -\left( \psi_l + \kappa^{0ff} \times (\psi_l - \psi_0)\right)}
                                               {\kappa_{\Delta} \times (\psi_l - \psi_0)} \right]  \right]$ \\
    & & 2: $\kappa = \mbox{kappa0} \times \frac{1}{2}
                    \left[  1 + \tanh \frac{\tilde{\psi} - \kappa^{off}}
                                           {  \kappa_{\Delta}}  \right] \mbox{for} \tilde{\psi} < 1 $  \\
    & & 2: $\kappa = \mbox{kappa0} \times \frac{1}{2}
                    \left[  1 + \tanh \frac{2 - \tilde{\psi} - \kappa^{off}}
                                           {  \kappa_{\Delta}}  \right] \mbox{for} \tilde{\psi} > 1 $  \\
    & & 3:  $\kappa = \mbox{kappat} + \mbox{kappa0} \times 1/(pn)^{1/2}$ \\
    & & 4:  $\kappa = \mbox{kappat} + \mbox{kappa0} \times ( 1. + \mbox{kappadelt} \times |\nabla T_e|^2 ) $  \\
    & & 5:  $\kappa = \mbox{kappat} + \mbox{kappa0}/T_e$ limited by kappa\_max  \\ 
    & & 10: read from profile\_kappa file in $m^2/\mbox{sec}$  \\ 
    & & 11: read from profile\_kappa file in normalized units  \\
    & & 12: option to go with itaylor=27   \\
 \texttt{kappa\_max}  & 0 & if .NE. 0, max $\kappa$ for ikappafunc=5 \\
 \texttt{kappai\_fac} & 1 & ion thermal conduction is kappai\_fac* kappa \\
 \texttt{ikapscale}   & 0 & if 1: kappar gets scaled by kappa  \\
 \texttt{ikappar\_ni} & 0 & 1: include 1/n terms in parallel heat flux \\
 \texttt{kappaoff}    & 0 & $\kappa^{off}$ see ikappafunc \\
 \texttt{kappadelt}   & 0 & $\kappa_{\Delta} $ see ikappafunc \\
 \texttt{kappat}      & 0 & isotropic thermal conductivity \\
 \texttt{kappa0}      & 0 & see ikappafunc \\
 \texttt{ikapparfunc} & 0 &  select parallel thermal conductivity (PTC) function \\
                      &   & 0: PTC = kappar  \\
                      &   & 1: PTC = $\mbox{kappar}/  \left[ (T_{crit}/T)^{5/2} + 1   \right] $ \\
 \texttt{kappar} & 0 & Parallel thermal conductivity\\
 \texttt{tcrit}  & 0 & $T_{crit}$ for ikapparfunc = 1  \\

 \texttt{kappari\_fac}& 1 & ion parallel thermal conductivity is kappari\_fac x electron value \\
 \texttt{kappax}      & 0 & coefficient of $B \times \nabla T$ temperature diffusion \\
 \texttt{kappah}      & 0 & if nonzero, $\mbox{kappa} = \mbox{kappah} \times \tanh^2 \left[ (\tilde{\psi}-1.)/2  \right] $ \\
  \texttt{kappaf} & 1 & Factor multiplying kappa when $\nabla p < \nabla p_{crit} $ \\
  \texttt{kappag} & 0 & Thermal diffusion proportional to pressure gradient \\
  \texttt{gradp\_crit} & 0 & $\nabla p_{crit} $ for kappaf,kappag model \\
  \texttt{k\_fac} & 1 & Factor by which TF is multiplied in denominator of kappa\_par \\
  \texttt{temin\_q0}  & 0 & Min temperature used in equipartition for ipres=1 \\
  \hline
  \texttt{idenmfunc}   & 0 & selects from of particle diffusion (PD) \\
                      &   &  0: PD = denm \\
                      &   &  1: PD = denm + denmt/Te \\
                      &   &  10: read from file profile\_denm in $m^2/\mbox{sec}$  \\
                      &   &  11: read from file profile\_denm in normalized units \\
  \texttt{denm}   & 0 &   see idenmfunc \\
  \texttt{denmt}  & 0 &  multiplier of 1/Te for idenmfunc=1 \\
  \texttt{denmmin} & 0 & minimum value of denm \\
  \texttt{denmmax} & 1.E6 & maximum value of denm \\
 
\end{tabular}

\subsection{Hyper-Diffusivity}

\begin{tabular}{lcp{4.0in}}
  \textbf{Option} & \textbf{Default} & \textbf{Description}\\
  \hline
  \texttt{imp\_hyper}  & 0 & switch for evaluating hyper resistivity \\
                       &   & 0: $\lambda_H \nabla^2 {\bf J} $ explicit for $\psi$ implicit for F  \\                 \\
                       &   & 1: $\lambda_H \nabla^2 {\bf J} $ implicit for $\psi$ implicit for F  \\                
                       &   & 2: $ ({\bf B}/B^2)\nabla \bullet \lambda_H \nabla \sigma $ implicit for $\psi$ and F    
                                   ($\sigma= {\bf J \bullet B}/B^2 $)  \\                \\
 \texttt{deex}         & 1 & scale length used in hyper \\
 \texttt{hyper}        & 0 & hyper coefficient for $\psi$ equation \\
 \texttt{hyperc}       & 0 & hyper coefficient for poloidal velocity   \\
 \texttt{hyperi}       & 0 & hyper coefficient for toroidal field   \\
 \texttt{hyperp}       & 0 & hyper coefficient for pressure   \\
 \texttt{hyperv}       & 0 & hyper coefficient for toroidal flow   \\
 \texttt{ihypdx}       & 2 & hyper terms multiplied by deex**ihypdx \\
 \texttt{ihypeta}      & 1 & swithc for multipliers of hyper terms   \\
                       & 1 & magnetic field hyper multiplied by eta  \\
                       & 2 & magnetic field hyper multiplied by p\\
 \texttt{ihypamu}            & 1 & 1: velocity hyper coefficient multiplied by amu \\
 \texttt{ihypkappa}          & 1 & 1: pressure hyper coefficient multiplied by kappa \\
\end{tabular}

\subsection{Unit Normalizations}
\begin{tabular}{lcp{4in}}
  \textbf{Option}&\textbf{Default}&\textbf{Description}\\
  \hline
  \texttt{n0\_norm} & $10^{14}$ & Density normalization (in cgs)\\
  \texttt{b0\_norm} & $10^4$    & Magnetic field normalization (in cgs)\\
  \texttt{l0\_norm} & $100$     & Length normalization (in cgs)
\end{tabular}

\subsection{Boundary Conditions}

\begin{tabular}{lcp{4.0in}}
  \textbf{Option} & \textbf{Default} & \textbf{Description}\\
  \hline
 
  \texttt{isurface} & 1 & include surface terms in Galerkin method \\
  \texttt{icurv}    & 2 & if $>$ 0, include curvature from mesh \\
  \texttt{nonrect}  & 0 & 1: non-rectangular boundary \\
  \texttt{ifixedb} & 0 & Set $\psi=0$ on boundary\\
  \texttt{inonormalflow}& 1 & 1: No-normal-flow boundary\\
  \texttt{inoslip\_pol} & 0 & 1: No-slip boundaries for poloidal velocity\\
  \texttt{inoslip\_tor} & 1 & 1: No-slip boundaries for toroidal velocity\\
  \texttt{inostress\_tor}&0 & 1: No-normal-stress boundary for toroidal 
                                 velocity\\
  \texttt{iconst\_bz} & 1 & 1: Toroidal field held constant on boundary\\
  \texttt{iconst\_bn} & 1 & 1: Hold normal field constant on boundary \\
  \texttt{iconst\_n}  & 0 & 1: Density held constant on boundary\\
  \texttt{iconst\_p}  & 1 & 1: Pressure held constant on boundary\\
  \texttt{iconst\_t}  & 0 & 1: Temperature held constant on boundary\\
  \texttt{inograd\_p} & 0 & 1: No normal pressure gradient\\
  \texttt{inograd\_t} & 0 & 1: No normal temperature gradient\\
  \texttt{inograd\_n} & 0 & 1: No normal density gradient \\
  \texttt{com\_bc}& 0 & 1: $\nabla^2 \chi = 0$\\
  \texttt{vor\_bc}& 0 & 1: $\Delta^* U = 0$\\
  \texttt{inocurrent\_pol} & 0 & 1: no poloidal current on boundary \\
  \texttt{inocurrent\_tor} & 0 & 1: no toroidal current on boundary \\
  \texttt{inocurrent\_norm}& 0 & 1: no normal current on boundary \\
  \texttt{ifbound}  & -1 & boundary condition on $f\prime$  \\
                    &    & 1: Dirichlet  \\
                    &    & 2: Neumann    \\
  \texttt{iconstflux}  & 0 & 1: conserves toroidal flux in nonlinear calculation \\
  \texttt{tebound}     & -1 & boundary condition for electron temperature \\
  \texttt{tibound}     & -1 & boundary condition for ion temperature \\
  \texttt{iper}   & 0 & 1: Left/right boundaries periodic\\
  \texttt{jper}   & 0 & 2: Top/bottom boundaries periodic\\

\end{tabular}



\subsection{Time Step}
\begin{tabular}{lcp{4in}}
  \textbf{Option}&\textbf{Default}&\textbf{Description}\\
  \hline
  \texttt{dt}         & 0.1 & Initial size of ime step\\
  \texttt{ntimemax}   & 20  & Total number of time steps\\
  \texttt{integrator} & 0   & 0: Crank-Nicholson (CN); 1: BDF2\\
  \texttt{imp\_mod}   & 0   & 
  \begin{minipage}[t]{4in}
    0: $\theta$-implicit\\
    1: Implicit leapfrof (\texttt{isplitstep} = 1 only)\\
  \end{minipage}\\
  \texttt{thimp}      & 0.5 & Implicitness parameter\\
  \texttt{thimpsm}    & 1   & Implicitness of the smoother functions\\
  \texttt{isplitstep} & 1   & 0: Fully implicit time step; 
                              1: split time step.\\
  \texttt{iteratephi} & 0   & 1: Calculate transport coefficients after
    field advance, then recalculate field advance. \\
  \texttt{idiff} & 0 & 1: solve for difference between n and n+1 in B,p \\
  \texttt{idifv} & 0 & 1: solve for difference between n and n+1 for V \\
  \texttt{irecalc\_eta} & 0 & 1:recalculate transport coefficients after density solve \\
  \texttt{iconst\_eta}  & 0 & 1" don't evolve resistivity \\
  \texttt{itime\_independent} & 0 & 1:exclude d/dt terms \\
  \texttt{harned\_mikic}  & 0 & coefficient of Harned-Mikic 2F stabilization term \\
  \texttt{isources}       & 0 & 1:include source terms in velocity advance \\
  \texttt{nskip}   & 1 & number of time steps per matrix recalculation \\
  \texttt{pskip}   & 1 & number of timesteps the preconditioner is reused \\
  \texttt{iskippc} & 1 & number of times preconditioner is reused \\
  \texttt{ddt}     & 0 & change in timestep per timestep \\
  \texttt{frequency} & 0 & frequency in time-independent calculation \\
  \texttt{dtmin}     & 4.0 & minimum timestep for variable timestep calculation \\
  \texttt{dtmax}     & 40. & maximum timestep for variable timestep calculation \\
  \texttt{dtkecrit}  & 0 &  lower timestep if ekin es above this (0.01 typical) \\
  \texttt{ dtfrac}   & .10 & max fractional change of timestep in 1 cycle \\
  \texttt{max\_repeat}  & 3 & max number time step is repeated for ksp\_max iterations exceeded \\
  \texttt{ksp\_max}     & 10000 & max number of ksp iterations before repeating time step \\
  \texttt{ksp\_min}     & 1200  & increase dt if ksp < ksp\_min  \\
  \texttt{ksp\_warn}    & 1600  & decrease dt if ksp > ksp\_warn \\
\end{tabular}

\subsection{Mesh}

\begin{tabular}{lcp{4.0in}}
  \textbf{Option} & \textbf{Default} & \textbf{Description}\\
  \hline
  \texttt{nplanes}& 1 & number of toroidal planes for 3D nonlinear \\
  \texttt{xzero}  & 0 & $R$-coordinate of bottom left corner of domain\\
  \texttt{zzero}  & 0 & $z$-coordinate of bottom left corder of domain\\
  \texttt{tiltangled} & 0 & angle a rectangular mesh is tilted \\
  \texttt{mesh\_model}&   & model file name from which the mesh is generated \\
  \texttt{mesh\_filename} & & mesh name of .smb files (without number) \\
  \texttt{imatassemble}  & 0 & 1: use petsc matrix parallel assembly instead of scorec \\
  \texttt{imulti\_region} & 0 & 1: Mesh has multiple regions that include resistive wall and
                               vacuum.   Wall resistivity is "eta\_wall", vacuum
                               resistivity is "eta\_vac"  \\
  \texttt{toroidal\_pack\_angle} & 0 & toroidal angle of maximum mesh packing \\
  \texttt{toroidal\_pack\_factor} & 1 & ratio of longest to shortest toroidal element \\

  \texttt{iread\_vmec}    & 0 & 1: read geometry from VMEC file \\
  \texttt{vmec\_filename} & geometry.nc& name of vmec data file \\
  \texttt{nperiods} & 1 & model 1/nperiods of a torus when ifull\_torus=0 \\
  \texttt{iread\_planes} & 0 & Read positions of toroidal planes from plane\_positions \\
  \texttt{bloat\_distance} & 0 & factor to expand VMEC domain \\
  \texttt{bloat\_factor}   & 0 & factor to expand VMEC domain \\
  \texttt{ifull\_torus}    & 0 & 0: one field period, 1: full torus \\
  \texttt{igeometry}       & 0 & 0: default, identity \\
  \texttt{xcenter}         & 0 & center of logical mesh (x) \\
  \texttt{zcenter}         & 0 & center of logical mesh (z) \\
  \texttt{bound\_type(i)}   & 0 & Boundary conditions to apply on mesh loop \texttt{i}.  0 = None, 1 = First wall, 2 = Domain boundary\\
  \texttt{zone\_type(i)}    & 0 & Physics model of mesh zone \texttt{i}.  1 = plasma, 2 = conductor, 3 = vacuum.\\
\end{tabular}

\subsection{Solver}
\begin{tabular}{lcp{4in}}
  \textbf{Option}&\textbf{Default}&\textbf{Description}\\
  \hline
  \texttt{solver\_type} & 0 &  for PETSc only,  0: direct solve, 1: iterative solver.   
                               for Trilinios, iterative solver is used \\
  \texttt{solver\_tol}  & 1.e-9 & solver tolerance \\
  \texttt{num\_iter}    & 100 &   maximum number of iterations 
\end{tabular}

\subsection{Mesh Adaptation (will be depricated soon)}
\begin{tabular}{llcp{4in}}
\textbf{Option}&\textbf{Default}&\textbf{Description}\\
  \hline
 \texttt{iadapt} & 0 & 0: no adaptation \\
                 &   & 1:adapt mesh from the flux field in equilibrium
\end{tabular}


\subsection{Numerical Options}
\begin{tabular}{llp{4in}}
  \textbf{Option}&\textbf{Default}&\textbf{Description}\\
  \hline
 \texttt{int\_pts\_main}  & 25 & Sampling points for integrations in
                                main time step matrices\\
  \texttt{int\_pts\_aux}   & 25 & Sampling points for integrations in
                                calculations of auxiliary variables\\
  \texttt{int\_pts\_diag}  & 25 & Sampling points for integrations in
                                diagnostic calculations\\
  \texttt{int\_pts\_tor}   & 5 & Max number of toroidal integration points \\
  \texttt{ivform} & 0   & 0: $\u = \grad{U}\times\grad{\tor} + V
   \grad{\tor} + \grad{\chi}$\\
   & & 1: $\u = \r^2 \grad{U}\times\grad{\tor} + \r^2 \omega
   \grad{\tor} + \r^{-2} \grad{\chi}$\\
   & & Now depricated to ivform=1 only \\
  \texttt{jadv}   & 0   & 1: Use toroidal current density equation
                          instead of poloidal flux equation.\\
  \texttt{max\_ke}& 1.0  & Maximum value of kinetic energy before solution is
                          rescaled in linear simulations. (0 = don't rescale)\\
  \texttt{equilibrate} & 0 & 1: scale trial function so L2 norm = 1 \\
  \texttt{regular}     & 0 & regularization constant in chi equation \\
  \texttt{iset\_pe\_floor} & 0 & 1: do not let pe drop below pe\_floor \\
  \texttt{pe\_floor}       & 0 & minimum value for pe when iset\_pe\_floor=1 \\

 \texttt{iset\_pi\_floor} & 0 & 1: do not let pi drop below pe\_floor \\
  \texttt{pi\_floor}	   & 0 & minimum value for pi when iset\_pi\_floor=1 \\

 \texttt{iset\_te\_floor} & 0 & 1: do not let te drop below te\_floor \\
  \texttt{te\_floor}	   & 0 & minimum value for te when iset\_te\_floor=1 \\

 \texttt{iset\_ti\_floor} & 0 & 1: do not let ti drop below ti\_floor \\
  \texttt{ti\_floor}	   & 0 & minimum value for ti when iset\_ti\_floor=1 \\

 \texttt{iset\_ne\_floor} & 0 & 1: do not let ne drop below ne\_floor \\
  \texttt{ne\_floor}	   & 0 & minimum value for ne when iset\_ne\_floor=1 \\

 \texttt{iset\_ni\_floor} & 0 & 1: do not let ni drop below ni\_floor \\
  \texttt{ni\_floor}	   & 0 & minimum value for ni when iset\_ni\_floor=1 \\




  \texttt{iprecompute\_metric} & 0 & 1: precompute metric temsor
\end{tabular}

\subsection{Input Options}

\begin{tabular}{lcp{4in}}
  \textbf{Option}&\textbf{Default}&\textbf{Description}\\
  \hline
  \texttt{iread\_eqdsk}   & 0 & 1: Read EFIT g-file 'geqdsk'\\
                          &   & 2: Read psi from geqdsk, but use analytic profiles for p and F \\
                          &   & 3: read profiles from geqdsk, but not the psi \\
  \texttt{iread\_dskbal}  & 0 & 1: Read BAL file 'dskbal'\\
  \texttt{iread\_jsolver} & 0 & 1: Read Jsolver file 'fixed' \\

 \texttt{iread\_omega}       & 0 & 1: reads in rotation profile \\
 \texttt{iread\_omega\_ExB}  & 0 & 1: read ExB rotation \\
 \texttt{iread\_omega\_e}    & 0 & 1: read electron rotation \\
 \texttt{iread\_ne}          & 0 & 1: read in electron  density profile \\
 \texttt{iread\_te}          & 0 & 1: read in temperature profile \\
 \texttt{iread\_p}           & 0 & 1: read pressure profile from profile\_p \\
 \texttt{iread\_neo}         & 0 & 1: read velocity profiles from NEO output \\
 \texttt{ineo\_subtract\_diamag} & 0 & 1: subtract diamagnetic term from input vel when reading neo velocity \\
 \texttt{iread\_heatsource}      & 0 & 1: read heat source profile (psi normalized) scaled by ghs\_rate \\
 \texttt{iread\_partilesource}   & 0 & 1: read particle source profile (psi normalized) scaled with pellet\_rate \\
 \texttt{iread\_f}               & 0 & 1: read R x BT from file \\
 \texttt{iread\_j}               & 0 & 1: read current density from a file





\end{tabular}

\subsection{Output Options}

\begin{tabular}{lcp{4in}}
  \textbf{Option}&\textbf{Default}&\textbf{Description}\\
  \hline
  \texttt{ntimepr}   & 5 & Number of timesteps per full field output\\
  \texttt{iprint}    & 0 & 1: Print detailed info to stdout\\
 \texttt{ifout}      & -1 & 1: ourput f field \\
 \texttt{idouble\_out} & 0 & 1: use double precision floating in putput hdf5 files \\
 \texttt{itemp\_plot}  & 0 & 1: output vdotgradt, deldotq\_perp, deldotq\_par, eta\_jsq \\
 \texttt{ibdgp}        & 0 & optons for plotting partial terms for bdgp plot \\
                       &   & (1) $[ \psi, \Phi] $, (2) $(f\prime,\Phi) $, (3) $-R^{-2} F \Phi \prime$ \\
 \texttt{iveldif}      & 0 & option for plotting partial terms for veldif plot \\
                       &   & (1) $|/psi,U|$, (2) $R^{-3}(\psi,chi) + R^{-2}|\chi,f\prime|$ \\
                       &   & (3)$ R^{-1} \Phi \prime$   (4) $R^{-1} \Phi \prime + R^{-1} F U\prime $ \\
                       &   & (5) $|\psi,U| + R(U,f\prime) - R^{-1}F U\prime $ (6) $R^{-2} |/chi,f\prime| $ \\
 \texttt{icalc\_scalars} & 1 & 1 : calculate scalar diagnostics \\
 \texttt{ike\_only}      & 0 & 1 " only calculate ke scalar diagnostic \\
 \texttt{ike\_harmonics} & 0 & number of toroidal harmonics of kinetic energy to be calculated for diagnostics \\
 \texttt{ibh\_harmonics} & 0 & number of toroidal harmonics of magnetic energy to be calculated for diagnostics \\
 \texttt{irestart}       & 0 & 0: start from timestep 0 \\
                         &   & 1: normal restart (can restart 3D from 2D) \\
                         &   & 3: start a 2D complex run from a 2D real restart \\
 \texttt{irestart\_slice}  & -1 & if set to an integer, restart from that time slice \\
 \texttt{itimer}           & 0 & 1: output internal timeing data \\
 \texttt{iwrite\_transport\_coefs}  & 1 & 1: output transport coefficients fields \\
 \texttt{iwrite\_aux\_vars}         & 1 & 1:output auxiliary variable fields \\

\end{tabular}


\subsection{Diagnostics}

\begin{tabular}{lcp{4in}}
  \textbf{Option}&\textbf{Default}&\textbf{Description}\\
  \hline
 \texttt{xray\_detector\_enabled} & 0 &  1:  enable xray detector \\
 \texttt{xray\_r0}     & 0 &  R coordinate of xray detector \\
 \texttt{xray\_phi0}   & 0 &  phi coordinate of xray detector \\
 \texttt{xray\_z0}     & 0 &  Z coordinate of xray detector \\
 \texttt{xray\_theta}  & 0 & angle of xray detector chord (degrees) \\
 \texttt{xray\_sigma} & 1 &  spread of xray detector chord (degrees) \\
 \texttt{imag\_probes}& 0 &  number of magnetic probes \\
 \texttt{mag\_probe\_x(i)} & 0 &    R coordinate of magnetic probe i \\
 \texttt{mag\_probe\_phi(i)} & 0 & phi coordinate of magnetic probe i \\
 \texttt{mag\_probe\_z(i)}   & 0 & Z coordinate of magnetic probe i \\
 \texttt{mag\_probe\_nx(i)}  & 0 & R component of normal vector of mag probe i \\
 \texttt{mag\_probe\_nphi(i)}& 0 & phi component of normal vector of mag probe i \\
 \texttt{mag\_probe\_nz(i)}  & 0 & Z component of normal vector of mag probe i \\
 \texttt{iflux\_loops}       & 0 & number of flux loops \\
 \texttt{flux\_loop\_x(i)}   & 0 & R coordinate of flux loop i \\
 \texttt{flux\_loop\_z(i)}   & 0 & Z coordinate of flux loop i \\
 \texttt{ifixed\_temax}      & 0 & 1: temax evaluated at (xmag0,0,zmag0)

\end{tabular}
\subsection{Sources/Sinks}

\begin{tabular}{llp{4.5in}}
  \textbf{Option}&\textbf{Default}&\textbf{Description}\\
  \hline
  \texttt{ibeam}    & 0 &  neutral beam source \\
                    &   &  1:include neutral beam particle, energy, and momentum source: \\
                    &   &  $S = \frac {nb_n  }{4 r \pi^2 n b_{dr}  }
                            \exp - \left[ (r - nb_r)^2 + (z - nb_z)^2 \right]/ 2 nb_{dr}^2$  \\
                    &   &  2:include only particle and energy source  \\
                    &   &  3:include only energy source  \\
                    &   &  4:include only momentum and energy source  \\
                    &   &  5:include only momentum source  \\
  \texttt{beam\_x}  & 0 & R coordinate of beam center (in m) \\
  \texttt{beam\_z}  & 0 & Z coordinate of beam center (in m) \\
  \texttt{beam\_v}  & 1.e4 & beam voltage (in volts) \\
  \texttt{beam\_rate}  & 0 & ions/second deposited by beam \\
  \texttt{beam\_dr}    & .1 & dispersion of beam deposition \\
  \texttt{beam\_dv}    & 100. & dispersion of beam voltage (in volts) \\                        
  \texttt{beam\_fracpar}  & 1 & cosine of beam angle relative to parallel (for momentum source)   \\
\hline                  
  \texttt{vloop}          & 0 & initial loop voltage,   NOTE: to change vloop at restart time, must have control\_type = -1 \\
  \texttt{tcur}           & 0 & target (scaled) plasma current for current control: $\mu_0 I_P$ \\
  \texttt{tcuri}          & 0 & if tcuri .ne. tcurf, the target current is a function of time \\
  \texttt{tcurf}          & 0 & tcur = tcuri + (tcurf-tcuri) x .5 x (1. + tanh((t-tcur\_t0)/tcur\_tw)) \\
  \texttt{tcur\_t0}       & 0 & see tcurf \\
  \texttt{tcur\_tw}       & 0 & see tcurf \\                                       
  \texttt{control\_type}  & -1. &  0 old current control algorithm (not recommended) \\
                          &     &  1: standard PID  control with the following control parameters \\                                                        
  \texttt{control\_p}     & 0 & proportional control coefficient \\
  \texttt{control\_i}     & 0 & integral control coefficient \\                 
  \texttt{control\_d}     & 0 & derivative control coefficient  \\

  \hline
  \texttt{ipellet}      & 0    & density source if non-zero (3D part equals 1 for 2D runs).  
                                 Double-digit values have volume integrals normalized to 1.  
                                 Make netative for initial perturbation only. \\
                        &   & Define:  $G_{2D} = \frac{1}{2 \pi R  V_P^2}\exp \left[ - \frac{ (R - R_P)^2 + (Z - Z_P)^2}{2 V_P^2}
                                                                                \right]$  \\
                        &   & 1: $ S = G_{2D} \times \frac{R}{\sqrt{2 \pi}V_t} 
                                  \exp \left[- \frac{RR_P\left(1 - \cos(\phi - \phi_P)\right)}{V_t^2} \right] $  \\
                        &   & 2: $ S = \mbox{den0} \times \left( \mbox{max} (p , p_{edge} )/p_0 \right)^{expn} $ 2D and 3D \\
                        &   & 3: Gaussian sourrce proportional to pressure \\
                        &   & $ S = p \times G_{2D} \times \frac{R}{\sqrt{2 \pi}V_p}
                                  \exp \left[- \frac{RR_P\left(1 - \cos(\phi - \phi_P)\right)}{V_p^2} \right] $  \\
                        &   & 4: Same distribution as ipellet=1 in 3D \\
                        &   & $ S = \sqrt{2 \pi} R V_p \times G_{2D} \times \frac{1}{2 \pi V_p V_t}
                                  \exp \left[- \frac{RR_P\left(1 - \cos(\phi - \phi_P)\right)}{V_t^2} \right] $ 
\end{tabular}

\begin{tabular}{llp{4.5in}}
  \textbf{Option}&\textbf{Default}&\textbf{Description}\\
  \hline
        &  & 11: Same as \#1 but numerically normalized \\
        &  & 12: Spherical, Cartesian Gaussian numerically normalized \\
        &  & 2D: $ S = R G_{2D} $ \\
        &  & 3D: $ S = \exp \left[ - \frac{(R\cos\phi - R_p\cos\phi_p)^2 
                                          +(R\sin\phi - R_p\sin\phi_p)^2
                                          +(Z - Z_p)^2}{2 V_P^2} \right] $ \\
        &  & 13: Axisymmetric, toroidal Gaussian, numerically normalized 2D and 3D:
                 $S = G_{2D} $ \\
        &  & 14: Toroidal distribution is a blend of a von Mises and Cauchy distribution \\
        &  &     \begin{equation}
                 \begin{aligned}  
              &  S = G_{2D}\times \frac{R}{\sqrt{2 \pi} V_p} \times \nonumber \\
              &  \left[ (1 - f_c) \exp \left[  - \frac {R R_p \left( 1 - \cos(\phi-\phi_p) \right)}{V_t^2} \right] 
                       + f_c \times \frac{\cosh(\frac{V_t}{\sqrt{RR_p}}) - cos(\phi_p)}
                                             {\cosh(\frac{V_t}{\sqrt{RR_p}}) - cos(\phi - \phi_p)} \right]
                   \nonumber \end{aligned} 
                   \end{equation} \\


        &  & 15: Toroidal von-Mises distribution with angular half-width \\
        &  & $ S = G_{2D} \times \exp \left[ \cos(\phi - \phi_p) / V_p^2                   
                                      \right] $ \\
        &  & Pellet\_var\_tor radians for ipellet=15, distance otherwise \\

  \texttt{ipellet\_z}  & 0  & Atomic number of pellet (0 for main-ion species) \\
  \texttt{ipellet\_abl} & 0 & Turn on pellet ablation.  (Recommend double-digit ipellet for particle conservation) \\
                        &   & 1: Include ablation model [Parks NF94] calibrated on DIII-D (Li) \\
                        &   & 2: Include new ablation model [Parks, 2015] for small pellets (Li) \\
                        &   & 3: Parks model developed 6/20/2017 (for Neon) \\
  \texttt{temin\_abl}   & 0 & Minimum temperature at which ablation turns on \\
  \texttt{iread\_pellet}& 0 & 0: Single pellet defined by scalar parameters below \\
                        &   & 1: Read pellet.dat, with one row per pellet.   The 13
                                 space-delimited columns are: \\
                        &   & pellet\_r, pellet\_phi, pellet\_z, pellet\_rate, pellet\_var,
                              pellet\_var\_tor, pellet\_velr, pellet\_vel\_phi,
                              pellet\_vel\_z, r\_p, cloud\_pel, pellet\_mix, cauchy\_fraction \\    
  \texttt{pellet\_rate} & 0 & Particle number injection rate\\
  \texttt{pellet\_var}  & 1    & Variance of injection profile\\
  \texttt{pellet\_var\_tor} & 0 & toroidal spatial dispersion of pellet source $(V_t)$.  
                                  If zero, pellet\_var\_tor = pellet\_var. \\
  \texttt{pellet\_R}    & 0 & $R$-coordinate of injection profile\\
  \texttt{pellet\_z}    & 0 & $Z$-coordinate of injection profile\\
  \texttt{pellet\_velr} & 0 & initial radial velocity of the pellet \\
  \texttt{pellet\_velphi} & 0 & initial toroidal velocity of pellet \\
  \texttt{pellet\_velz}   & 0 & initial vertical velocity of pellet \\
  \texttt{r\_p}           & 1.e-3 & initial pellet radius \\
  \texttt{cloud\_pel}     & 1     & Parameter used to change the width of the density source if
                                    ablating.   In this case, pellet\_var = cloud\_pel * r\_p \\
  \texttt{pellet\_mix}   & 0 & Molar fraction of diatomic main ion molecures in pellet (e.g., $D_2$ )

\end{tabular}
\begin{tabular}{llp{4.5in}}
  \textbf{Option}&\textbf{Default}&\textbf{Description}\\
  \hline

  \texttt{irestart\_pellet} & 0 & 1: will read the following from C1input at restart (the rest from *.h5)
                                     pellet\_rate, pellet\_rate\_D2, pellet\_var\_tor, 
                                     pellet\_var, cloud\_pel, pellet\_mix, cauchy\_fraction \\
  \texttt{abl\_fac} & 1.0 &  factor to multiply ablation rate from predefined formula \\ 

  \hline
  \texttt{n\_control\_type} & -1 & -1:  no density control \\
                            &    &  0:  old density control algorithm (not recommended) \\
                            &    &  1:  Standard PID control with the following parameters: \\
  \texttt{n\_control\_p}    & 0  &  proportional feedback constant \\
  \texttt{n\_control\_i}    & 0 &  integral feedback constant \\
  \texttt{n\_control\_d}    & 0 &  derivative feedback constant \\
  \hline
  \texttt{igaussian\_heat\_source} & 0 & 1: include Gaussian heat fource \\
  \texttt{ghs\_x}   & 0 & R coordinate of Gaussian heat source \\
  \texttt{ghs\_z}   & 0 & Z coordinate of Gaussian heat source \\
  \texttt{ghs\_rate}& 0 & amplitude of Gaussian heat source \\
  \texttt{ghs\_var} & 0 & variance of Gaussian heat source \\
  \texttt{ghs\_phi} & 0 & phi coordinate of Gaussian heat source \\
  \texttt{ghs\_var\_tor} & 0 & toroidal variance of GAussian heat source \\
  \hline

  \texttt{ionization}   & 0  & 1: include neutral ionization source \\
  \texttt{ionization\_rate} & 0 & Ionization rate coefficient\\
  \texttt{ionization\_temp} & 0.01 & Ionization energy\\
  \texttt{ionization\_depth}& 0.01 & Temperature scale-length of neutral burn-out
\end{tabular}


\begin{tabular}{llp{4.5in}}
  \hline
current drive source & & $\dot{\psi} = ... + \eta(\Delta^* \psi - J_{CD}) $ \\
                     & & $J_{CD} = J_0 \exp- \left[ (R-R_{0cd})^2 + (Z-Z_{0cd})^2   
                                             \right]/W_{cd} - \Delta_{cd}  $ \\
 \texttt{icd\_source} & 0 & 1: include current drive \\
 \texttt{J\_0cd}      & 0 & $J_0$: Magnitude of Gaussian \\
 \texttt{R\_0cd}      & 0 & $R_{0cd}$: R coordinate of maximum \\
 \texttt{Z\_0cd}      & 0 & $Z_{0cd}$: Z coordinate of maximum \\
 \texttt{W\_cd}       & 0 & $W_{cd}$: Width of maximum \\
 \texttt{delta\_cd}   & 0 & $\Delta_{cd}$: shift of Gaussian \\
\hline
 \texttt{isink}          & 0 & number of density sinks: 0,1,or 2 \\
 \texttt{sink1\_x}       & 0 & R coordinate of first density sink \\
 \texttt{sink1\_z}       & 0 & Z coordinate of first density sink \\
 \texttt{sink1\_rate}    & 0 & rate of first density sink \\
 \texttt{sink1\_var}     & 1 & variance of first density sink \\
 \texttt{sink2\_x}       & 0 & R coordinate of second density sink \\
 \texttt{sink2\_z}       & 0 & Z coordinate of second density sink \\
 \texttt{sink2\_rate}    & 0 & rate of second density sink \\
 \texttt{sink2\_var}     & 1 & variance of second density sink \\
\hline
 \texttt{idenfloor}      & 0 & 1: density in vacuum pegged to den\_edge \\
 \texttt{alphadenfloor}  & 0 & multiplier of (den\_edge - den).  Must be .lt. 1/DT \\ 
%\end{tabular}
%\begin{tabular}{llp{4.5in}}
\hline
 \texttt{ipforce} & 0 & 1: include poloidal momentum source \\
                  &   & $F = f(\tilde{\psi}) \nabla \psi \times \nabla \phi$  \\
                  &   & $f(\tilde{\psi}) = \alpha(1 - \tilde{\psi})^N \frac{ \delta^2    }
                                                                           { (\tilde{\psi} - \psi_0)^2    + \delta^2} $  \\
                  &   & 2:  Include Luca Guzzatto form of momentum source  


\end{tabular}

\begin{tabular}{llp{4.5in}}
  \textbf{Option}&\textbf{Default}&\textbf{Description}\\
  \hline
  \texttt{dforce}  & 0 &  $\delta$ \\
  \texttt{xforce}  & 0 &  $\psi_0$ \\
  \texttt{nforce}  & 0 &  $N$ \\
  \texttt{aforce}   & 0 &  $ \alpha $ \\
  \texttt{iheat\_sink}         & 0 & 1:  special heat sink for itaylor=27  \\
  \texttt{coolrate}            & 0 & S = coolrate*(pedge-p) for iheat\_sink = 1 \\
  \texttt{iarc\_source}        & 0 & 1: density source due to halo current \\
  \texttt{arc\_source\_alpha}  & 0 &   parameter for arc\_source \\
  \texttt{arc\_source\_eta}    & .01 & parameter for arc\_source

\end{tabular}
\subsection{Resistive Wall}

\begin{tabular}{llp{4.5in}}
  \textbf{Option}&\textbf{Default}&\textbf{Description}\\
  \hline
  \texttt{eta\_vac}    & 1 &  resistivity of vacuum region \\
  \texttt{eta\_wall}   & .001 & resistivity of conducting wall regions \\
  \texttt{eta\_wallRZ} & .001 & poloidal resistivity of wall region (if different from eta\_wall) \\
  \texttt{iwall\_break}& 0 & number of wall breaks \\
  \texttt{eta\_break} & 1 & resistivity of wall break (array) \\
  \texttt{wall\_break\_phimax} & 0 & max phi coordinate for break (array) \\
  \texttt{wall\_break\_phimin} & 0 & min phi coordinate for break (array) \\
  \texttt{wall\_break\_xmax}   & 0 & max R coordinate for break (array) \\
  \texttt{wall\_break\_xmin}   & 0 & min R coordinate for break (array) \\
  \texttt{wall\_break\_zmax}   & 0 & max Z coordinate for break (array) \\
  \texttt{wall\_break\_zmin}   & 0 & min Z coordinate for break	(array)	\\ 
  \texttt{iwall\_regions}      & 0 & number of resistive wall regions \\
  \texttt{wall\_region\_eta()}  & 1.e-3 & resistivity of each wall region \\
  \texttt{wall\_region\_etaRZ()}& 1.e-3 & poloidal restivity (if different from wall\_region\_eta) \\
  \texttt{wall\_region\_filename()} & &   file name will wall contour points \\
  \texttt{eta\_zone(i)} & eta\_wall & Resistivity of mesh region \texttt{i}.  Only applies if \texttt{zone\_type(i)} = 2 (conductor).\\
  \texttt{etaRZ\_zone(i)} & eta\_zone(i) & Poloidal resistivity of mesh region \texttt{i}.  Only applies if \texttt{zone\_type(i)} = 2 (conductor).\\ 
  \texttt{eta\_rekc}  & 0 & resistivity of runaway electron killer coil (REKC) \\
  \texttt{ntor\_rekc} & 0 & toroidal mode number of REKC  \\
  \texttt{mpol\_rekc} & 0 & poloidal mode number of REKC \\
  \texttt{phi\_rekc}  & 0 & toroidal angle of fixed point of REKC \\
  \texttt{theta\_rekc}& 0 & poloidal angle of fixed point of REKC \\
  \texttt{rzero\_rekc}& 0 & R0 for computing theta of REKC \\
  \texttt{zzero\_rekc}& 0 & Z0 for computing theta of REKC \\
  \texttt{isym\_rekc} & 0 & if non-zero, coil is double helix with (+,-) mpol\_rekc
\end{tabular}

\subsection{Miscellaneous}

\begin{tabular}{llp{4.5in}}
  \textbf{Option}&\textbf{Default}&\textbf{Description}\\
  \hline
 \texttt(
 \texttt{gam}                    & 5/3 & ratio of sepcific heatx \\
 \texttt{db}                     & 0 & ion skin depth (overrides db\_fac \\
 \texttt{db\_fac}                & 0 & factor multiplying physical value of ion skin depth \\
 \texttt{mass\_ratio}            & 0 & ratio of ion to electron mass \\
 \texttt{lambdae}                & 0 & lambdae \\
 \texttt{z\_ion}                 & 1 & Z-effective \\
 \texttt{ion\_mass}              & 1 & ion mass in units of m\_p \\
 \texttt{lambda\_coulomb}         & 17 & Couloumb logarithm \\
 \texttt{thermal\_force\_coeff}  & 0 &  coefficient of thermal force \\
 \texttt{ntor}                   & 0 &  toroidal mode number for 3D linear (complex) \\
 \texttt{mpol}                   & 0 &  poloidal mode number for certain test problem initializations 

\end{tabular}

\subsection{Deprecated}

\begin{tabular}{llp{4.5in}}
  \textbf{Option}&\textbf{Default}&\textbf{Description}\\
  \hline
 \texttt{ipartitioned}   & 0 &     \\
 \texttt{igs\_method}    & -1&     \\
 \texttt{ibform}         & -1&     \\
 \texttt{delta\_wall}    & 1.&
\end{tabular}

\subsection{Trilinos Options}

\begin{tabular}{llp{4.5in}}
  \textbf{Option}&\textbf{Default}&\textbf{Description}\\
  \hline
 \texttt{drop\_tolerance}    & 0 &   ILU drop tolerance \\
 \texttt{graph\_fill}        & 0 & graph fill level \\
 \texttt{ilu\_fill\_level}   & 1 & ILU fill level \\
 \texttt{ilu\_omega}         & 1 & relaxation parameter for rILU \\
 \texttt{krylov\_solver}     & gmres & Krylov solver \\
 \texttt{poly\_ord}          & 1     & polynomial order for certain perconditioners \\
 \texttt{preconditioner}     &dom\_decomp & preconditioner \\
 \texttt{sub\_dom\_solver}   & ilu & subdomain solver in preconditioner \\
 \texttt{subdomain\_overlap} & 1   & subdomain overlap 

\end{tabular}

\subsection{Simple Radiation Model}

\begin{tabular}{llp{4.5in}}
  \textbf{Option}&\textbf{Default}&\textbf{Description}\\
  \hline
 \texttt{iprad}    & 0 & 1: call Prad radiation module with one impurity species \\
                   &   & $P_{rad} = n_e n_D L_D (T_e) + n_e n_Z L_Z(T_e)   $  \\
                   &   & Cooling rate of deuterium is $L_D = 5.35 \times 10^{-37}T_e^{1/2}[keV] W-m $ \\
                   &   & $L_Z(T_e)$ taken from Post, et al, Atomic data and nuclear data tables,
                                      {\bf 20} pp. 397-439 (1977)  \\
 \texttt{prad\_fz}  & 1 & density of impurity species as fraction of $n_e$ \\
 \texttt{prad\_z}      & 1 & Z of impurity species: Z=6(C), 18(Argon), 26(Fe) are available \\
 \texttt{iread\_prad}  & 0 & 1: Read impurity density from profile\_nz (units of $10^{20}/m^3$ ) 
\end{tabular}

\subsection{KPRAD Radiation Model}

\begin{tabular}{llp{4.5in}}
  \textbf{Option}&\textbf{Default}&\textbf{Description}\\
  \hline
 \texttt{ikprad}        & 0 & 1: KPRAD module with one impurity species \\
 \texttt{ikprad\_z}     & 1 & Z of impurity species in KPRAD module.   Presently available:
                                  2(He), 4(Be), 6 (C),10(Ne), 18 (Ar) \\
 \texttt{kprad\_fz}      & 0 & density of neutrals as fraction of ne \\
 \texttt{kprad\_nz}      & 0 & Density of neutral impurities \\
 \texttt{kprad\_nemin}   & $10^{-12}$& Minimum normalized electron density for KPRAD evolution \\
 \texttt{kprad\_temin}   & $ 2 \times 10^{-7} $ & Minimum normalized electron temperature for KPRAD evolution \\
 \texttt{ikprad\_max\_dt}&  0 & Set max time step for KPRAD ionization \\
                         &    & 0: MHD time step dt \\
                         &    & 1: {\bf RECOMMENDED:} dt/( kprad\_z + 1 ) : ensures evolution through all charge states. \\
 \texttt{ikprad\_evolve\_internal}  & 0 & Update local temperature during KPRAD subcycling: \\
                                    &   & 0: Te fixed before subcycling \\
                                    &   & 1: {\bf RECOMMENDED} Local ne and Te used for KPRAD 
                                                  ionization/radiation updated during subcycling each 
                                                  KPRAD time step due to density and energy changes \\
\texttt{ikprad\_evolve\_neutrals}   & 0 & Determines how KPRAD  neutrals evolve spatially: \\
                                    &   & 0: Neither advect nor diffuse \\
                                    &   & 1: {\bf RECOMMENDED} Advect and diffuse like other charge states \\
                                    &   & 2: Diffuse but do not advect \\
 

\texttt{ikprad\_min\_option}        & 1 & Determine how KPRAD behaves below minimum density/temperature
                                          (kprad\_nemin, kprad\_temin) \\ 
                                    &   & 1: No radiation/ionization/recombination (based on ne/te before subcycling) \\
                                    &   & 2: {\bf RECOMMENDED} Recombination but no radiation/ionization (based on ne/Te during subcycling) \\
                                    &   & 3: No radiation/ionization/recombination (based on ne/Te during subcycling).  
                                             NOTE: 1 and 3 behave the same if ikprad\_evolve\_internal = 0) \\

\texttt{iread\_lp\_source} & 0 & Read impurity source from Lagrangian Particle code cloud.txt (UNDER DEVELOPMENT)
\end{tabular}

\subsection{Stellarator Geometry}

\begin{tabular}{llp{4.5in}}
  \textbf{Option}&\textbf{Default}&\textbf{Description}\\
  \hline
 \texttt{type\_ext\_field}        & -1& External fields to be read in.. \\
				  &   &	0: Axisymmetric only. For RMP and error fields. \\
				  &   &	1: Free-boundary stellarator (\texttt{itaylor=41}) data. \\
				  &   &	2: Free-boundary stellarator (\texttt{itaylor=41}) data with \texttt{extsubtract=1}. \\
 \texttt{file\_ext\_field}        &   & (string) Vacuum/external field to be subtracted for \texttt{itaylor=41}.\\
				  &   & Currently supported: FIELDLINES, MGRID. \\
				  &   & Must start with `fieldlines' or `mgrid'. \\
                                  &   & Must have \texttt{type\_ext\_field=2}.\\
 \texttt{file\_total\_field}      &   & (string) Total field for free-boundary stellarator (\texttt{itaylor=41}).\\
				  &   & Currently supported: FIELDLINES, MGRID. \\
				  &   & Must start with `fieldlines' or `mgrid'. \\
                                  &   & Must have \texttt{type\_ext\_field=1,2}.\\
 \texttt{iread\_vmec}             & 0 & 1: read VMEC file to determine geometry. Must be 1 for stellarator. \\
 \texttt{vmec\_filename}          &   & (string) VMEC output .nc file \\
 \texttt{bloat\_factor}           & 0 & Free boundary only:  Scale factor to bloat computational boundary using input geometry. \\
 \texttt{bloat\_distance}         & 0 & Free boundary only:  Distance to expand computational boundary from input geometry. \\
 \texttt{igeometry}               & 0 & 1: must be set to use stellarator version \\
 \texttt{nperiods}                & 1 & Number of field periods (stellarator geometry; VMEC nfp must be divisible by nperiods).    Note nplanes
                                        should be equal to at least 2 x (number of toroidal modes 
                                        per field period) x nperiods \\
 \texttt{ifull\_torus}            & 0 & 0:  Solve on one field period \\
                                  &   & 1:  Solve on full torus \\
 \texttt{nzer\_factor}            & -1& (integer) Scale factor for resolution of Zernike polynomial
                                                  (used for interpolation of VMEC) \\
                                  &   &-1: n\_zeri= 2 x mpol (fixed boundary) and 1 x mpol (free boundary). \\
                                  &   &Otherwise: n\_zer = nzer\_factor x mpol where mpol is poloidal resolution in VMEC \\
 \texttt{nzer\_manual}            & -1&(int) Resolution of Zernike polynomial (mainly for testing)

\end{tabular}
