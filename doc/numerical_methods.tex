\section{Numerical Methods}

\subsection{Flux / Potential Representation}

The magnetic field is represented using a vector potential
\begin{equation}
  \vec{A} = R^2 \nabla \varphi \times \nabla f + \psi \nabla \varphi
\end{equation}
This vector potential satisfies the gauge condition $\nabla_p \cdot
\left( \vec{A} / R^2 \right) = 0$.  where
\[
\nabla_p \equiv \nabla - \nabla \varphi\, \partial_\varphi
\]
The magnetic field is

\begin{eqnarray}
  \vec{B} & = & \nabla \times \vec{A}\\
          & = & \nabla \psi \times \nabla \varphi + F \nabla \varphi - \nabla_p f'
\end{eqnarray}
where $f' \equiv \partial_\varphi f$ and
\begin{equation}
  F \equiv R^2 \nabla_p^2 f
\end{equation}
The current density is
\begin{eqnarray}
  \vec{J} & = & \nabla \times \vec{B}\\
          & = & \nabla \left(F + f'' \right) \times \nabla \varphi
 - \Delta^*\psi \nabla \varphi + R^{-2} \nabla_p \psi'
\end{eqnarray}
where
\[
\Delta^* \psi \equiv R^2 \nabla \cdot \left( \nabla_p \psi / R^2 \right)
\]

The velocity is defined as
\begin{equation}
  \vec{u} = R^2 \nabla U \times \nabla \varphi 
+ R^2 \omega \nabla \varphi 
+ R^{-2} \nabla_p \chi
\end{equation}


\subsection{Finite Elements}

\subsubsection{Integration quadrature}

The integrals required to calculate the weak-form equations of the
Galerkin method are computed numerically using an $n$-point Gaussian
quadrature.  That is, the value of each field is calculated at 79
points for each triangular element, and a weighted sum of these values
is computed to approximate the integral.
\[
  \int dA\ f(x) \simeq \sum_{i=1}^{n} w_i f(x_i),
\]
where the integrand $f(x)$ is restricted to a single element.  The
sampling points and weights appropriate for a equilateral triangle are
taken from ref.~\cite{Dunavant85}.

The coordinates of the sampling points are given in the ``natural
coordinates'' $(\alpha, \beta, \gamma)$ in ref.~\cite{Dunavant85}.
These coordinates may be converted to cartesian coordinates $(r, Z)$
for an equilateral triangle $e$ having vertices
\[
\left\{\left(-\frac{\sqrt{3}}{2},-\frac{1}{2}\right),
\left(\frac{\sqrt{3}}{2},-\frac{1}{2}\right), (0,1)\right\} 
\]
using the linear transformation
\[ 
\phi_{n \to e}(\alpha, \beta, \gamma) = 
\left(\frac{\sqrt{3}}{2}(\beta-\gamma), \frac{1}{2}(3 \alpha - 1)
\right).
\]
The weights must be multiplied by the Jacobian of this transformation,
\[
\mathcal{J}_{\phi_{n \to e}} = \frac{3 \sqrt{3}}{4}.
\]
To find the coordinates of the sampling points for a general triangle
$g$ having vertices $\{(-b,0), (a,0), (0,c)\}$, as in
ref.~\cite{Jardin04}, one may use the linear transformation
\begin{eqnarray*}
  \phi_{e \to g}(r,Z) & = & 
    \left(\frac{a+b}{\sqrt{3}} x + \frac{a-b}{3} (1-y), 
    \frac{c}{3}(2y+1) \right) \\
  \mathcal{J}_{\phi_{e \to g}} & = &  \frac{2 c}{3 \sqrt{3}} (a+b).
\end{eqnarray*}
The transformation from natural coordinates to cartesian coordinates
for a triangle having vertices $\{(-b,0), (a,0), (0,c)\}$ is therefore
\begin{eqnarray*}
  \phi_{n \to g}(r,Z) & = & 
  \left(\frac{1}{2} (a+b) (\beta-\gamma) +
  \frac{1}{2} (a-b)(1-\alpha), c \alpha \right) \\
  \mathcal{J}_{\phi_{n \to g}} & = & \frac{1}{2} (a+b) c.
\end{eqnarray*}

The 79-point quadrature gives the exact results for integrands which
are polynomials of degree 20 (or less).  In the case of quintic finite
elements, this means the integration is exact for terms involving
products of three fields or fewer, not including the degree-five
trial function $\nu$.  In cylindrical geometry, the presence factors
of $1/R$ will cause the quadrature not to be exact, as $1/R$ is not in
the form of a polynomial.  The weights $w_i$ must also be multiplied
by $R_i$ in cylindrical coordinates to account for the Jacobian of the
transformation from cartesian to cylindrical coordinates.


\subsection{Time Step}

For the implicit time advance, the fields are evaulated at the
$\theta$-advanced time (\emph{e.g.} $F(\psi) \to F(\psi + \theta \dt
\dot{\psi} + \cdots)$), linearized (\emph{i.e.} $\order{\dt^2}$ and
higher are dropped), and then discretized temporally according to the
chosen time integration method (\emph{i.e.} $\dot{\psi} \to
(\psi^{(n+1)} - \psi^{(n)})/\dt$).


\begin{eqnarray}
  \begin{pmatrix}
    \cola{S^v_{1 1}} & \cola{R^v_{1 1}} &
    \colb{S^v_{1 2}} & \colb{R^v_{1 2}} & 
          S^v_{1 3}  &        0         &
          R^v_{1 4}  &       R^v_{1 3}
    \\
    \cola{R^B_{1 1}} & \cola{S^B_{1 1}} &
    \colb{R^B_{1 2}} & \colb{S^B_{1 2}} & 
          R^B_{1 3}  &       S^B_{1 3}  &
              0      &        0
    \\
    \colb{S^v_{2 1}} & \colb{R^v_{2 1}} & 
    \colb{S^v_{2 2}} & \colb{R^v_{2 2}} & 
          S^v_{2 3}  &        0         &
	  R^v_{2 4}  &       R^v_{2 3}
    \\
    \colb{R^B_{2 1}} & \colb{S^B_{2 1}} &
    \colb{R^B_{2 2}} & \colb{S^B_{2 2}} & 
          R^B_{2 3}  &       S^B_{2 3}  &
              0      &        0
    \\
          S^v_{3 1}  &       R^v_{3 1}  &
          S^v_{3 2}  &       R^v_{3 2}  &
          S^v_{3 3}  &        0         &
	  R^v_{3 4}  &       R^v_{3 3}  
    \\
          R^B_{3 1}  &       S^v_{3 1}  &
          R^B_{3 2}  &       S^v_{3 2}  &
          R^B_{3 3}  &       S^v_{3 3}  &
              0      &        0
    \\
          R^n_{3 1}  &        0         &
          R^n_{3 2}  &        0         &
          R^n_{3 3}  &        0         &
          S^n        &        0
    \\
          R^p_{3 1}  &        0         &
          R^p_{3 2}  &        0         &
          R^p_{3 3}  &        0         &
              0      &       S^p        &
  \end{pmatrix}
  \begin{pmatrix}
    \cola{U}\\ \cola{\psi}\\ 
    \colb{V}\\ \colb{F}   \\
    \chi \\ p_e \\ 
    n \\ p
  \end{pmatrix}^{(n+1)} = \nonumber \\
  \begin{pmatrix}
    \cola{D^v_{1 1}} & \cola{Q^v_{1 1}} &
    \colb{D^v_{1 2}} & \colb{Q^v_{1 2}} & 
          D^v_{1 3}  &        0         &
          Q^v_{1 4}  &       Q^v_{1 3}
    \\
    \cola{Q^B_{1 1}} & \cola{D^B_{1 1}} &
    \colb{Q^B_{1 2}} & \colb{D^B_{1 2}} & 
          Q^B_{1 3}  &       D^B_{1 3}  &
              0      &        0
    \\
    \colb{D^v_{2 1}} & \colb{Q^v_{2 1}} & 
    \colb{D^v_{2 2}} & \colb{Q^v_{2 2}} & 
          D^v_{2 3}  &        0         &
	  Q^v_{2 4}  &       Q^v_{2 3}
    \\
    \colb{Q^B_{2 1}} & \colb{D^B_{2 1}} &
    \colb{Q^B_{2 2}} & \colb{D^B_{2 2}} & 
          Q^B_{2 3}  &       D^B_{2 3}  &
              0      &        0
    \\
          D^v_{3 1}  &       Q^v_{3 1}  &
          D^v_{3 2}  &       Q^v_{3 2}  &
          D^v_{3 3}  &        0         &
	  Q^v_{3 4}  &       Q^v_{3 3}  
    \\
          Q^B_{3 1}  &       D^v_{3 1}  &
          Q^B_{3 2}  &       D^v_{3 2}  &
          Q^B_{3 3}  &       D^v_{3 3}  &
              0      &        0
    \\
          Q^n_{3 1}  &        0         &
          Q^n_{3 2}  &        0         &
          Q^n_{3 3}  &        0         &
          D^n        &        0
    \\
          Q^p_{3 1}  &        0         &
          Q^p_{3 2}  &        0         &
          Q^p_{3 3}  &        0         &
              0      &       D^p        &
  \end{pmatrix}
  \begin{pmatrix}
    \cola{U}\\ \cola{\psi}\\ 
    \colb{V}\\ \colb{F}   \\
    \chi \\ p_e \\ 
    n \\ p
  \end{pmatrix}^{(n)} +   
  \begin{pmatrix}
    Q_1 \\ Q_2 \\ 
    Q_3 \\ Q_4 \\
    Q_5 \\ Q_6 \\ 
    Q_7 \\ Q_8
  \end{pmatrix}
\end{eqnarray}

\subsubsection{Split Time Step Method}



Time is advanced using a split time-step method in which the velocity
field is advanced first, then the density and total pressure fields
are advanced separately, and finally the magnetic field and electron
pressure are advanced together.  Though the velocity and magnetic
field are advanced separately, the Alfv\'en and magnetosonic waves are
treated implicitly by using the pressure equation and Faraday's law to
calculate analytically the advanced-time values of the pressure and
magnetic field for use in the velocity time step.



\begin{eqnarray}
  \label{eq:velocity_advance}
  \lefteqn{
  \begin{pmatrix}
    \cola{S^v_{1 1}} & \colb{S^v_{1 2}} & S^v_{1 3}\\
    \colb{S^v_{2 1}} & \colb{S^v_{2 2}} & S^v_{2 3}\\
          S^v_{3 1}  &       S^v_{3 2}  & S^v_{3 3}\\
  \end{pmatrix} 
  \begin{pmatrix}
    \cola{U}\\ \colb{V}\\ \chi
  \end{pmatrix}^{(n+1)}}\\
  & = & 
  \begin{pmatrix}
    \cola{D^v_{1 1}} & \colb{D^v_{1 2}} & D^v_{1 3}\\
    \colb{D^v_{2 1}} & \colb{D^v_{2 2}} & D^v_{2 3}\\
          D^v_{3 1}  &       D^v_{3 2}  & D^v_{3 3}\\
  \end{pmatrix} 
  \begin{pmatrix}
    \cola{U}\\ \colb{V}\\ \chi
  \end{pmatrix}^{(n)}
  + 
  \begin{pmatrix}
    \cola{Q^v_{1 1}} & \colb{Q^v_{1 2}} & Q^v_{1 3}\\
    \colb{Q^v_{2 1}} & \colb{Q^v_{2 2}} & Q^v_{2 3}\\
          Q^v_{3 1}  &       Q^v_{3 2}  & Q^v_{3 3}\\
  \end{pmatrix} 
  \begin{pmatrix}
    \cola{\psi}\\ \colb{F}\\ p
  \end{pmatrix}^{(n)} \nonumber
  \\ & & \mbox{} + 
  \begin{pmatrix}
    \cola{O^v_{1}}\\
    \colb{O^v_{2}}\\
          O^v_{3} \\
  \end{pmatrix} \nonumber
\end{eqnarray}

\begin{eqnarray}
  \label{eq:density_advance}
  S^n n^{(n+1)} & = & D^n n^{(n)} + 
  \begin{pmatrix} R^n_1 & R^n_2 & R^n_3\end{pmatrix}
  \begin{pmatrix}\cola{U}\\ \colb{V}\\ \chi\end{pmatrix}^{(n+1)}
  \\ & & \mbox{} + 
  \begin{pmatrix}Q^n_1    &   Q^n_2   & Q^n_3\end{pmatrix}
  \begin{pmatrix}\cola{U}\\ \colb{V}\\ \chi\end{pmatrix}^{(n)} \nonumber
\end{eqnarray}


\begin{eqnarray}
  \label{eq:pressure_advance}
  S^p p^{(n+1)} & = & D^p p^{(n)} + 
  \begin{pmatrix}R^p_1 & R^p_2 & R^p_3\end{pmatrix}
  \begin{pmatrix}\cola{U}\\ \colb{V}\\ \chi \end{pmatrix}^{(n+1)}
  \\ & & \mbox{} + 
  \begin{pmatrix}Q^p_1 & Q^p_2 & Q^p_3\end{pmatrix}
  \begin{pmatrix}\cola{U}\\ \colb{V}\\ \chi \end{pmatrix}^{(n)} \nonumber
\end{eqnarray}


\begin{eqnarray}
  \label{eq:field_advance}
  \lefteqn{
  \begin{pmatrix}
    \cola{S^B_{1 1}} & \colb{S^B_{1 2}} & S^B_{1 3}\\
    \colb{S^B_{2 1}} & \colb{S^B_{2 2}} & S^B_{2 3}\\
          S^B_{3 1}  &       S^B_{3 2}  & S^B_{3 3}\\
  \end{pmatrix} 
  \begin{pmatrix}
    \cola{\psi}\\ \colb{F}\\ p_e
  \end{pmatrix}^{(n+1)}}\\
  & = & 
  \begin{pmatrix}
    \cola{D^B_{1 1}} & \colb{D^B_{1 2}} & D^B_{1 3}\\
    \colb{D^B_{2 1}} & \colb{D^B_{2 2}} & D^B_{2 3}\\
          D^B_{3 1}  &       D^B_{3 2}  & D^B_{3 3}\\
  \end{pmatrix} 
  \begin{pmatrix}
    \cola{\psi}\\ \colb{F}\\ p_e
  \end{pmatrix}^{(n)} +
  \begin{pmatrix}
    \cola{R^B_{1 1}} & \colb{R^B_{1 2}} & R^B_{1 3}\\
    \colb{R^B_{2 1}} & \colb{R^B_{2 2}} & R^B_{2 3}\\
          R^B_{3 1}  &       R^B_{3 2}  & R^B_{3 3}\\
  \end{pmatrix} 
  \begin{pmatrix}
    \cola{U}\\ \colb{V}\\ \chi
  \end{pmatrix}^{(n+1)} \nonumber
  \\ & & \mbox{} +
  \begin{pmatrix}
    \cola{Q^B_{1 1}} & \colb{Q^B_{1 2}} & Q^B_{1 3}\\
    \colb{Q^B_{2 1}} & \colb{Q^B_{2 2}} & Q^B_{2 3}\\
          Q^B_{3 1}  &       Q^B_{3 2}  & Q^B_{3 3}\\
  \end{pmatrix} 
  \begin{pmatrix}
    \cola{U}\\ \colb{V}\\ \chi
  \end{pmatrix}^{(n)} +
  \begin{pmatrix}
    \cola{O^B_{1}}\\
    \colb{O^B_{2}}\\
          O^B_{3} \\
  \end{pmatrix} \nonumber
\end{eqnarray}
