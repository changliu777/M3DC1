\documentclass[10pt]{article}

\usepackage[pdftex]{color, graphicx}
\usepackage{hyperref} %for url
\usepackage{graphics}
\usepackage{epsfig}
\usepackage{latexsym}
\usepackage{mathrsfs}
\usepackage{eufrak}
\usepackage{amsmath}
\usepackage{amssymb}
\usepackage{graphicx}
\usepackage{longtable}
\usepackage{url}
\usepackage[T1]{fontenc}

\usepackage[% Set the height and width of the paper
includehead,
nomarginpar,% We don't want any margin paragraphs
textwidth=5.5in,
headheight=1in,
top=0.5in
]{geometry}

\usepackage{sectsty}
\sectionfont{\fontfamily{phv}\selectfont}
\subsectionfont{\fontfamily{phv}\selectfont}
\subsubsectionfont{\fontfamily{phv}\selectfont}

\usepackage[backend=biber]{biblatex}
\addbibresource{m3dc1.bib}



\newcommand{\dt}{\ensuremath{\delta t}}
\newcommand{\ddt}[1]{\frac{\partial #1}{\partial t}}
\newcommand{\thimp}{\ensuremath{\theta}}

\newcommand{\order}[1]{\ensuremath{\mathcal{O}(#1)}}

\renewcommand{\vec}[1]{\ensuremath{\mathbf{#1}}}
\newcommand{\tensor}[1]{\mathsf{#1}}
\newcommand{\tor}{\varphi}              % toroidal coordinate
\newcommand{\A}{\vec{A}}
\newcommand{\B}{\vec{B}}
\newcommand{\E}{\vec{E}}
\newcommand{\R}{\vec{R}}
\newcommand{\x}{\vec{x}}
\renewcommand{\r}{R}
\renewcommand{\v}{\vec{v}}
\renewcommand{\u}{\vec{u}}
\newcommand{\F}{\vec{F}}
\renewcommand{\j}{\vec{J}}
\newcommand{\q}{\vec{q}}
\newcommand{\g}{\vec{g}}
\newcommand{\jn}{\frac{\j}{n}}
\renewcommand{\P}{\tensor{\Pi}}
\renewcommand{\b}{\vec{b}}
\newcommand{\W}{\tensor{W}}
\newcommand{\codename}{\textsc{M3D-$C^1$}}

\newcommand{\grad}[1]{\nabla #1}
\newcommand{\gradp}[1]{\nabla_\perp #1}
\renewcommand{\div}[1]{\nabla \cdot #1}
\newcommand{\divp}[1]{\nabla_\perp \cdot #1}
\newcommand{\curl}[1]{\nabla \times #1}

\newcommand{\dotdot}{:}
\newcommand{\dottimes}{\dot\times}
\newcommand{\timestimes}{\stackrel{\times}{\times}}

\newcommand{\gs}[1]{\Delta^* #1}
\newcommand{\lp}[1]{\nabla^2 #1}
\newcommand{\pb}[2]{\left[#1,#2\right]}
\newcommand{\ip}[2]{\left\langle  #1,#2\right\rangle}
\newcommand{\funcss}[2]{
  \left\langle\left\langle #1,#2 \right\rangle\right\rangle}
\newcommand{\funcsa}[2]{\left[\left\langle #1,#2 \right\rangle\right]}
\newcommand{\funcaa}[2]{\left[\left[ #1,#2 \right]\right]}

\newcommand{\cola}[1]{\textcolor{red}{#1}}
\newcommand{\colb}[1]{\textcolor{blue}{#1}}

\newcommand{\uvec}[1]{\ensuremath{\vec{\hat{#1}}}}
\newcommand{\n}{\ensuremath{\uvec{n}}}

\newcommand{\repositoryloc}{portal.pppl.gov/p/tsc/C1/svn}
\newcommand{\svnurl}{http://subversion.tigris.org/}





%%%%%%%%%%%%%%%%%%%%%%%%%%%%%%%%%%%%%%%%%%%%%%%%%%%%%%%%%%%%%%%%%%
%\title{The M3D-C1 User's Guide}

%\author{
%M3D-C1 Team \\
%RPI SCOREC}

%\date{\today}

\begin{document}

%\maketitle

\begin{titlepage}
    \newgeometry{top=2in,bottom=1in}
    \begin{center}
        {\fontfamily{phv}\selectfont\Huge {\textbf{M3D-C1 User's Guide}}}\\
        \vspace{0.5in}
        {\fontfamily{phv}\selectfont\Large 
	{S. Jardin, N. Ferraro, J. Chen, and the M3D-C1 Team\\
	{M. Shephard, S. Seol, U. Riaz, and the RPI SCOREC Team}}}\\
	\vspace{0.5in}
	{\fontfamily{phv}\selectfont\Large \today}
	\vfill
    \end{center}
\end{titlepage}

\newpage
\restoregeometry
\tableofcontents
\newpage

\section{Obtaining M3D-C1}

%%%%%%%%%%%%%%%%%%%%%%%%%%%%%%%
\subsection{License Agreement}
\label{sec:license}
%%%%%%%%%%%%%%%%%%%%%%%%%%%%%%%

M3D-C1 is a code developed with funding from the US Department of
Energy, and is intended for open scientific research.  If you intend
to run M3D-C1, please complete the following steps:

\begin{enumerate}
\item Sign the license agreement
  \url{https://m3dc1.pppl.gov/M3D-C1\_License.pdf} and return to
  \href{mailto:nferraro@pppl.gov}{nferraro@pppl.gov}.
\item Request access to the code using the form
  \url{https://pppl.tiny.us/code-release-form}.  This step will
  involve review by PPPL to clear any potential export control issues.
\end{enumerate}

Please be aware that permission to run M3D-C1 does not carry an
implicit agreement to provide technical support for compiling,
running, modifying, or interpreting output of, M3D-C1.


%%%%%%%%%%%%%%%%%%%%%%%%%%%%%%%
\subsection{Preinstalled M3D-C1 Executables}
%%%%%%%%%%%%%%%%%%%%%%%%%%%%%%%

For the general user, we recommend using precompiled executables and
associated modules for release versions, where available.
Installations are presently available on a number of systems,
including:

\begin{description}
\item{GA Iris}:\\
  \texttt{module use /fusion/projects/codes/m3dc1/modules}\\
  \texttt{module load m3dc1/1.14}
\item{NERSC Perlmutter}:\\
  \texttt{module use /global/cfs/projectdirs/mp288/C1/modules/perlmutter}\\
  \texttt{module load m3dc1/1.14-cpu}
\item{PPPL Cluster}:\\
  \texttt{module use /p/m3dc1/modules}\\
  \texttt{module load m3dc1/1.14}
\item{PU Stellar}:\\
  \texttt{module use /projects/M3DC1/modules}\\
  \texttt{module load m3dc1/1.14}
\end{description}

These modules will modify the user's enviroment variables
appropriately to access the M3D-C1 executables, python libraries, and
IDL routines.



%%%%%%%%%%%%%%%%%%%%%%%%%%%%%%%
\subsection{Accessing the M3D-C1 Source Code}
%%%%%%%%%%%%%%%%%%%%%%%%%%%%%%%

If you choose to build the code yourself, either to use an unreleased
version or to do code development yourself, you will need access to
the M3D-C1 code repository.  The M3D-C1 source code is located in the
Github repository: \texttt{PrincetonUniversity/M3DC1}. To get access
to this repository, please complete the license form and software
access request form as described in section~\ref{sec:license}.


%%%%%%%%%%%%%%%%%%%%%%%%%%%%%%%
\subsection{Makefiles and Dependencies}
%%%%%%%%%%%%%%%%%%%%%%%%%%%%%%%

Some of the build scripts depend on the following environment
variables to be set, to specify the location of the M3D-C1 source code
and the directory in which to install any compiled executables:
\begin{description}
\item{\texttt{M3DC1\_DIR}} should be set to the directory containing
  the M3D-C1 source code.  For example:\\
  \texttt{setenv M3DC1\_DIR \$HOME/src/M3DC1}
\item{\texttt{M3DC1\_INSTALL\_DIR}} should be set to the directory in
  which M3D-C1 will be installed.  For example:\\
  \texttt{setenv M3DC1\_INSTALL\_DIR \$HOME/M3DC1}
\end{description}
It is recommended to set these values in your login script.

Makefiles for a number of systems are included in the repository,
with filenames \texttt{\$M3DC1\_DIR/unstructured/*.mk}.  For most of
these systems, environment modules are also included in the
repository.  These modules will load the appropriate software modules
for building on that particular system, and can be loaded using:

\begin{description}
\item{NERSC Perlmutter}:\\
  \texttt{module use \$M3DC1\_DIR/modules/perlmutter}\\
  \texttt{module load m3dc1/devel}
\item{PPPL Cluster}:\\
  \texttt{module use \$M3DC1\_DIR/modules/pppl}\\
  \texttt{module load m3dc1/devel-centos7}
\item{PU Stellar}:\\
  \texttt{module use \$M3DC1\_DIR/modules/stellar}\\
  \texttt{module load m3dc1/devel}
\end{description}

It is recommended to place the appropriate \texttt{module use}
statement in your login script.  The makefile should have the name
\texttt{\$\{M3DC1\_ARCH\}.mk}.  If \$M3DC1\_ARCH is not defined, it
will default to \texttt{\$HOST}, stripped of any trailing numbers
identifying a node index on multinode systems (\textit{e.g.} if
\texttt{\$HOST==``sunfire06''} then \texttt{\$M3DC1\_ARCH} will
default to ``sunfire'').

If you are building M3D-C1 on a system for which no makefile or development
module is provided, please refer to the existing makefiles and modules
as examples.  In general, M3D-C1 requires the following dependencies:
\begin{itemize}
\item Compilers for C, C++, and Fortran;
\item MPI
\item HDF5 compiled with support for Fortran and MPI
\item netcdf
\item GSL
\item FFTW
\item PETSc compiled with support for Fortran, complex-valued
  functions, MUMPS and/or SuperLU\_dist
\item PUMI meshing libraries
\end{itemize}

%%%%%%%%%%%%%%%%%%%%%%%%%%%%%%%
\subsection{Building}
%%%%%%%%%%%%%%%%%%%%%%%%%%%%%%%

Once the appropriate makefile has been defined, the M3D-C1 executables
can be built by entering \texttt{\$M3DC1\_DIR/unstructured} and
running

\texttt{make all}

This will run the following commands:

\begin{tabular}{ll}
\texttt{make OPT=1} & Builds the 2D version\\
\texttt{make OPT=1 COM=1} & Builds the complex version\\
\texttt{make OPT=1 3D=1 MAX\_PTS=60} & Builds the 3D version\\
\texttt{make OPT=1 3D=1 MAX\_PTS=60 ST=1} & Builds the stellarator version\\
\texttt{make a2cc} & Builds a utility for extracting coil currents
from a-eqdsk files\\
\texttt{make bin} & Puts executables into the \texttt{\_\$M3DC1\_ARCH} subdirectory\\
\end{tabular}


%%%%%%%%%%%%%%%%%%%%%%%%%%%%%%%
\subsection{Regression Tests}
%%%%%%%%%%%%%%%%%%%%%%%%%%%%%%%

The M3D-C1 source code includes a suite of regression tests that
should be run before commiting any new code to the repository.  To run
these tests:

\texttt{cd \$M3DC1\_DIR/unstructured/regtest}
\texttt{./run <arch> <test>}

where \texttt{<arch>} and \texttt{<test>} are optional arguments
specifying the specific batch script to run, and the specific
regression test to run, respectively.  By default,
\texttt{<arch>=\$M3DC1\_ARCH}.

If \texttt{<test>} is not
specified, then all the regression tests will be run, using the batch
scripts\\
\texttt{\$M3DC1\_DIR/unstructured/regtest/*/base/batchjob.<arch>}.

The \texttt{run} script will create new directories in which to run
these regression tests, named\\
\texttt{\$M3DC1\_DIR/unstructured/regtest/*/\$\{M3DC1\_VERSION\}\_<arch>/}.

If \texttt{<test>} is specified, then only
\texttt{\$M3DC1\_ARCH/unstructured/regtest/<test>/base/batchjob.<arch>}
will be run (again, in a new directory named as described above).  If
\texttt{<test>} is specified, then \texttt{<arch>} must also be
specified.

To check the results of the regression tests

\texttt{./check <arch>}




\section{Physics Model}
\label{sec:physics_model}

\section{Extended-MHD Equations}

The basic physics model for a single-species plasma in M3D-C1 is the following:
\begin{eqnarray}
  \frac{\partial n_i}{\partial t} + \nabla \cdot (n_i \vec{v})
  & = & \nabla \cdot (D \nabla n_i) + \sigma_i
\\
  \label{eq:force_balance}
  \rho \left( \frac{\partial \vec{v}}{\partial t}
  + \vec{v} \cdot \nabla \vec{v} \right)
  & = & 
  \vec{J} \times \vec{B} - \nabla p - \nabla \cdot \Pi
  - \varpi\, \vec{v}
  %+ \vec{F}_s
\end{eqnarray}
For a single-species plasma, $n_e = Z_i n_i$, $\rho = m_i n_i$, and
$\varpi = m_i \sigma_i$ (these are modified for;5D multi-species plasmas,
as described in section~\ref{sec:impurities}).  The final term in
equation~(\ref{eq:force_balance}) represents the slowing of the fluid
velocity to conserve momentum as new particles are added.

The magnetic field is advanced using Faraday's law
\begin{equation}
\label{eq:Faraday}
\frac{\partial \vec{B}}{\partial t} = -\nabla \times \vec{E}
\end{equation}
The electron momentum equation defines the generalized Ohm's law
\begin{equation}
\label{eq:ohm}
  \vec{E} = - \vec{v} \times \vec{B} + \eta\, \left(\vec{J} - \vec{J}_x \right)
  + \frac{d_i}{n_e}\left(\vec{J} \times \vec{B} - \nabla p_e \right)
\end{equation}
where $\vec{J}_x$ represents an external force on the electrons.

Several models for advancing the ion and electron temperatures, either
together or independently, are implemented in M3D-C1.  Each of these
models are built from the electron temperature equation:
\begin{multline}
  \label{eq:electron_temperature}
  n_e \left[ \frac{\partial T_e}{\partial t} + \vec{v} \cdot \nabla T_e
  + (\Gamma - 1) T_e \nabla \cdot \vec{v} \right] + \sigma_e T_e
  = \\
  (\Gamma - 1) \left[ \eta \vec{J} \cdot \left(\vec{J} - \vec{J}_x \right)
  - \nabla \cdot \vec{q}_e + Q_e
  -\Pi_e : \nabla \vec{v} \right]
\end{multline}
and the ion temperature equation is:
\begin{multline}
  \label{eq:ion_temperature}
  n_i \left[ \frac{\partial T_i}{\partial t} + \vec{v} \cdot \nabla T_i
  + (\Gamma - 1) T_i \nabla \cdot \vec{v} \right] + \sigma_* T_i
  = \\
  (\Gamma - 1) \left[ - \nabla \cdot \vec{q}_i + Q_i
  -\Pi_i : \nabla \vec{v}  + \frac{1}{2} \varpi\, v^2
  \right]
\end{multline}
The final term in equation~(\ref{eq:ion_temperature})
accounts for the net loss of kinetic energy caused by the slowing of
the fluid velocity as new particles are added.  Models implemented
include a single total pressure equation in which $p_e / p_i$ is
assumed fixed; a single total temperature equation in which $T_e /
T_i$ is assumed fixed; a two-pressure equation in which the total
pressure $p$ and electron pressure $p_e$ are evolved separately; and a
two-temperature model in which the electron temperature $T_e$ and ion
temperature $T_i$ are evolved independently using
equations~(\ref{eq:electron_temperature}) and
(\ref{eq:ion_temperature}).  Each of these models conserves total
energy, in the sense that all sources and sinks are are due to
physical processes that are included in the model.

The electron heat source density $Q_e$ is the sum of the heating from
radiation and ionization $Q_{rad}$ (this is negative for power lost
from the plasma) and the collisional transfer of energy from the ions
to the electrons $Q_\Delta$~\cite{Braginskii65}.  The radiated power
is calculated using the KPRAD module, and is the sum of the
bremsstrahlung, line radiation, and recombination losses.  We note
that KPRAD calculates both the kinetic energy and potential energy
released as radiation during recombination, but we only include the
kinetic contribution in $Q_e$, since the potential contribution does
not subtract from the kinetic energy of the electron fluid.

The equipartition term is the sum of the energy transfer from all ion
species to electrons:
\begin{equation}
  Q_\Delta = 3 m_e n_e (T_i - T_e)
  \left(\frac{\nu_{ei}}{m_i}
  + \sum_{j=1}^Z \frac{\nu_{eZ}^{(j)}}{m_Z} \right)
\end{equation}
where the collision rate between electrons and ion species $j$ is
\begin{equation}
  \nu_{ej} = \frac{4 \sqrt{2 \pi} e^4
    Z_j^2 n_j \ln \Lambda}{3 \sqrt{m_e} T_e^{3/2}}
\end{equation}
Therefore one can write
\begin{equation}
  Q_\Delta = 3 \mu \, \nu_{e H}\, n_e\, (T_i - T_e)
\end{equation}
where
\begin{equation}
  \mu = \frac{m_e}{n_e} \left(\frac{Z_i^2 n_i}{m_i} +
  \sum_{j=1}^Z \frac{j^2 n_Z^{(j)}}{m_Z} \right)
\end{equation}
and where $\nu_{e H}$ is the collision frequency for a pure Hydrogen
plasma ($Z_j=1$, $m_j = m_p$, $n_j = n_e$).  Similarly, summing the
collisional drag between the electrons and each ion species, the total
resistivity can be written in terms of the collision frequency for a
pure Hydrogen plasma scaled by an ``effective $Z$''~\cite{Wesson87}:
\begin{align}
  \label{eq:eta}
  \eta & = \frac{m_e \nu_{e H}}{n_e e^2} Z_{eff}
  \\
  Z_{eff} & = \frac{Z_i^2 n_i + \sum_{j=1}^Z j^2 n_Z^{(j)}}{n_e}
\end{align}
The heat flux densities are defined by
\begin{align}
  \vec{q}_e & =
  -\kappa^{e} \nabla T_e
  - \kappa^{e}_\parallel \frac{\vec{B} \vec{B}}{B^2} \cdot \nabla T_e
  \\
  \vec{q}_i & =
  -\kappa^{i} \nabla T_i
  - \kappa^{i}_\parallel \frac{\vec{B} \vec{B}}{B^2} \cdot \nabla T_i
\end{align}



\subsection{Impurity model}
\label{sec:impurities}

The impurity model evolves each individual
charge state of the impurity species, $n_Z^{(j)}$ for $1 \le j \le Z$,
by including the continuity equations
\begin{equation}
  \frac{\partial n_Z^{(j)}}{\partial t} + \nabla \cdot (n_Z^{(j)} \vec{v})
  = \nabla \cdot (D \nabla n_Z^{(j)}) + \sigma_Z^{(j)}
\end{equation}
The electron density is defined to satisfy quasi-neutrality
\begin{equation}
  n_e = Z_i n_i + \sum_{j=1}^Z j\, n_Z^{(j)}
\end{equation}
The particle source density for each impurity charge state due to
ionization and recombination, $\sigma_Z^{(j)}$, is calculated using
the KPRAD model~\cite{Whyte97}.

In addition, the mass density $\rho$ and mass density source rate
$\varpi$ are modified to account for the impurity density and sources:
\begin{eqnarray}
  \rho & = & m_i n_i + \sum_{j=1}^Z m_Z n_Z^{(j)}
  \\
  \varpi & = & m_i \sigma_i + \sum_{j=1}^Z m_Z \sigma_Z^{(j)}
\end{eqnarray}

It is assumed that all ionized impurities have the same temperature as
the main ion species, $T_i$.  The ion temperature equation is replaced
by the sum of the temperature equations for all ion species
(\textit{i.e.} main ions and all ionized impurities):
\begin{multline}
  \label{eq:ion_temperature_multispecies}
  n_* \left[ \frac{\partial T_i}{\partial t} + \vec{v} \cdot \nabla T_i
  + (\Gamma - 1) T_i \nabla \cdot \vec{v} \right] + \sigma_* T_i
  = \\
  (\Gamma - 1) \left[ - \nabla \cdot \vec{q}_* + Q_*
  -\Pi_* : \nabla \vec{v}  + \frac{1}{2} \varpi\, v^2
  \right]
\end{multline}
Here, $n_*$, $\sigma_*$,$\Pi_*$, $Q_*$, and $\vec{q}_*$ are sum over  
all ion species of the particle densities, particle source densities,
stresses, energy density sources, and energy density fluxes,                    
respectively.


\include{tutorials}

%\include{github}

%\include{equilibria}

\section{Mesh Generation and Management}

Given a mesh, all the mesh support needed to run M3D-C$^{1}$ is provided by PUMI (Parallel Unstructured Mesh Infrastructure) developed at RPI Scientific Computation Research Center (SCOREC). The PUMI is used to manage the mesh information as it is processed within the M3D-C$^{1}$ code. PUMI is a freely available open source software. For more information on how to get and install, visit \href{http://www.scorec.rpi.edu/pumi}{http://www.scorec.rpi.edu/pumi}.

The generation of M3D-C1 meshes for Tokamak geometries using the M3D-C1 mesh generation program involves both PUMI and Simmetrix mesh generation software. Simmetrix is commercial software which provides a set of tools and libraries for engineering simulation. As Simmetrix runs only with a valid license, any M3D-C1 user who wants to install his/her own version of the M3D-C1 mesh generation program should contact Simmetrix to purchase its license. For more information on Simmetrix, visit \href{http://simmetrix.com}{http://simmetrix.com}. {M3D-1}$^{1}$ users with accounts on \texttt{PPPL Portal} or \texttt{Princeton Stellar} are allowed to run M3D-C1 mesh mesh generation programs on those systems.
%In order to obtain an account on those two machines, contact M3D-C1 group (\href{mailto:jardin@pppl.gov}{jardin@pppl.gov}).

Note that Simmetrix software is not required to run a M3D-C1 simulation. Running a M3D-C1 simulation using a mesh generated by a different mesh generator requires the development of a tool that creates PUMI readable model and mesh files with M3D-C1 related control information required as the M3D-C1 input. For more information, contact RPI SCOREC (\href{mailto:shephard@rpi.edu}{shephard@rpi.edu}).
\newline

With respect to the mesh support, there are a number of files involved with housing the geometry and mesh information. First, the model file extensions are the following
\begin{itemize}
\item \texttt{.smd}: Simmetrix-readable binary format model file  
\newline  The model generated with Simmetrix is saved in this format.
\item \texttt{.dmg}: PUMI-readable binary format model file
\newline	The model generated from PUMI mesh
\item	\texttt{.txt}: M3D-C1-readable ascii format model file 
\newline	The model is generated from mesh generation tool (See Section~\ref{ch:mesh-gen})
\end{itemize}

Second, the mesh file extensions are the following.
\begin{itemize}
\item	\texttt{.sms}: Simmetrix-readable binary format mesh file
\begin{itemize}
\item	The mesh generated with Simmetrix is saved in this format.
\item	If a mesh is serial (1-part), the mesh file doesn't have a number before the extension
\item	If a mesh is distributed (\texttt{P}-part, P$>$1), the mesh file has a number before the extension to represent the global part ID.
\end{itemize}
\item	\texttt{.smb}: PUMI-readable binary format mesh file
\begin{itemize}
\item	This format is used in M3D-C1 to import/export a mesh
\item	No matter if a mesh is serial (1-part) or distributed (\texttt{P}-part, P$>$1), the mesh file has a number before the extension to represent the global part ID.
\end{itemize}
\item \texttt{.vtu/pvtu}: binary format mesh file for visualization with Paraview. For more information, visit \href{http://paraview.org}{http://paraview.org}.
\end{itemize}

An overview of the Model/Mesh requirements for the M3D-C1 mesh generation process are as follows:
\begin{itemize}
\item	The model and mesh shall be generated as described in Section~\ref{ch:mesh-gen}.
\item	The mesh file must be PUMI-readable \texttt{.smb} file. Note that a mesh file name contains a number before the extension (.smb) to denote a global part ID.
\item	The model and mesh file must be present in the work directory
\item	The name of model and mesh file must be specified in \texttt{C1input} file in the work directory
\begin{itemize}
\item	mesh\_model = model\_file
\item	mesh\_filename = mesh\_file.smb (NOTE: do not specify a number before the file extension)
\end{itemize}
\item In a 2D run with \texttt{P} processes, there should be \texttt{P} mesh files with part ID from \texttt{0} to \texttt{P-1}
\item	In a 3D run with \texttt{P$\times$N} processes where 2D mesh is distributed to \texttt{P} parts, 
\begin{itemize}
\item	there should be \texttt{P} mesh files with part ID from \texttt{0} to \texttt{P-1}
\item	in \texttt{C1input} file, specify \texttt{nplanes} to \texttt{N} (e.g. nplanes=8), where \texttt{nplanes} describes how many 2D mesh copies to be loaded
\item	the M3D-C1 code should be compiled with options \texttt{"3D=1, MAX\_PTS=60"}.
\end{itemize}
\end{itemize}

The rest of this section is organized as follows: Section \ref{ch:mesh-gen} describes a mesh generation program \texttt{m3dc1\_meshgen}. Section \ref{ch:mfm-gen} describes a mesh generation program \texttt{m3dc1\_mfmgen}. Section \ref{ch:polar-gen} describes a mesh generation program \texttt{polar\_meshgen}. Section \ref{ch:mesh-ptn} presents a mesh partitioning program \texttt{"split\_smb"} and \texttt{"collapse"} which changes the number of parts of the mesh. For how to visualize a mesh with \texttt{Paraview}, see Appendix~\ref{ch:app-paraview}.

%%%%%%%%%%%%%%%%%%%%%%%%%%%%%%%%%%%%%%%%%
\subsection{m3dc1\_meshgen}
\label{ch:mesh-gen}
%%%%%%%%%%%%%%%%%%%%%%%%%%%%%%%%%%%%%%%%%

\texttt{m3dc1\_meshgen} requires an ascii input file of arbitrary name that contains the following parameters.

\begin{itemize}
\item modelType: 0, 1, 2, 3, or 4
\begin{itemize}
\item Type 0: a parameterized vacuum region defined by five doubles for analytic expression. 
For five doubles $X_0$   $X_1$   $X_2$   $Z_0$   $Z_1$, vacuum boundary is defined by 
\begin{equation}
X = X_0 + X_1 cos(\theta + X_2*sin(\theta))
\end{equation}
\begin{equation}
Z =  Z_0 + Z_1 sin(\theta)
\end{equation}
\item	Type 1: a vacuum region defined by piece-wise linear points
\item	Type 2: a vacuum region defined by piece-wise polynomials
\item	Type 3: spline-fitted 3-region model (plasma, wall and vacuum)
\item	Type 4: spline-fitted 3-region model (plasma, wall, and vacuum) with inner \& outer boundary points to set resistive wall
\end{itemize}

\item reorder: if 1, reorder PUMI mesh based on adjacency (default: 0) and generate vtk folders for mesh visualization. The mesh before and after reodering is saved in \texttt{original-mesh.vtk} and \texttt{reordered-mesh.vtk}, respectively. Note that the element order of Simmetrix mesh is not affected.
\item inFile: (modelType 0) not required
(modelType 1 and 2) geometry file describing the vacuum (modelType 3 and 4) geometry file describing the inner plasma wall
\item bdryFile: (modelType 0-3) not required
 (modelType 4) geometry file describing the outer plasma wall
\item outFile: output file name to save model and mesh
\item meshSize: relative mesh size for each region (default 0.05)
\newline
	for modelType 3, set three doubles for plasma, resistive, vacuum, respectively
\item useVacuumParams: for modelType 0 or 3, if 1, use parameterized vacuum wall (default 0)
\item vacuumParams: five doubles to describe parameterized vacuum wall. Required if useVacuumParams=1.
\item adjustVacuumParams: for modelType 0 or 3, if 1, multiply coordinates and parametric values of nodes on vacuum wall by vacuumFactor. Valid only if useVacuumParams=1 (default 0)
\item vacuumFactor: for modelType 0 or 3, an optional double value used to multiply coordinates and parametric values of nodes on vacuum wall when adjustVacuumParams=1. Valid only if adjustVacuumParams=1 (default 2$\times$PI)
\item numVacuumPts: optional \# interpolation points on parameterized vacuum wall. Valid only if useVacuumParams=1 (default 20)
\item meshGradationRate: for modelType 3 or 4, optional mesh gradation rate (default: 0.3). This value should be greater than or equal to 0.3. Otherwise the mesh will be fine everywhere.
\item resistive-width: for modelType 3, the width of resistive wall. If resistive-width=0, only plasma region is created (default 0.02)
\item plasma-offsetX: for modelType 3, the offset in x direction to the left (default 0.0)
\item plasma-offsetY: for modelType 3, the offset in y direction to the bottom (default 0.0)
\item vacuum-width: for modelType 3 or 4, the width of vacuum region (default 2.5)
\item vacuum-height: for modelType 3 or 4, the height of vacuum region (default 4.0) 
\end{itemize}

Locate input parameter file and all files listed as bdryFile (if applicable) in the work folder and do \texttt{m3dc1\_meshgen input\_param\_file}, then the following output will be generated.

\begin{itemize}
\item The output model in three formats
  \begin{itemize}
  \item M3D-C1 readable \texttt{.txt}
  \item Simmetrix-readable file \texttt{.smd} and
  \item PUMI-readable \texttt{.dmg}
    \begin{itemize}
    \item[$\triangleright$] For modelType 0-2, the model is saved in \texttt{outFile.*}
    \item[$\triangleright$] For modelType 3 with resistive width \texttt{R}, vacuum-width \texttt{W} and vacuum-height \texttt{H}, the model is saved in \texttt{outFile-R-W-H.*}.
    \item[$\triangleright$] For modelType 4 with vacuum-width \texttt{W} and vacuum-height \texttt{H}, the model is saved in \texttt{outFile-W-H.*}.
    \end{itemize}
  \end{itemize}
\item The output mesh in three formats
  \begin{itemize}
  \item Simmetrix-readable\texttt{.sms}
  \item M3D-C1/PUMI readable \texttt{.smb}
  \item Paraview
    \begin{itemize}
    \item[$\triangleright$] For modelType 0-2 with \# mesh faces \texttt{F},
      \begin{itemize}
      \item[-] if \texttt{F} $>$ 1000, the mesh is saved in \texttt{outFile-(F/1000).*}
      \item[-] if \texttt{F} $<$ 1000, the mesh is saved in \texttt{outFile-F.*}
      \end{itemize}
    \item[$\triangleright$] For modelType 3 with \# mesh faces \texttt{F}, resistive width \texttt{R}, vacuum-width \texttt{W}, vacuum-height \texttt{H},
      \begin{itemize}
      \item[-] if \texttt{F} $>$ 1000, the mesh is saved in \texttt{outFile-R-W-H-(F/1000).*}
      \item[-] if \texttt{F} $<$ 1000, the mesh is saved in \texttt{outFile-R-W-H-F.*}
      \end{itemize}
    \item[$\triangleright$] For modelType 4 with \# mesh faces \texttt{F}, vacuum-width \texttt{W} and vacuum-height \texttt{H},
      \begin{itemize}
      \item[-] if \texttt{F} $>$ 1000, the mesh is saved in \texttt{outFile-W-H-(F/1000).*}
      \item[-] if \texttt{F} $<$ 1000, the mesh is saved in \texttt{outFile-W-H-F.*}
      \end{itemize}
    \end{itemize}
  \end{itemize}
\end{itemize}

%%%%%%%%%%%%%%%%%%%%%%%%%%%%%%%%%%%%%%%%%
\subsubsection{Type 0 (parameterized vacuum)}
%%%%%%%%%%%%%%%%%%%%%%%%%%%%%%%%%%%%%%%%%

\begin{figure}
\centering
\includegraphics[width=3in]{./figures/meshgen-type0.pdf}
\caption[Mesh with vacuum region defined by five parameters for analytic expression]
{A mesh with vacuum region defined by five parameters for analytic expression}
\label{fig:meshgen-type0}
\end{figure}

The figure~\ref{fig:meshgen-type0} illustrates a mesh generated by the following input file.

\begin{verbatim}
modelType 0
outFile analytic 
meshSize 0.04

useVacuumParams 1
vacuumParams 1.65908 0.46 0.2 -0.02504 0.8
numVacuumPts 20

adjustVacuumParams 0
vacuumFactor  6.28319
\end{verbatim}

\begin{figure}
\centering
\includegraphics[width=3in]{./figures/meshgen-type0-ms.pdf}
\caption[Mesh generated with vacuum region parameters and mesh size 0.1]
{A mesh generated with vacuum region parameters and mesh size 0.1}
\label{fig:meshgen-type0-ms}
\end{figure}

The figure~\ref{fig:meshgen-type0-ms} illustrates a mesh generated with the same vacuum region parameters and a higher mesh size value.

%%%%%%%%%%%%%%%%%%%%%%%%%%%%%%%%%%%%%%%%%
\subsubsection{Type 2 (piece-wise polynomial vacuum)}
%%%%%%%%%%%%%%%%%%%%%%%%%%%%%%%%%%%%%%%%%

\begin{figure}
\centering
\includegraphics[width=3in]{./figures/meshgen-type2.pdf}
\caption[Mesh with vacuum region defined by piece-wise polynomials]
{A mesh with vacuum region defined by piece-wise polynomials}
\label{fig:meshgen-type2}
\end{figure}

The figure~\ref{fig:meshgen-type2} illustrates a mesh generated by the following input file. The vacuum region's geometry information is defined by piece-wise polynomials and stored in the file \texttt{in-poly}. 

\begin{verbatim}
modelType 2
inFile in-poly
outFile poly
\end{verbatim}

The vacuum region's geometry information is defined by piece-wise polynomials and stored in the file \texttt{in-poly} and the example file can be found in 
\newline\newline
\texttt{/p/tsc/m3dc1/lib/SCORECLib/rhel7/intel2019u3-openmpi4.0.3/16.0-220226/bin}.

%%%%%%%%%%%%%%%%%%%%%%%%%%%%%%%%%%%%%%%%%
\subsubsection{Type 3 (three-regions with inner wall points)}
%%%%%%%%%%%%%%%%%%%%%%%%%%%%%%%%%%%%%%%%%
With the model type 3, a geometric model consists of three model faces where each represents plasma region, resistive region and vacuum region, respectively. An ascii file name which describes inner plasma wall boundary has to be provied with the parameter \texttt{inFile}.

\begin{figure}
\centering
\includegraphics[width=3in]{./figures/meshgen-type3.pdf}
\caption[Mesh with spline-fitted 3-region model]
{A mesh with spline-fitted 3-region model}
\label{fig:meshgen-type3}
\end{figure}

The figure~\ref{fig:meshgen-type3} illustrates a mesh generated by the following input file. In the figure, geometric model faces are different colored.

\begin{verbatim}
modelType 3
inFile in-circle
outFile circle
meshSize 0.1 0.5 0.1
useVacuumParams 1
adjustVacuumParams 1
vacuumParams 5.0 1.5 0.0 0.0 1.5
numVacuumPts 20
meshGradationRate 0.4
resistive-width 0.4
\end{verbatim}

The example files \texttt{circle-input} and \texttt{in-circle} can be found in
\newline\newline
\texttt{/p/tsc/m3dc1/lib/SCORECLib/rhel7/intel2019u3-openmpi4.0.3/16.0-220226/bin}.

%%%%%%%%%%%%%%%%%%%%%%%%%%%%%%%%%%%%%%%%%
\subsubsection{Type 4 (three-regions with inner $\&$ outer wall points)}
%%%%%%%%%%%%%%%%%%%%%%%%%%%%%%%%%%%%%%%%%

With the model type 4, a geometric model consists of three model faces where each represents plasma region, resistive region and vacuum region, respectively.  The parameter \texttt{inFile} denotes a file name that contains inner plasma wall boundary. The parameter \texttt{bdryFile} denotes a file name that contains resistive wall boundary.

\begin{figure}
\centering
\includegraphics[width=3in]{./figures/meshgen-type4.pdf}
\caption[Mesh with inner $\&$ outer plasma wall boundary points]
{A mesh with inner $\&$ outer plama wall boundary points}
\label{fig:meshgen-type4}
\end{figure}

The figure~\ref{fig:meshgen-type4} illustrates a mesh generated by the following input file. In the figure, geometric model faces are different colored.

\begin{verbatim}
modelType 4
inFile inner_bdry.pts
bdryFile outer_bdry.pts
outFile iter
meshSize 0.7 0.5 0.7
useVacuumParams 1
adjustVacuumParams 1
vacuumParams 8.25 8.0 0.2 0.0 12.5
meshGradationRate 1
\end{verbatim}

The example files \texttt{bdry-input}, \texttt{inner\_bdry.pts} and \texttt{outer\_bdry.pts} can be found in
\newline\newline
\texttt{/p/tsc/m3dc1/lib/SCORECLib/rhel7/intel2019u3-openmpi4.0.3/16.0-220226/bin}.

%%%%%%%%%%%%%%%%%%%%%%%%%%%%%%%%%%%%%%%%%
\subsection{m3dc1\_mfmgen}
\label{ch:mfm-gen}
%%%%%%%%%%%%%%%%%%%%%%%%%%%%%%%%%%%%%%%%%

\texttt{m3dc1\_mfmgen} requires an ascii input file of arbitrary name that contains the following parameters.
The parameters can be in any order.

\begin{itemize}
\item numBdry: the number of boundary files defined by peice-wise linear points given for the construction of the loops (default: 0)
  \begin{itemize}
  \item Each boundary file corresponds to a loop in PUMI
  \item For \texttt{numBdry}=$N$, $N$ lines of \texttt{bdryFile} should be provided, $N \ge 0$
  \end{itemize}
\item bdryFile: For each boundary file, the user has to provide its file name followed by the unique loop ID and desired mesh size on the loop.
  \begin{itemize}
  \item Each boundary file corresponds to a loop in PUMI
  \item The unique ID can be an arbitrary integer defined by the user
  \item The unique ID is used with input parameter ``faceBdry" to specify the boundaries (loops) of model face
  \item For more than one boundary files (numBdry$>$1), the boundary files can be in any order
  \end{itemize}

\item useVacuum: A parameter to control the vacuum boundary
  \begin{itemize}
  \item The first number sets the mode of vacuum boundary and can be 0, 1 or 2. If 0, no vacuum boundary will be created. If 1, vacuum boundary will be created without user defined parameters. If 2, a parameterized vacuum boundary will be created by using the parameters defined in the parameter "vacuumParams".
  \item The second number is the desired unique loop ID for the vacuum loop
  \item  The third number defines the mesh size on the vacuum boundary
  \end{itemize}
\item vacuumParams: if ``useVacuum = 2", the user has to provide five doubles to define parameterized vacuum wall
\item numVacuumPts: if ``useVacuum = 2", \# interpolation points on parameterized vacuum wall (default=20)

\item thickWall: three integers and one double to control finite thickness wall
  \begin{itemize}
  \item The first number can either be 0 or 1. If it is set to 0, no finite thickness wall will be created. If it is set to 1, a finite thickness wall will be created
  \item The second number is the loop ID that will be offset for given thickness
  \item The third number is the desired unique loop ID for the new loop created from offsetting for the finite thickness wall
  \item The last number is the desired wall thickness
  \end{itemize} 

\item layeredMesh: two integers to create an extruded layeded mesh on the finite thickness wall
  \begin{itemize}
  \item The first integer is 0, no layered mesh will be created. If 1 (default), an extruded mesh with desired number of mesh layers will be created
  \item The second integer defines the number of mesh layers
  \end{itemize}

\item numFace: the number of geometric model faces in PUMI (default 1)
    \begin{itemize}
  \item Each geometric face corresponds to regions (e.g. plasma, resistive, vacuum) in M3D-C1
  \item For \texttt{numFace}=$N$, $N$ lines of \texttt{faceBdry} should be provided, $N > 0$
  \end{itemize}
\item faceBdry: For each model face, the user has to provide the number of loops, loop ID(s), and desired mesh size
  \begin{itemize}
  \item The first number gives the total number of loops bounding the face
  \item the first number is followed by the loops IDs of the bounding loops. If number of loops = $n$, there should be $n$ loop ID
  \item The last number is the desired mesh size on the geometric face
    \end{itemize}

\item meshGradationRate: Global mesh gradation rate for the meshing. This parameter is optional and if not specified a default mesh gradation rate = 0.3 is used. This value should be greater than or equal to 0.3. Otherwise the mesh will be fine everywhere.
\item outFile: output file name to save model and mesh
\end{itemize}

Locate input parameter file and all files listed as bdryFile (if applicable) in the work folder and do \texttt{m3dc1\_mfmgen input\_param\_file}. The output files are the same as those of \texttt{m3dc1\_meshgen}.

%%%%%%%%%%%%%%%%%%%%%%%%%%%%%%%%%%%%%%%%%
\subsubsection{Mesh with parameterized vacuum wall}
%%%%%%%%%%%%%%%%%%%%%%%%%%%%%%%%%%%%%%%%%

This section presents a mesh created with a parameterized vacuum wall. This is equivalent to \texttt{Type 0} mesh of \texttt{m3dc1\_meshgen}.

\begin{verbatim}
numBdry 0
useVacuum 1 1 0.1
numFace 1
faceBdry 1 1 0.09
outFile analytic-0.09
\end{verbatim}

\begin{figure}
\centering
\includegraphics[width=3in]{./figures/meshgen-analytic-20pts-05.pdf}
\includegraphics[width=3in]{./figures/meshgen-analytic-20pts-09.pdf}
\caption{Mesh with a parameterized vacuum region and different mesh size for model face (left) 0.2 (right) 0.09}
\label{fig:analytic-mesh}
\end{figure}

The figure~\ref{fig:analytic-mesh} presents the mesh generated by the input file above with two different mesh sizes.

%\begin{verbatim}
%numBdry 0
%numFace 1
%faceBdry 1 1
%outFile analytic-0.05
%meshSize 0.05
%useVacuumParams 1
%vacuumParams 1.65908 0.46 0.2 -0.02504 0.8
%numVacuumPts 50
%\end{verbatim}

%\begin{figure}
%\centering
%\includegraphics[width=3in]{./figures/meshgen-analytic-50pts-05.pdf}
%\caption[Mesh with parameterized vacuum region II]
%{A mesh with parameterized vacuum region of 50 interpolation points and mesh size 0.05}
%\label{fig:analytic-mesh-2}
%\end{figure}
%
%The figure~\ref{fig:analytic-mesh-2} presents the mesh generated by the input file above.

%%%%%%%%%%%%%%%%%%%%%%%%%%%%%%%%%%%%%%%%%
\subsubsection{Mesh with single boundary file}
%%%%%%%%%%%%%%%%%%%%%%%%%%%%%%%%%%%%%%%%%

\begin{verbatim}
numBdry 1
bdryFile loop1.dat 3 0.1
numFace 1
faceBdry 1 3 0.2
outFile input1
\end{verbatim}

\begin{figure}
\centering
\includegraphics[width=3in]{./figures/meshgen-input1-novacuum.pdf}
\caption{Mesh with single boundary file and no vacuum wall}
\label{fig:input1}
\end{figure}

The figure~\ref{fig:input1} presents the mesh generated by the input file above.

%%%%%%%%%%%%%%%%%%%%%%%%%%%%%%%%%%%%%%%%%
%\subsubsection{Mesh with single boundary file and a vacuum wall} 
%%%%%%%%%%%%%%%%%%%%%%%%%%%%%%%%%%%%%%%%%
%\begin{verbatim}
%numBdry 1
%bdryFile loop1.dat 3 0.1
%
%thickWall 0 19 8 0.07
%useVacuum 2 9 0.01
%vacuumParams 1.8 1.5 0.4 0.0 2.5
%numVacuumPts 20
%
%numFace 1
%faceBdry  1 3 0.2
%outFile input1
%
%meshGradationRate 0.3
%\end{verbatim}
%
%\begin{figure}
%\centering
%\includegraphics[width=3in]{./figures/meshgen-input1.pdf}
%\caption[Mesh with one boundary file and vacuum wall]
%{Mesh with one boundary file and vacuum wall}
%\label{fig:input1-vacuum}
%\end{figure}

%The figure~\ref{fig:input1-vacuum} presents the mesh generated by the input file above.

%%%%%%%%%%%%%%%%%%%%%%%%%%%%%%%%%%%%%%%%%
\subsubsection{Mesh with two boundary files and a parameterized vacuum wall}
%%%%%%%%%%%%%%%%%%%%%%%%%%%%%%%%%%%%%%%%%

This section presents a mesh created with two boundary files and a parameterized vacuum wall. This is equivalent to \texttt{Type 4} mesh of \texttt{m3dc1\_meshgen}.

\begin{verbatim}
numBdry 2
bddyFile loop1.pts 1 0.5
bdryFile loop2.pts 2 0.5
useVacuum 2 3 0.5
vacuumParams 8.25 8.0 0.2 0.0 12.5
numFace 3
faceBdry 1 1 0.7
faceBdry 2 1 2 0.5
faceBdry 2 2 3 0.7
meshGradationRate 1
outFile iter
\end{verbatim}

\begin{figure}
\centering
\includegraphics[width=3in]{./figures/meshgen-input2.pdf}
\caption{Mesh with two boundary files and a parameterized vacuum region}
\label{fig:input2}
\end{figure}

The figure~\ref{fig:input2} presents the mesh generated by the input file above. As you can see, the mesh in Figure ~\ref{fig:input2} and ~\ref{fig:meshgen-type4} are almost identical.


%%%%%%%%%%%%%%%%%%%%%%%%%%%%%%%%%%%%%%%%%
\subsubsection{Mesh with three boundary files}
%%%%%%%%%%%%%%%%%%%%%%%%%%%%%%%%%%%%%%%%%

\begin{figure}
\centering
\includegraphics[width=3in]{./figures/meshgen-input3-novacuum.pdf}
\includegraphics[width=3in]{./figures/meshgen-input3-novacuum2.pdf}
\caption
{Mesh with three boundary files and different meshGradationRate (left) 0.3 (right) 0.9}
\label{fig:input3-novacuum}
\end{figure}

\begin{verbatim}
numBdry 3
bdryFile loop1.dat 3 0.1
bdryFile loop2.dat 10 0.05
bdryFile loop3.dat 11 0.09

numFace 3
faceBdry  1 3 0.2
faceBdry  2 3 10 0.1
faceBdry  2 10 11 0.09

outFile input3
\end{verbatim}

The figure~\ref{fig:input3-novacuum} presents the mesh generated by the input file above.

%%%%%%%%%%%%%%%%%%%%%%%%%%%%%%%%%%%%%%%%%
\subsubsection{Mesh with seven boundary files and a vacuum wall}
%%%%%%%%%%%%%%%%%%%%%%%%%%%%%%%%%%%%%%%%%

\begin{figure}
\centering
\includegraphics[width=4in]{./figures/meshgen-input7.pdf}
\caption{Mesh with seven boundary files and a parameterized vacuum wall}
\label{fig:input7-vacuum}
\end{figure}

\begin{verbatim}
numBdry 7
bdryFile loop1.dat 3 0.2
bdryFile loop2.dat 10 0.3
bdryFile loop3.dat 11 0.4
bdryFile loop4.dat 21 0.4
bdryFile loop5.dat 25 0.4
bdryFile loop6.dat 17 0.2
bdryFile loop7.dat 19 0.1

useVacuum 1 9 0.1
vacuumParams 1.8 1.5 0.4 0.0 2.5

numFace 8
faceBdry  1 3 0.2
faceBdry  1 10 0.3
faceBdry  1 11 0.1
faceBdry  2 21 25 0.2
faceBdry  2 25 17 0.09
faceBdry  2 17 19 0.1
faceBdry  2 19 9 0.1
faceBdry  4 3 10 11 21 0.11

outFile input7
\end{verbatim}

The figure~\ref{fig:input7-vacuum} presents the mesh generated by the input file above.

%%%%%%%%%%%%%%%%%%%%%%%%%%%%%%%%%%%%%%%%%
\subsubsection{Mesh with finite thickness wall and layers}
%%%%%%%%%%%%%%%%%%%%%%%%%%%%%%%%%%%%%%%%%

\begin{figure}
\centering
\includegraphics[width=3.5in]{./figures/FiniteThicknessWall-full.pdf}
\caption
{Add Caption Here}
\label{fig:thickness-full}
\end{figure}

\begin{figure}
\centering
\includegraphics[width=2.5in]{./figures/FiniteThicknessWall-zoom1.pdf}
\includegraphics[width=2.5in]{./figures/FiniteThicknessWall-zoom2.pdf}
\caption
{Add Caption Here}
\label{fig:thickness-zoom}
\end{figure}

Add text here

%%%%%%%%%%%%%%%%%%%%%%%%%%%%%%%%%%%%%%%%%
\subsection{polar\_meshgen}
\label{ch:polar-gen}
%%%%%%%%%%%%%%%%%%%%%%%%%%%%%%%%%%%%%%%%%
\texttt{polar\_meshgen} requires an ascii file of arbitrary name that contains input parameters as the following:
\begin{itemize}
\item inFile: input file name containing equilibrium generation by jsolver
\item outFile: output file name to save model and mesh
\item meshSize: relative mesh size for each region (default 0.05)
\item reorder: if 1, reorder PUMI mesh based on adjacency (default: 0) and generate vtk folders for mesh visualization. The mesh before and after reodering is saved in \texttt{original-mesh.vtk} and \texttt{reordered-mesh.vtk}, respectively. Note that the element order of Simmetrix mesh is not affected.
\end{itemize}

The following presents an example input file ``\texttt{polar\_input}''.
\begin{verbatim}
inFile POLAR
outFile polar
meshSize 0.04
\end{verbatim}

To run \texttt{polar\_meshgen}, place \texttt{polar\_input} and \texttt{POLAR} in your work folder and do ``\texttt{polar\_meshgen polar\_input}''. The program will read \texttt{POLAR} and generate various model and mesh files starting with ``polar''. For instance, \texttt{polar-2K0.smb, pol-2K.sms, pol-2K.vtk, polar.dmg, polar.smd, polar.txt}. If the resulting mesh is too fine, increase the value of \texttt{meshSize}. If the resulting mesh is too coarse, decrease the value of \texttt{meshSize}. If \texttt{meshSize} is not specified in the input file, the default value is 0.05.   

The program \texttt{read\_jsolver} generates equilibrium and stores in the file \texttt{POLAR}. Given the input file \texttt{POLAR}, \texttt{m3dc1\_meshgen} generates the following files:
\begin{itemize}
\item	model.dmg: PUMI-readable model file
\item	model.txt: M3DC1-readable model file
\item	mesh0.smb: PUMI/M3DC1-readable mesh file
\item	mesh.vtk: Paraview data files
\item	norm\_curv: ascii file containing nodes' normal/curvature information
\end{itemize} 

%%%%%%%%%%%%%%%%%%%%%%%%%%%%%%%%%%%%%%%%%
\subsection{Mesh Control with SimModeler}
SimModeler is a graphical user interface to the Simmetrix geometry and mesh generation software. In cases where the currently available capabilities of m3dc1\_meshgen do not provide a satisfactory mesh, SimModeler can be used to apply alternative mesh control information to the Tokamak cross section geometry to generate different meshes. The information below indicates the application of a subset of the mesh controls that can be applied. For additional information of the full range of SimModeler mesh control options see: ********** FILL IN POINTER TO SIMMETRIX DOCUMENTATION *****
\label{ch:simmodeler}
%%%%%%%%%%%%%%%%%%%%%%%%%%%%%%%%%%%%%%%%%
(Contributed by D. Pfefferle on 4/27/16) On PPPL Portal, load a module \texttt{simmodeler} and run it.
\begin{enumerate}
\item From the menu \texttt{"File$\rightarrow$Open Model"}, load a model file (\texttt{.smd}) generated by \texttt{m3dc1\_meshgen}
\item In the upper panel, in the views section, click on \texttt{Front} to view the model, then go to \texttt{Meshing} tab
\item Select outer region, click \texttt{+} in \texttt{Mesh Attributes} and select \texttt{Mesh Size$\rightarrow$relative}. 
Enter a value (typically 0.1)
\item Select wall region, click \texttt{+} in \texttt{Mesh Attributes} and select \texttt{Mesh Size$\rightarrow$relative}. 
Enter a value (typically 0.02)
\item Select inner region, click \texttt{+} in \texttt{Mesh Attributes} and select \texttt{Mesh Size$\rightarrow$relative}. 
Enter a value (typically 0.04). Here, one can already generate the mesh by clicking on \texttt{Generate Mesh} and verify if the mesh sizes are suitable 
\item 	Select both inner and wall regions (holding shift key), click \texttt{+} in \texttt{Mesh Attributes} and select \texttt{Mesh Size$\rightarrow$relative}. Enter a function, e.g. \texttt{0.01$\times$abs(\$y+1.5)\^{}2+0.004} to specify an anisotropic mesh density on top of previous settings
\\ There are many available parameters for fine-tuning the mesh density.  For example, \texttt{Mesh Curvature Refinement} with parameter packs more elements near the edges of the resistive wall. 
\item \texttt{Generate Mesh} and \texttt{Show Mesh} to view result in new windows
\item If the result is satisfactory, from the menu \texttt{File$\rightarrow$Save Mesh}, give it a meaningful name with the extension \texttt{.sms}. The original model file \texttt{.smd} has been automatically saved by the program with your mesh modifications.
\item Close \texttt{simmodeler} then it will release a license. Until you quit Simmodeler, no one cannot run neither \texttt{m3dc1\_meshgen} nor \texttt{simmodeler}.
\item Copy the \texttt{.txt, .smd} and \texttt{.sms} files to the simulation directory and run the following splitting routine to obtain PUMI-readable \texttt{.smb} mesh files.
\newline\newline
\texttt{/p/tsc/C1/m3dc1-sunfire.r6-1.5/bin/part\_mesh.sh model\_file.smd mesh\_file.sms X}, 
where \texttt{X} is the number of parts you need in the \texttt{.smb} mesh.
\item Modify the \texttt{C1input} file accordingly
\newline\newline
\texttt{mesh\_filename = `part.smb'
\\
mesh\_model = `filename.txt'
}
\end{enumerate}

%%%%%%%%%%%%%%%%%%%%%%%%%%%%%%%%%%%%%%%%%
\subsection{Model and Mesh conversion from Simmetrix to M3DC1-PUMI files (simToM3dc1)}
\label{ch:simToM3dc1}
%%%%%%%%%%%%%%%%%%%%%%%%%%%%%%%%%%%%%%%%%
To support geometries directly imported from modeling kernels such as Parasolid, a tool is developed that converts the geometries and associated meshes in a format that is supported in M3DC1. This workflow needs following two steps.
\begin{enumerate}
  \item Load the native model files from Parasolid model files (.x\_t, .xmt\_txt, .x\_b, .xmt\_bin) in Simmodeler. This converts Parasolid model file to a Simmetrix model file. Use the Simmodeler mesh controls ( see section ~\ref{ch:simmodeler}) to create a mesh of desired resolution. This gives us a Simmetrix model (.smd) and a Simmetrix mesh (.sms).
  \item  Use simToM3dc1 to convert Simmetrix model and mesh from above step to PUMI model (.dmg), PUMI mesh (.smb) and file (.txt) that contains the additional model information since PUMI model(.dmg) only holds adjacency information. 
\end{enumerate}
To use simToM3dc1, an input file file is required to feed all the neccessary information to the convertor. The parameters required in simToM3dc1 are given below:
\begin{itemize}
\item sim\_model: Input Simmetrix model (.smd)
\item sim\_mesh: Input Simmetrix mesh (.sms)
\item native\_model: Native Parasolid file (.x\_t, .xmt\_txt, .x\_b, .xmt\_bin). Required if model was created from a Parasolid model. If model and mesh were not created from Parasolid (or any other geometric kernel) and were created directly in Simmetrix, then we do not need this parameter.
\item out\_file (optional): Output files name. If this parameter is not provided, default name (output.smb, output.dmg, output\_modelInfo.txt) is used. 
\item inner\_face: Innermost model face number in Simmetrix model. Check model face \# from Simmodeler GUI.
\item outer\_face: Outermost model face number in Simmetrix model. Check model face \# from Simmodeler GUI.
\end{itemize}

The normal vectors and curvature on the mesh vertices at boundary loops are stored as tags on the PUMI meshes. Since Simmodeler does not provide any informaton on loops numbers, model face numbers are needed in the input file to define the inner and outer most model faces in the domain. From model face information, desired loop information is extracted. The outputs files are: 

\begin{itemize}
  \item PUMI model (.dmg)
  \item PUMI mesh (.smb)
  \item A .txt file (Output file name\_modelInfo.txt) that contains the following additional model information
    \begin{itemize}
      \item Minimum of bounding box (x\_min, y\_min)
      \item Maximum of bounding box (x\_max, y\_max)
      \item Number of model edges on Innermost loop
      \item model edges Ids on Innermost loop
      \item Number of model edges on Outermost loop
      \item model edges Ids on Outermost loop
    \end{itemize}
\end{itemize}

A test case with all the needed inputs, and usage information is provided in the M3DC1 repository. Check the following link: 
\newline \url{https://github.com/PrincetonUniversity/M3DC1/tree/master/meshgen/test/data/sim-parasolid-testcase}. 
\newline Check the input file to see how to set up the parameters and readme.md for the usage. 
%%%%%%%%%%%%%%%%%%%%%%%%%%%%%%%%%%%%%%%%%
\subsection{Mesh Partitioning}
\label{ch:mesh-ptn}
%%%%%%%%%%%%%%%%%%%%%%%%%%%%%%%%%%%%%%%%%

\subsubsection{Splitting}

The program \texttt{split\_smb} increases the number of parts in a mesh from \texttt{P} to \texttt{N} (\texttt{P$<$N}). 
In each machine, the program \texttt{split\_smb} is availble in \texttt{\$SCOREC\_UTIL\_DIR} provided in \texttt{hostname.mk} file.

In order to split \texttt{P}-part mesh to \texttt{N} parts (\texttt{N$>$P}), run
\texttt{"mpirun -np N ./split\_smb input-mesh(.smb) output-mesh(.smb) X"}
\begin{itemize}
\item	the file extension of input-mesh should be .smb 
\item	the file extension of output-mesh should be .smb
\item	\texttt{N} is the number of parts in the output mesh
\item	For a \texttt{P}-part input mesh, \texttt{X} must be \texttt{N/P}
\item	For both input and output mesh, do not specify a number before the file extension
\item	\texttt{split\_smb} will insert a number in the output mesh file. The number represents a global part ID.
\item	Make sure that the output mesh doesn't have any empty part. Otherwise, the program crashes with the following error message:
\newline
\texttt{APF warning: 1 empty parts}
\newline
\texttt{split\_smb: \ldots/mds/mds.c:614: check\_ent: Assertion `e $>$= 0' failed}
\end{itemize}

Examples on portal:
\begin{enumerate}
\item To split a serial (1-part) mesh to 6 parts, run\\
 \texttt{"mpirun -np 6 ./split\_smb struct-curveDomain.smb part.smb 6"}
\begin{itemize}
\item	Input mesh: struct-curveDomain0.smb 
\item	Output mesh: part0.smb, part1.smb, part2.smb, part3.smb, part4.smb, part5.smb
\end{itemize}

\item To split a 2-part mesh to 6 parts, run
 \texttt{"mpirun -np 6 ./split\_smb  struct-curveDomain.smb part.smb 3"}
\begin{itemize}
\item	Input mesh: struct-curveDomain0.smb, struct-curveDomain1.smb
\item	Output mesh: part0.smb, part1.smb, part2.smb, part3.smb, part4.smb, part5.smb
\end{itemize}
\end{enumerate}

See \texttt{readme.split\_smb} for detailed instructions and trouble shooting tips.

%%%%%%%%%%%%%%%%%%%%%%%%%%%%%%%%%%%%%%%%%
\subsubsection{Mesh Merging}
\label{ch:mesh-mg}
%%%%%%%%%%%%%%%%%%%%%%%%%%%%%%%%%%%%%%%%%

The program \texttt{collapse} decreases the number of parts in a mesh from \texttt{N} to \texttt{P} (\texttt{P$<$N}). 
In each machine, the program \texttt{collapse} is availble in \texttt{\$SCOREC\_UTIL\_DIR} provided in \texttt{hostname.mk} file.

In order to merge \texttt{N}-part .smb mesh to \texttt{P} parts (\texttt{P$>$0}), run
\texttt{"mpirun -np N ./collapse input-mesh(.smb) output-mesh(.smb) X"}
\begin{itemize}
\item	the file extension of input-mesh should be .smb 
\item	the file extension of output-mesh should be .smb
\item	\texttt{N} is the number of parts in the input mesh
\item	For a \texttt{P}-part output mesh, \texttt{X} must be \texttt{N/P}
\item	For both input and output mesh, do not specify a number before the file extension
\item	\texttt{collapse} will insert a number in the output mesh file. The number represents a global part ID.
\end{itemize}

Example on portal:
\newline
In order to merge 4-part mesh into a serial (1-part) mesh, run
\texttt{"mpirun -np 4 ./collapse part.smb serial.smb 4"}
\begin{itemize}
\item	Input mesh: part0.smb, part1.smb, part2.smb, part3.smb
\item	Output mesh: serial0.smb
\end{itemize}

See \texttt{readme.collapse} for detailed instructions and trouble shooting tips.

%%%%%%%%%%%%%%%%%%%%%%%%%%%%%%%%%%%%%%%%%
\subsection{Miscellaneous}
\label{ch:mesh-misc}
%%%%%%%%%%%%%%%%%%%%%%%%%%%%%%%%%%%%%%%%%

\subsubsection{Verification}

The program \texttt{check\_smb} investigates an input mesh and prints any invalid aspects of the mesh. It prints out the mesh size (the number of global, local, and owned entities per dimension) at the end. 

In order to run, do \texttt{"mpirun -np N ./check\_smb input-mesh(.smb)"}.


\include{mesh-adapt}

\include{running_jobs}

\section{PETSc Options}
\label{petscoption}

\subsection{2D}

\noindent
The petsc option to run 2D modes is default to superlu\_dist. You can add the following line on your "srun" command line
to change to to mumps:

\begin{verbatim}
-pc_factor_mat_solver_type mumps
\end{verbatim}

\subsection{3D}

\noindent
When running M3D-C1 in the 3D nonlinear mode, you need to include PETSc Options file.
There is a number ``8'' in the file below. It must be equal to the number of toroidal planes. It
should be changed whenever you change the number of toroidal planes in the C1input file. The recommended
{\it options\_bjacobi} file is as follows:

\begin{verbatim}
-pc_type bjacobi
-pc_bjacobi_blocks 8
    (for 8 toroidal planes should be equal to nplanes in C1input)
-sub_pc_type lu
-sub_pc_factor_mat_solver_package superlu_dist
    (can exchange mumps for superlu_dist)
-mat_superlu_dist_rowperm NOROWPERM (only needed for superlu_dist)
-mat_mumps_icntl_14 50 (only needed for mumps)
    (50 means 50% of memory increase when needed.
     Users can make it 100 or more if encountering a runtime memory issue.)
-sub_ksp_type preonly
-ksp_type fgmres
-ksp_gmres_restart 220
-ksp_rtol 1.e-9
-ksp_max_it 10000
-on_error_abort

-hard_pc_type bjacobi
-hard_pc_bjacobi_blocks 8
    (for 8 toroidal planes…should be equal to nplanes in C1input)
-hard_sub_pc_type lu
-hard_sub_pc_factor_mat_solver_type superlu_dist
    (can change mumps for superlu_dist)
-mat_superlu_dist_rowperm NOROWPERM
    (only needed for superlu_dist)
-mat_mumps_icntl_14 50
    (only needed for mumps.)
    (50 means 50% of memory increase when needed.
     Users can make it 100 or more if encountering a runtime memory issue.)
-hard_sub_ksp_type preonly
-hard_ksp_type lgmres
-hard_ksp_lgmres_argument 4
-hard_ksp_gmres_restart 220
-hard_ksp_rtol 1.e-9
-hard_ksp_max_it 10000
\end{verbatim}

\subsection{More}

\noindent
More examples are in regtest/pellet/base/ directory, such as 

\begin{verbatim}
options_bjacobi.type_superludist
options_bjacobi.type_mumps 
\end{verbatim}

\noindent
The following are additional optional arguments:

\begin{verbatim}
-ksp_converged_reason
-ksp_view

-help

-options_table
-options_left

-trdump
-malloc_log
\end{verbatim}

for diagnosing purpose.


\include{output}

\section{Numerical Methods}

\subsection{Flux / Potential Representation}

The magnetic field is represented using a vector potential
\begin{equation}
  \vec{A} = R^2 \nabla \varphi \times \nabla f + \psi \nabla \varphi
\end{equation}
This vector potential satisfies the gauge condition $\nabla_p \cdot
\left( \vec{A} / R^2 \right) = 0$.  where
\[
\nabla_p \equiv \nabla - \nabla \varphi\, \partial_\varphi
\]
The magnetic field is

\begin{eqnarray}
  \vec{B} & = & \nabla \times \vec{A}\\
          & = & \nabla \psi \times \nabla \varphi + F \nabla \varphi - \nabla_p f'
\end{eqnarray}
where $f' \equiv \partial_\varphi f$ and
\begin{equation}
  F \equiv R^2 \nabla_p^2 f
\end{equation}
The current density is
\begin{eqnarray}
  \vec{J} & = & \nabla \times \vec{B}\\
          & = & \nabla \left(F + f'' \right) \times \nabla \varphi
 - \Delta^*\psi \nabla \varphi + R^{-2} \nabla_p \psi'
\end{eqnarray}
where
\[
\Delta^* \psi \equiv R^2 \nabla \cdot \left( \nabla_p \psi / R^2 \right)
\]

The velocity is defined as
\begin{equation}
  \vec{u} = R^2 \nabla U \times \nabla \varphi 
+ R^2 \omega \nabla \varphi 
+ R^{-2} \nabla_p \chi
\end{equation}


\subsection{Finite Elements}

\subsubsection{Integration quadrature}

The integrals required to calculate the weak-form equations of the
Galerkin method are computed numerically using an $n$-point Gaussian
quadrature.  That is, the value of each field is calculated at 79
points for each triangular element, and a weighted sum of these values
is computed to approximate the integral.
\[
  \int dA\ f(x) \simeq \sum_{i=1}^{n} w_i f(x_i),
\]
where the integrand $f(x)$ is restricted to a single element.  The
sampling points and weights appropriate for a equilateral triangle are
taken from ref.~\cite{Dunavant85}.

The coordinates of the sampling points are given in the ``natural
coordinates'' $(\alpha, \beta, \gamma)$ in ref.~\cite{Dunavant85}.
These coordinates may be converted to cartesian coordinates $(r, Z)$
for an equilateral triangle $e$ having vertices
\[
\left\{\left(-\frac{\sqrt{3}}{2},-\frac{1}{2}\right),
\left(\frac{\sqrt{3}}{2},-\frac{1}{2}\right), (0,1)\right\} 
\]
using the linear transformation
\[ 
\phi_{n \to e}(\alpha, \beta, \gamma) = 
\left(\frac{\sqrt{3}}{2}(\beta-\gamma), \frac{1}{2}(3 \alpha - 1)
\right).
\]
The weights must be multiplied by the Jacobian of this transformation,
\[
\mathcal{J}_{\phi_{n \to e}} = \frac{3 \sqrt{3}}{4}.
\]
To find the coordinates of the sampling points for a general triangle
$g$ having vertices $\{(-b,0), (a,0), (0,c)\}$, as in
ref.~\cite{Jardin04}, one may use the linear transformation
\begin{eqnarray*}
  \phi_{e \to g}(r,Z) & = & 
    \left(\frac{a+b}{\sqrt{3}} x + \frac{a-b}{3} (1-y), 
    \frac{c}{3}(2y+1) \right) \\
  \mathcal{J}_{\phi_{e \to g}} & = &  \frac{2 c}{3 \sqrt{3}} (a+b).
\end{eqnarray*}
The transformation from natural coordinates to cartesian coordinates
for a triangle having vertices $\{(-b,0), (a,0), (0,c)\}$ is therefore
\begin{eqnarray*}
  \phi_{n \to g}(r,Z) & = & 
  \left(\frac{1}{2} (a+b) (\beta-\gamma) +
  \frac{1}{2} (a-b)(1-\alpha), c \alpha \right) \\
  \mathcal{J}_{\phi_{n \to g}} & = & \frac{1}{2} (a+b) c.
\end{eqnarray*}

The 79-point quadrature gives the exact results for integrands which
are polynomials of degree 20 (or less).  In the case of quintic finite
elements, this means the integration is exact for terms involving
products of three fields or fewer, not including the degree-five
trial function $\nu$.  In cylindrical geometry, the presence factors
of $1/R$ will cause the quadrature not to be exact, as $1/R$ is not in
the form of a polynomial.  The weights $w_i$ must also be multiplied
by $R_i$ in cylindrical coordinates to account for the Jacobian of the
transformation from cartesian to cylindrical coordinates.


\subsection{Time Step}

For the implicit time advance, the fields are evaulated at the
$\theta$-advanced time (\emph{e.g.} $F(\psi) \to F(\psi + \theta \dt
\dot{\psi} + \cdots)$), linearized (\emph{i.e.} $\order{\dt^2}$ and
higher are dropped), and then discretized temporally according to the
chosen time integration method (\emph{i.e.} $\dot{\psi} \to
(\psi^{(n+1)} - \psi^{(n)})/\dt$).


\begin{eqnarray}
  \begin{pmatrix}
    \cola{S^v_{1 1}} & \cola{R^v_{1 1}} &
    \colb{S^v_{1 2}} & \colb{R^v_{1 2}} & 
          S^v_{1 3}  &        0         &
          R^v_{1 4}  &       R^v_{1 3}
    \\
    \cola{R^B_{1 1}} & \cola{S^B_{1 1}} &
    \colb{R^B_{1 2}} & \colb{S^B_{1 2}} & 
          R^B_{1 3}  &       S^B_{1 3}  &
              0      &        0
    \\
    \colb{S^v_{2 1}} & \colb{R^v_{2 1}} & 
    \colb{S^v_{2 2}} & \colb{R^v_{2 2}} & 
          S^v_{2 3}  &        0         &
	  R^v_{2 4}  &       R^v_{2 3}
    \\
    \colb{R^B_{2 1}} & \colb{S^B_{2 1}} &
    \colb{R^B_{2 2}} & \colb{S^B_{2 2}} & 
          R^B_{2 3}  &       S^B_{2 3}  &
              0      &        0
    \\
          S^v_{3 1}  &       R^v_{3 1}  &
          S^v_{3 2}  &       R^v_{3 2}  &
          S^v_{3 3}  &        0         &
	  R^v_{3 4}  &       R^v_{3 3}  
    \\
          R^B_{3 1}  &       S^v_{3 1}  &
          R^B_{3 2}  &       S^v_{3 2}  &
          R^B_{3 3}  &       S^v_{3 3}  &
              0      &        0
    \\
          R^n_{3 1}  &        0         &
          R^n_{3 2}  &        0         &
          R^n_{3 3}  &        0         &
          S^n        &        0
    \\
          R^p_{3 1}  &        0         &
          R^p_{3 2}  &        0         &
          R^p_{3 3}  &        0         &
              0      &       S^p        &
  \end{pmatrix}
  \begin{pmatrix}
    \cola{U}\\ \cola{\psi}\\ 
    \colb{V}\\ \colb{F}   \\
    \chi \\ p_e \\ 
    n \\ p
  \end{pmatrix}^{(n+1)} = \nonumber \\
  \begin{pmatrix}
    \cola{D^v_{1 1}} & \cola{Q^v_{1 1}} &
    \colb{D^v_{1 2}} & \colb{Q^v_{1 2}} & 
          D^v_{1 3}  &        0         &
          Q^v_{1 4}  &       Q^v_{1 3}
    \\
    \cola{Q^B_{1 1}} & \cola{D^B_{1 1}} &
    \colb{Q^B_{1 2}} & \colb{D^B_{1 2}} & 
          Q^B_{1 3}  &       D^B_{1 3}  &
              0      &        0
    \\
    \colb{D^v_{2 1}} & \colb{Q^v_{2 1}} & 
    \colb{D^v_{2 2}} & \colb{Q^v_{2 2}} & 
          D^v_{2 3}  &        0         &
	  Q^v_{2 4}  &       Q^v_{2 3}
    \\
    \colb{Q^B_{2 1}} & \colb{D^B_{2 1}} &
    \colb{Q^B_{2 2}} & \colb{D^B_{2 2}} & 
          Q^B_{2 3}  &       D^B_{2 3}  &
              0      &        0
    \\
          D^v_{3 1}  &       Q^v_{3 1}  &
          D^v_{3 2}  &       Q^v_{3 2}  &
          D^v_{3 3}  &        0         &
	  Q^v_{3 4}  &       Q^v_{3 3}  
    \\
          Q^B_{3 1}  &       D^v_{3 1}  &
          Q^B_{3 2}  &       D^v_{3 2}  &
          Q^B_{3 3}  &       D^v_{3 3}  &
              0      &        0
    \\
          Q^n_{3 1}  &        0         &
          Q^n_{3 2}  &        0         &
          Q^n_{3 3}  &        0         &
          D^n        &        0
    \\
          Q^p_{3 1}  &        0         &
          Q^p_{3 2}  &        0         &
          Q^p_{3 3}  &        0         &
              0      &       D^p        &
  \end{pmatrix}
  \begin{pmatrix}
    \cola{U}\\ \cola{\psi}\\ 
    \colb{V}\\ \colb{F}   \\
    \chi \\ p_e \\ 
    n \\ p
  \end{pmatrix}^{(n)} +   
  \begin{pmatrix}
    Q_1 \\ Q_2 \\ 
    Q_3 \\ Q_4 \\
    Q_5 \\ Q_6 \\ 
    Q_7 \\ Q_8
  \end{pmatrix}
\end{eqnarray}

\subsubsection{Split Time Step Method}



Time is advanced using a split time-step method in which the velocity
field is advanced first, then the density and total pressure fields
are advanced separately, and finally the magnetic field and electron
pressure are advanced together.  Though the velocity and magnetic
field are advanced separately, the Alfv\'en and magnetosonic waves are
treated implicitly by using the pressure equation and Faraday's law to
calculate analytically the advanced-time values of the pressure and
magnetic field for use in the velocity time step.



\begin{eqnarray}
  \label{eq:velocity_advance}
  \lefteqn{
  \begin{pmatrix}
    \cola{S^v_{1 1}} & \colb{S^v_{1 2}} & S^v_{1 3}\\
    \colb{S^v_{2 1}} & \colb{S^v_{2 2}} & S^v_{2 3}\\
          S^v_{3 1}  &       S^v_{3 2}  & S^v_{3 3}\\
  \end{pmatrix} 
  \begin{pmatrix}
    \cola{U}\\ \colb{V}\\ \chi
  \end{pmatrix}^{(n+1)}}\\
  & = & 
  \begin{pmatrix}
    \cola{D^v_{1 1}} & \colb{D^v_{1 2}} & D^v_{1 3}\\
    \colb{D^v_{2 1}} & \colb{D^v_{2 2}} & D^v_{2 3}\\
          D^v_{3 1}  &       D^v_{3 2}  & D^v_{3 3}\\
  \end{pmatrix} 
  \begin{pmatrix}
    \cola{U}\\ \colb{V}\\ \chi
  \end{pmatrix}^{(n)}
  + 
  \begin{pmatrix}
    \cola{Q^v_{1 1}} & \colb{Q^v_{1 2}} & Q^v_{1 3}\\
    \colb{Q^v_{2 1}} & \colb{Q^v_{2 2}} & Q^v_{2 3}\\
          Q^v_{3 1}  &       Q^v_{3 2}  & Q^v_{3 3}\\
  \end{pmatrix} 
  \begin{pmatrix}
    \cola{\psi}\\ \colb{F}\\ p
  \end{pmatrix}^{(n)} \nonumber
  \\ & & \mbox{} + 
  \begin{pmatrix}
    \cola{O^v_{1}}\\
    \colb{O^v_{2}}\\
          O^v_{3} \\
  \end{pmatrix} \nonumber
\end{eqnarray}

\begin{eqnarray}
  \label{eq:density_advance}
  S^n n^{(n+1)} & = & D^n n^{(n)} + 
  \begin{pmatrix} R^n_1 & R^n_2 & R^n_3\end{pmatrix}
  \begin{pmatrix}\cola{U}\\ \colb{V}\\ \chi\end{pmatrix}^{(n+1)}
  \\ & & \mbox{} + 
  \begin{pmatrix}Q^n_1    &   Q^n_2   & Q^n_3\end{pmatrix}
  \begin{pmatrix}\cola{U}\\ \colb{V}\\ \chi\end{pmatrix}^{(n)} \nonumber
\end{eqnarray}


\begin{eqnarray}
  \label{eq:pressure_advance}
  S^p p^{(n+1)} & = & D^p p^{(n)} + 
  \begin{pmatrix}R^p_1 & R^p_2 & R^p_3\end{pmatrix}
  \begin{pmatrix}\cola{U}\\ \colb{V}\\ \chi \end{pmatrix}^{(n+1)}
  \\ & & \mbox{} + 
  \begin{pmatrix}Q^p_1 & Q^p_2 & Q^p_3\end{pmatrix}
  \begin{pmatrix}\cola{U}\\ \colb{V}\\ \chi \end{pmatrix}^{(n)} \nonumber
\end{eqnarray}


\begin{eqnarray}
  \label{eq:field_advance}
  \lefteqn{
  \begin{pmatrix}
    \cola{S^B_{1 1}} & \colb{S^B_{1 2}} & S^B_{1 3}\\
    \colb{S^B_{2 1}} & \colb{S^B_{2 2}} & S^B_{2 3}\\
          S^B_{3 1}  &       S^B_{3 2}  & S^B_{3 3}\\
  \end{pmatrix} 
  \begin{pmatrix}
    \cola{\psi}\\ \colb{F}\\ p_e
  \end{pmatrix}^{(n+1)}}\\
  & = & 
  \begin{pmatrix}
    \cola{D^B_{1 1}} & \colb{D^B_{1 2}} & D^B_{1 3}\\
    \colb{D^B_{2 1}} & \colb{D^B_{2 2}} & D^B_{2 3}\\
          D^B_{3 1}  &       D^B_{3 2}  & D^B_{3 3}\\
  \end{pmatrix} 
  \begin{pmatrix}
    \cola{\psi}\\ \colb{F}\\ p_e
  \end{pmatrix}^{(n)} +
  \begin{pmatrix}
    \cola{R^B_{1 1}} & \colb{R^B_{1 2}} & R^B_{1 3}\\
    \colb{R^B_{2 1}} & \colb{R^B_{2 2}} & R^B_{2 3}\\
          R^B_{3 1}  &       R^B_{3 2}  & R^B_{3 3}\\
  \end{pmatrix} 
  \begin{pmatrix}
    \cola{U}\\ \colb{V}\\ \chi
  \end{pmatrix}^{(n+1)} \nonumber
  \\ & & \mbox{} +
  \begin{pmatrix}
    \cola{Q^B_{1 1}} & \colb{Q^B_{1 2}} & Q^B_{1 3}\\
    \colb{Q^B_{2 1}} & \colb{Q^B_{2 2}} & Q^B_{2 3}\\
          Q^B_{3 1}  &       Q^B_{3 2}  & Q^B_{3 3}\\
  \end{pmatrix} 
  \begin{pmatrix}
    \cola{U}\\ \colb{V}\\ \chi
  \end{pmatrix}^{(n)} +
  \begin{pmatrix}
    \cola{O^B_{1}}\\
    \colb{O^B_{2}}\\
          O^B_{3} \\
  \end{pmatrix} \nonumber
\end{eqnarray}


%\include{doc}

% Bibliography
%\bibliographystyle{plain}
%\bibliography{m3dc1}
\printbibliography

\appendix
%\include{idl-postproc}
%\section{SCOREC API}
This appendix chapter should document all SCOREC/ITAPS functions called from M3D-C1. 
In this document the term process is used instead of processor since multiple processes
may be run on a single processor.  Currently it is assumed that there is
one partition per process.  In Fortran, the standard convention
is that a function does not modify any passed in variables and returns a value. 
In this appendix a function follows C/C++ conventions in both syntax and functionality. 
Functions listed below should be thought of as Fortran subroutines as they are allowed
to modify the passed in parameters and do not return any value. 
The datatypes listed below in the C/C++ functions are:
\begin{description}
\item[int *] A 4 byte integer in Fortran.
\item[int[X]] A 4 byte integer array that should be allocated of at least length \textit{X}.
\item[char *] A character string in Fortran.
\item[double *] An 8 byte double precision/real*8 variable in Fortran. When using complex variables this may also be a 16 byte double precision complex number.  
\item[double[X]]  An 8 byte double precision/real*8 variable array in Fortran of at least length \textit{X}.  When using complex variables this may also be a 16 byte double precision complex array of \textit{X}*16 bytes length.
\end{description}
To make clear what is discussed below, a mesh vertex, edge, and face are the topological
entities of a mesh.  Many people associate nodes of a mesh with mesh vertices and elements with
the mesh faces for 2D meshes.  

The \textit{clearscorecdata} function cleans up everything associated with the SCOREC libraries.
Beyond this, there are 5 main sets of functions: model, mesh, ordering, vector and matrix.  The model
interface can be thought of as an independent object and is used to represent the geometric domain independent
of any mesh discretizing it.  The mesh only depends on the model that it discretizes.  The ordering
depends on the mesh and the model only and gives numberings of the degrees-of-freedom (DOFs) associated
with mesh entities (for the reduced quintic DOFs are only associated with mesh vertices).  The vector
and matrix interface depends mainly on the ordering but indirectly on the model and mesh as well.
 The two main functions
are to load a mesh and model which is used with the \textit{loadmesh} function below.  
\begin{center}
\begin{figure}
%\centerline{\psfig{figure=./partitioned_mesh.eps,height=2.5in,angle=0}} %for epsfig
\caption{A partitioned mesh with the DOFs locally numbered for the blue partition with one DOF per node and iper=1 and jper=0.}\label{meshpartition} \end{figure}
\end{center}


\begin{itemize}
\item void loadmesh( char * modelfilename, char * meshfilename); This function
loads a model file called \textit{modelfilename} which should be ``struct.dmg'' 
and a mesh file called \textit{meshfilename} which should be ``struct-dmg.sms''.
\item void clearscorecdata(); This function clears all SCOREC software data.
\end{itemize}


\subsection{Model Interface}
\begin{itemize}
\item  void getmodeltags(int * bottom, int * right, int * top, int * left); This
function assumes that the domain is rectangular shaped and returns the four model entity 
IDs for the sides of the rectangle.  The main use of this function is for setting boundary 
conditions on DOFs associated with mesh entities classified on the boundary.  The mesh entity
classification is obtained through the \textit{zonedg} function 
for mesh edges and \textit{zonenod} function for
mesh nodes/vertices.
\item  void setperiodicinfo(int* xperiodic, int* zperiodic); Sets the flag that periodic
boundary conditions exist in the horizontal direction if \textit{xperiodic} is anything but zero
and that periodic
boundary conditions exist in the vertical direction if \textit{zperiodic} is anything but zero.
\item  void getperiodicinfo(int * xperiodic, int* zperiodic);	Gets the flag that periodic
boundary conditions exist.  If \textit{xperiodic} is one than they have been set in
the horizontal direction and zero if they have not been. If
 \textit{zperiodic} is one than they have been set in
the vertical direction and zero if they have not been.
\item  void getmincoord(double* xmin, double* zmin); Gets the minimum coordinate dimensions of the geometric domain. 
\textit{xmin} is the minimum horizontal coordinate value and \textit{zmin} is the minimum vertical coordinate value
returned by the function.
\item  void getmaxcoord(double* xmax, double* zmax); Gets the maximum coordinate dimensions of the geometric domain.  
\textit{xmax} is the maximum horizontal coordinate value and \textit{zmax} is the maximum vertical coordinate value
returned by the function.
\end{itemize}

\subsection{Mesh Interface}
\begin{itemize}
\item void numnod( int *NumNodes ) ; Returns the number of nodes, \textit{NumNodes}, in the process's partition of the mesh.  Since
 some nodes  exist on partition boundaries (i.e. shared by multiple processes) summing \textit{NumNodes} over all
processes will result in more nodes than exist in the total mesh. The IDs of the nodes on a process are numbered between
one and \textit{NumNodes}.
\item void xyznod( int *iNode , double X[3] ) ; For input node ID \textit{iNode}, this function returns the coordinate location in 
the \textit{X} array.  The \textit{X} array should be of at least length 3.
\item void numfac( int *NumFaces ) ; Returns the number of faces, \textit{NumFaces}, in the process's partition of the mesh. 
For a 2D mesh, the faces will all be elements. Since
 any face for a 2D mesh  exists only on  one partition,  summing \textit{NumFaces} over all
processes will result in the total number of faces that exist in the mesh. The IDs of the faces on a process are numbered between
one and \textit{NumFaces}.
\item void nodfac( int *iFace, int Nodes[4] ) ; For an input face ID, \textit{iFace}, this function returns the adjacent nodal IDs
in the \textit{Nodes} array.  
For mixed topology meshes the faces could be triangles or quadrilaterals so it is safest to have the \textit{Nodes} array to be of
length four.  The nodes' IDs are returned in counter-clockwise order and if the face is a triangle then only the first three
values in the \textit{Nodes} array contain usable data.
\item void numglobalents(int * numnodes, int * numedges, int * numfaces, int * numregions); This function returns
the global number of nodes, edges, faces, and regions in the mesh in \textit{numnodes, numedges, numfaces}, and \textit{numregions}.
\item  void zonenod( int *iNod , int *iZone , int * iZoneDim); This function returns the model entity ID in 
\textit{iZone} and the model entity dimension in \textit{iZoneDim} for passed in mesh vertex \textit{iNod}.
\item void zonedg( int *iEdge , int *iZone , int * iZoneDim ); This function returns the model entity ID in 
\textit{iZone} and the model entity dimension in \textit{iZoneDim} for passed in mesh edge \textit{iEdge}.

\item void createsearchstructure(); This function sets up a search structure on each mesh partition.  
\item void deletesearchstructure(); This function deletes the search structure on each mesh partition.
\item void usesearchstructure(double* x, double* y, int* iFace); This function returns the face ID \textit{iFace}
of a mesh face that contains the given point \textit{(x,y)}.  The point \textit{(x,y)} is given in the
global coordinate system of the mesh.  The function returns -1 for \textit{iFace} if the point is not contained
in the domain of any mesh faces on the local process's partition.  Note  that multiple faces may contain
a given point (e.g. a mesh vertex coordinate).
\item void getelmsizes(int * iFace, double sizes[4]); For mesh face \textit{iFace}, this function returns an array
of doubles for the mesh size at each of the mesh vertices of the face.  
\end{itemize}





\subsection{Ordering Interface}
It should be noted that when using complex numbers that the orderings will act the same regardless
of the data type. As an example, for DOF number \textit{i}, the double precision array value will 
be stored at \textit{d(i)} and the complex array value will be stored at \textit{c(i)}.  Since
C and C++ have no intrinsic complex datatype, the complex arrays will be treated as double arrays
and the real part will be accessed at \textit{d[(i-1)*2]} and the imaginary part will be
accessed at \textit{d[(i-1)*2+1]}.

\begin{itemize}
\item   void createdofnumbering(int * numberingid, int * iper, int * jper,
                           int * dofspervertex, int * dofsperedge, 
                           int* dofsperface, int * dofsperregion, int * numdofs);
 This function creates a numbering with  ID \textit{numberingid}.  Currently, \textit{iper}
and \textit{jper} are not used as the periodic information is read in from the C1input file during the 
call to \textit{loadmesh}.
Eventually though if some orderings are periodic and others are not (e.g. a potential function)
than these options can be used to specify the difference in orderings. The other parameters
passed in (\textit{dofspervertex, dofsperedge, dofsperface,dofsperregion}) are to specify
how many DOFs per mesh entity type (e.g. \textit{dofspervertex} would be 18 for a numvar=3 
vector and 0 for the other mesh entity types).  The only returned value is \textit{numdofs} 
which is the total number of DOFs associated with mesh entities that exist on the process that
the function is called on. This is also the allocated size of a vector on this process.  Note
though that a global sum of \textit{numdofs} on all processes will be greater than the total number
of global DOFs unless it is a single process run. 

\item void deletedofnumbering(int * numberingid);
This function deletes the numbering associated with \textit{numberingid}.  It is an error
to query the SCOREC software with \textit{numberingid} after it has been deleted
unless another numbering is created
with this id.


\item  void entdofs(int * numberingid, int * meshentid, int * meshentdim,
               int * begindofnumber, int * enddofnumberplusone);
     For \textit{numberingid}, of topological dimension \textit{meshentdim} with mesh entity
ID \textit{meshentid} this function outputs \textit{begindofnumber} which is the beginning of
the DOF array location (stored contiguously) and \textit{enddofnumberplusone} is one
beyond the last DOF array location.  Note that the returned array index values are in Fortran
ordering convection such that the first entry of the array is at index 1.
For a numvar=1 ordering an example would be \textit{begindofnumber}=1
and \textit{enddofnumberplusone}=7. 

\item  void numdofs(int * numberingid, int * numdofs); This function returns the total number of 
DOFs associated with all mesh entities existing on a specific mesh partition.  For the example
in Figure \ref{meshpartition}, the blue partition would have \textit{numdofs} equal to 10 as both DOF 1
and 2 are periodic DOFs and only get counted once even though the number of nodes for this 
partition is 12.
  
\item  void numglobaldofs(int * numberingid, int * numglobaldofs);
This function returns the global number of DOFs \textit{numglobaldofs} for numbering \textit{numberingid}. 
 For the mesh in Figure \ref{meshpartition} with given periodicity \textit{numglobaldofs} would be 19.

\end{itemize}
\subsection{Vector Interface}
It is the responsibility of M3D-C1 to allocate and deallocate memory for vectors.  The \textit{space}
subroutine in M3D-C1 is used for allocating and deallocating all vectors of DOFs of fields
that are to be transferred during mesh adaptation.  This is used as a ``callback'' function to 
manage the memory of these arrays.  The proper amount of memory to be allocated for each vector
is obtained by using the \textit{numdofs} function for the same numbering id that is used
for the vector.

On the SCOREC side of the software, all arrays are considered to be arrays of double precision numbers.
Determining if they are actual double precision numbers (8 bytes) or complex double precision numbers
(16 bytes) is done through the \textit{type} parameter for both vector and matrix interfaces 
(\textit{itype}=0 indicates real-valued and \textit{itype}=1 indicates complex-valued).  

\begin{itemize}
\item   void createppplvec(double */complex * vectorid, int * orderingid, int * type); This function creates a vector 
for \textit{vectorid}
 for numbering \textit{orderingid}. If the vector is for a real-valued array \textit{type}
should be set to 0 and if the vector is for  a complex-valued array \textit{type} should be set to 1. 
\item   void deleteppplvec(double */complex * vectorid); This function deletes the vector but it is M3D-C1's 
responsibility to deallocate the memory used for the array.
\item   void sumsharedppplvecvals(double */complex * vectorid);  This function assumes that the values inserted into the
array of the vector has only local contributions on each process and then this sums the parts
of the vector that are distributed across multiple processes and updates each of those values
to a single value.  An example of this would be to figure out how many mesh faces use each DOF.  To do this, 
first zero the vector, then 
iterate over all of the faces on each process, then iterate over each mesh vertex, get the DOFs
associated with each vertex and add 1 to each array indexed with the DOF number, then call this function.
For this operation, for DOF number 10 in Figure \ref{meshpartition} the number would be 7.
\item   void updatesharedppplvecvals(double */complex * vectorid);  This function assumes that the single process that ``owns'' a DOF
has the correct value and that other processes that need the DOF may or may not have the correct value.  With
this assumption, this function can be called to update the correct values for a DOF on processes
that need the DOF but do not own it.  This is used primarily for distributing the DOF values after a matrix solve
from the process it was solved for on to the other processes that use the DOF.  This function might not be useful
for M3D-C1 but is documented anyways.
\item   void checkppplveccreated(double */complex * vectorid, int * iscreated); This function checks whether a vector \textit{vectorid} 
has already been created or not.  It returns 0 if it has not and 1 if it has been created but does no
action otherwise.  
\item void checksameppplvec(double */complex * id1, double */complex * id2, int * i); This function checks if vectors 
\textit{id1} and \textit{id2} are the same, and sets *i=1 if true, *i=0 if false
\end{itemize}

\subsection{Matrix Interface}
Note that some of these functions may have a ``2'' at the end of their names to make sure
that modifications done to get to using complex variables was done with less errors. As was done with
vectors, \textit{type} is used to indicate real-valued matrices (\textit{type}=0) and complex-valued
matrices (\textit{type}=1).
\begin{itemize}
\item  void zerosuperlumatrix(int * matrixid, it * type, int * numberingid);  This function creates
an ``empty'' matrix with id \textit{matrixid} that uses the numbering \textit{numberingid}
for use with the SuperLU\_DIST solver.  If the matrix is real-valued then \textit{type}
should be set to 0 and if the matrix is complex-valued then \textit{type} should be set to 1. 
If the matrix has already been created then it just cleans out all components of the matrix.
\item  void zeropetscmatrix(int * matrixid, int * type, int * numberingid); This function creates
an ``empty'' matrix with id \textit{matrixid} that uses the numbering \textit{numberingid}
for use with the SuperLU\_DIST solver.  If the matrix is real-valued then \textit{type}
should be set to 0 and if the matrix is complex-valued then \textit{type} should be set to 1. 
If the matrix has already been created then it just cleans out all components of the matrix.
 \item void zeromultiplymatrix(int * matrixid, int * type, int * numberingid); This function creates
an ``empty'' matrix with id \textit{matrixid} that uses the numbering \textit{numberingid}
for use for multiplying with vectors.  If the matrix is real-valued then \textit{type}
should be set to 0 and if the matrix is complex-valued then \textit{type} should be set to 1. 
If the matrix has already been created then it just cleans out all components of the matrix.
\item  void insertval(int * matrixid, double */complex * val, int * valtype, int * row, 
		   int * column, int * operation);  This function inserts \textit{val} into
matrix \textit{matrixid} at \textit{(row,column)} in the matrix where \textit{row}
and \textit{column} come from the ordering.  The type of value to
 be inserted can be real (\textit{valtype}=0)
or complex (\textit{valtype}=1).  A real type can be inserted into a complex matrix but
a complex type cannot be inserted into a real matrix.  If \textit{operation}
is zero then the value overwrites any existing value, otherwise the value is to be added
to existing values for that matrix component.
\item  void setdiribc(int * matrixid, int * row);  For matrix \textit{matrixid}, this function
zeroes out all off-diagonal values for \textit{row} and sets the diagonal value to unity.  The operation is
actually carried out during \textit{finalizematrix} so this function can be called before other values
are inserted into that row while still applying the correct operation.  For complex-valued arrays, 
only the real part of the diagonal is set to unity while the imaginary part is set to 0.  This function 
should be called on all processes that use the DOF number associated with the matrix row.
\item  void setgeneralbc(int * matrixid, int * row, int * numvals,
		     int  columninfo[numvals], double/complex vals[numvals], int * type);
This function sets multiple values for \textit{row} of \textit{matrixid}.  The number of values set is
\textit{numvals} and \textit{columninfo} specifies which columns to set the values for and 
\textit{vals} is the values to be set which must be in the same order as the \textit{columninfo} array.
 The type of values to
 be inserted can be real (\textit{type}=0)
or complex (\textit{type}=1). If only real values are inserted into a complex matrix then the corresponding
imaginary parts are set to zero.
This function 
should be called on all processes that use the DOF number associated with the matrix row.
\item  void finalizematrix(int * matrixid);  This function finalizes \textit{matrixid} such that no
more values can be inserted into the matrix and no more boundary conditions can be applied to the matrix.
\item  void solve(int * matrixid, double */complex * rhs\_sol, int * ier);  For a matrix created with \textit{zeropetscmatrix}
or \textit{zerosuperlumatrix}, \textit{matrixid} is solved with input right-hand-side \textit{rhs\_sol}. 
The output  is also in \textit{rhs\_sol}.  If \textit{ier} is non-zero there were problems encountered during
the solve process.  The only way to properly mix real and complex value types is if the matrix is real-valued
and the vector is complex-valued.  Then the system is solved twice with a real and an imaginary right-hand-side.  
\item  void matrixvectormult(int * matrixid, double */complex * inputvecid, double */complex * outputvecid);  For a matrix created with \textit{zeromultiplymatrix}, this function can be called to multiply \textit{matrixid} with \textit{inputvecid}.  The output is returned in \textit{outputvecid}.  If either \textit{matrixid} or \textit{inputvecid}
is complex-valued, \textit{outputvecid} must also be complex-valued.
\item  void deletematrix(int * matrixid);  This function deletes the matrix with ID \textit{matrixid}.
\item  void matrixrank(int* matrixid, int * rank);  This function returns the global rank of \textit{matrixid}
in \textit{rank}.  This value is the same as the returned value for \textit{numglobaldofs} for the matrix's
ordering.
\item  void initsolvers();  This function sets up the data structures required by the solvers.
\item  void finalizesolvers();  This function finalizes/destroys the data structures required by the solvers.
\item  void writematrixtofile(int * matrixid, int * fileid);  This function writes out the non-zero components
of \textit{matrixid} to a file that uses \textit{fileid} to indicate which file.
\end{itemize}
For the PETSc solver, the following two functions are defined in \textit{PETScInterface.c} and are designed
to give M3D-C1 the ability to modify the PETSc solver parameters for a given \textit{matrixid}.  
The FortranMatrixID enumeration must match the parameters specified in the sparse module in \textit{M3Dmodules.f90}.
\begin{itemize}
\item  int setPETScMat(int matrixid, Mat * A);  This function can be used to set options for storing \textit{matrix}
in PETSc data structures.
\item int setPETScKSP(int matrixid, KSP * ksp, Mat * A); This function can be used to set the solver
parameters for \textit{matrixid} in PETSc.
\end{itemize}

\subsection{Creating Meshes for M3D-C1}
There are a couple of ways to create meshes for M3D-C1 simulations.  

The first procedure for generating an initial mesh is using the 
structMesh.cc code in 
M3D-C1 in the Util subdirectory (i.e. M3D/Util/structMesh.cc).  This code
 creates a 2D structured mesh that can be used with FMDB.  Compilation is done with
 \textit{$<$C++ compiler$>$ -o structMesh.x structMesh.cc} and the input for the code is
\textit{./structMesh.x $<$number of nodes in the x-direction$>$ $<$number of nodes in the z-direction$>$ 
$<$x-length or rectangular domain$>$ $<$z-length of rectangular domain$>$}
Note that the domain is 0 $<$= x $<$= x-length or rectangular domain , 0 $<$= z $<$= z-length of rectangular domain.
After running structMesh.x, the output will be a file called struct.sms.  To be
able to use this with M3D-C1, this file must be processed by running the program
in \textit{$<$SCOREC SOFTWARE PATH$>$/mctk/Examples/PPPL/PPPL/test/DISCRETE/main} with the argument struct.sms 
(e.g. $<$SCOREC SOFTWARE PATH$>$/mctk/Examples/PPPL/PPPL/test/DISCRETE/main struct.sms)
which will output a struct-dmg.sms and struct.dmg file which can be used to
run M3D-C1.

The second procedure for generating an initial mesh is to use the Simmetrix (www.simmetrix.com)
mesh generation tools.  Some code already exists at SCOREC for creating meshes for PPPL 
using the Simmetrix
software and MCTK but this code is not available on PPPL machines or bassi.nersc.gov as the
Simmetrix software API is copyrighted and not available for general public distribution.  The two main
desired functionalities that the Simmetrix software provides are generating unstructured meshes
for an inputed analytic sizefield and the ability to create ``matching meshes'' for
periodic boundary conditions (e.g. meshes
that match vertex locations on periodic model boundaries).  The Simmetrix mesh generation tools
specific to PPPL are located in /users/acbauer/develop/mctk/generate\_mesh/generate\_mesh/test
subdirectories but have not been added to the MCTK repository since this is specific to PPPL work.
The build system is the usual SCOREC build system available in 
/users/acbauer/develop/mctk/generate\_mesh/generate\_mesh/
along with the executables in the test subdirectory.  Running these executables typically
is done by \textit{./main $<$ACIS/Parasolid model file$>$ $<$size parameter$>$ $<$curvature parameter$>$}
and the output is a mesh in both Simmetrix file format (*\_SimModS.sms) and
FMDB file format (*\_FMDB.sms).  The FMDB file format must then be processed as the struct.sms file
for the structured mesh generation tool above to create the struct.dmg and struct-dmg.sms files.

The third procedure for creating an adapted mesh from an M3D-C1 run is to use the 
\textit{hessianadapt} function in the SCOREC software.  The function parameters in C/C++ are:\\
\textit{void hessianadapt(double */complex * ppplvector, int * which, int * type, int * ntime, double * factor, double * hmin, double * hmax);}\\
Here, \textit{ppplvector} is a pointer to an array of double precisions or complex double precisions
used to store the DOFs of the desired fields, \textit{which} is used to indicate the specific field
since ppplvector may store DOFs for more than a single field and is a number between 1 and the total
number of fields that store their DOFs in the ppplvector, 
\textit{type} indicates whether to use the real part or imaginary part if the DOFs are complex valued, 
\textit{ntime} is the time step to adapt the field with respect to (the field may have been created
and stored in SCOREC software already from a previous time step and this is used to make sure that
the adaptation is done with respect to the most current time step),
\textit{factor} is the refinement factor, \textit{hmin} is the desired minimum edge length in the adapted mesh, 
and \textit{hmax} is the desired maximum edge length in the adapted mesh.  Note that the actual
minimum and maximum desired edges lengths may be smaller and larger, respectively, than the
desired values.  The \textit{factor, hmin,} and \textit{hmax} parameters are the same as the ones
used in Ken Jansen's phAdapt software.  This adaptation procedure only works with single process
runs of M3D-C1 and the adapted mesh is classified on the same model (struct.dmg) 
as the pre-adapted mesh.  During mesh adaptation, all fields that exist as PPPLVectors are transferred
using interpolation during local mesh modifications.  These values are then passed back to M3D-C1 by
resizing the arrays of DOFs that M3D-C1 uses and then setting the DOF values for these arrays.  
The resizing of the DOF arrays will be done in the /textit{space} subroutine in M3D-C1.


% appendix on modules that are used on different machines that M3D-C1 is installed on
\subsubsection{Module use on different machines}
This appendix is meant to make sure that people do not have trouble determining which modules to use on 
which machines.  Note that this list may include modules that are not necessary for M3D-C1.

\subsubsection{viz/m3d.pppl.gov}
The modules that I (Andy Bauer) have loaded on viz and m3d are  
 intel\_fc/9.0.033,
 intel\_cc/9.0.032,
 ncarg/4.4.1,
 hdf5/1.6.5,
 intel\_mkl/8.1.014,
   superlu\_dist\_2.0/20060102,
   intel\_vt/8.0.245,
   subversion/1.3.0,
   /parmetis/3.1,
  /autopack/1.3.2,
  petsc/2.3.3-p3,
  superlu\_3.0/20060102, and
  /zoltan/3.0 .

The commands that I use to load them are:
\begin{itemize}
\item      module load intel\_fc/9.0.033
\item       module load intel\_cc/9.0.032
\item       module load ncarg
\item       module load hdf5
\item       module load intel\_mkl/8.1.014
\item       module load superlu\_dist\_2.0/20060102
\item       module load intel\_vt
\item       module load subversion
\item       module load intel\_cc parmetis
\item       module load intel\_cc autopack
\item       module load petsc
\item       module load superlu\_3.0
\item       module load zoltan
\end{itemize}

\subsubsection{bassi.nersc.gov}
The modules that I (Andy Bauer) have loaded on bassi are 
null
   parmetis/3.1
   superlu\_dist/2.0\_64
   hdf5\_par/1.6.4
   netcdf/3.6.2
   ncar/4.4.2
   lapack/3.0
   petsc/2.3.3\_O
   zoltan/3.0 .

The commands that I use to load them are:
\begin{itemize}
\item module load parmetis
\item module load superlu\_dist
\item module load hdf5\_par
\item module load netcdf
\item module load ncar
\item module load petsc/2.3.3\_O
\item module load lapack
\item module load zoltan
\end{itemize}


\include{units}
%%%%%%%%%%%%%%%%%%%%%%%%%%%%%%%%%%%%%%%%%%
\section{Input Parameters}
\label{sec:input_parameters}
%%%%%%%%%%%%%%%%%%%%%%%%%%%%%%%%%%%%%%%%%%
\subsection{Model Options}

\begin{tabular}{llp{4in}}
  \textbf{Option}&\textbf{Default}&\textbf{Description}\\
  \hline
  \texttt{numvar} & 3   & MHD model. 1: 2-field;  2: 4-Field;  3: 6-Field.\\
  \texttt{linear} & 0   & 1: linear (perturbation terms only, no matrix
  recalculation)\\
  \texttt{eqsubtract}& 0& 1: remove equilibrium terms from equations\\
  \texttt{icsubtract}& 0& set to 1 if PF coils are in the domain.  These are
  defined in the files ``coils.dat'' and ``current.dat''\\
  \texttt{extsubtract} & 0 & 1: subtract fields from non-axisymmetric coils \\
  \texttt{idens}  & 1   & 1: include density equation\\
  \texttt{ipres}  & 0   & 1: include electron pressure equation\\
  \texttt{ipressplit} & 0 & 1: seperate pressure solve from the magnetic field
  solves when isplitstep=1.  (ipressplit must be 0 for isplitstep=0) \\
  \texttt{itemp} & 0 & 1: Advance temperatures rather than pressures (for ipressplit=1 only) \\
  \texttt{gyro}   & 0   & 1: include Braginskii gyroviscous term.  (note:
  needs db to be nonzero also) \\
  \texttt{igauge} & 0   & 0: loop voltage applied to boundary psi only \\
  \texttt{inertia} & 1  & 1: Include $\u \cdot \grad{\u}$ terms\\
  \texttt{itwofluid}& 1 & 1: Include $\j\times\B$ and
  $\grad{p_e}$ terms in Ohm's law (electron form).  2: ion form (not
  recommended)  3: parallel pressure gradient in Ohm's law only
  (not recommended) \\
  \texttt{ibootstrap} & 0 & 1: Using bootstrap coefficients as a function of $\psi$; \\
                      &   & 2: Using bootstrap coefficients as a function of $T_e$ \\
                      &   & 3: Using bootstrap coefficients as a function of $\hat{T}=1-T_e/\max{T_e}$, coefficients are updated within M3D-C1 \\
  \texttt{ibootstrap\_model} & 0 & 1: Sauter (with eqsubtract = 0/1); \\
                             &   & 2: Redl (with eqsubtract = 0/1); \\
                             &   & 3: Sauter Model (with eqsubtract = 0); \\
                             &   & 4: Redl Model (with eqsubtract = 0) \\
  \texttt{ibootstrap\_regular} & 1e-8 & regularization term used in bootstrap current calculations \\
  \texttt{bootstrap\_alpha} & 1 & Amplification factor for bootstrap current \\
  \texttt{imp\_bf} & 0 & 1: include implicit equation for f (recommended for
  3D and 2D complex) \\
  \texttt{nosig} & 0 & 1: drop sigma terms from momentum equation \\
  \texttt{itor}   & 0   & 0: cartesian; 1: cylindrical\\
  \texttt{istatic}& 0   & 1: Do not advance velocity\\
  \texttt{iestatic}&0   & 1: Do not advance magnetic fields\\
  \texttt{chiiner} & 1. & factor to multiply the chi equation inertial terms \\
  \texttt{ieq\_bdotgradt} & 1. & 1: include equilibrium parallel T gradient \\
  \texttt{no\_vdg\_T} & 0 & 1: do notinclude V dot grad T in Temp equation (debug) \\
  \texttt{iwall\_is\_limiter} & 1 & 1: wall acts as limiter \\
  \texttt{kinetic} & 0 & 1: Use kinetic PIC for hot pressure, 
                         2: Incompressible CGL,
                         3. Full CGL  
\end{tabular}

\begin{tabular}{llp{4in}}
  \textbf{Option}&\textbf{Default}&\textbf{Description}\\
  \hline
  \texttt{irunaway} & 0 & 1:  include runaway electron model \\
  \texttt{cre} & 0 & runaway speed \\
  \texttt{imp\_temp} & 0 & 0: compute temperatures for isplitstep=0, itemp=0 \\
  \texttt{iohmic\_heating} & 1 & 1: Include Ohmic heating term\\
  \texttt{irad\_heating} & 1 & 1: Include radiation heat sink \\
  \texttt{gravr} & 0 & gravitational acceleration in R-direction \\
  \texttt{gravz} & 0 & gravitational acceleration in Z-direction 
\end{tabular}

\subsection{Initial Conditions Options}

\begin{tabular}{llp{4in}}
  \textbf{Option}&\textbf{Default}&\textbf{Description}\\
  \hline
  \texttt{itaylor} & 0 & \begin{minipage}[t]{2.5in}
    Pre-defined initial conditions.\\
 {\bf for itor=1 (toroidal geometry)} \\ 
    0: Tilting cylinder \\
    1: Calls Grad-Shafranov solver \\
    2: magneto-rotational equilibrium \\
    3: rotational instability \\   
    40: Fixed boundary stellarator \\
    41: Free boundary stellarator \\
 {\bf for itor=0 (slab geometry) } \\
    0: Tilting cylinder\\
    1: Taylor Reconnection\\
    2: Force-Free equilibrium (Taylor state) \\
    3: GEM Reconnection\\
    4: Wave Propagation\\
    5: Gravitational Instability\\
    6: Strauss equilibrium \\
    7: circular field init \\
    8,9:  biharmonic \\
    10,11,12,13:: analytic RWM test problem \\
    14: 3D wave test \\
    15: 3D diffusion test \\
    16:  FRS cylindrical equilibrium \\
    17:  ftz init \\
    18:  eigen init \\
    19:  ASDEX profiles similar to YU's \\
    20: kstar profiles with multiple q=1 surfaces \\
    21,22: fixed q(r) and p(r) profiles \\
    23:  Startsev equilibrium with $ J = (1/R_0q_0)(1 - r^2)$ \\
    27:  cylindrical test problem \\
    29:  basicJ profiles \\
  \end{minipage}\\
  \texttt{iupstream} & 0 & 1: addsdiffusion term to convection-like upstream differencing \\
  \texttt{magus}  & 5.e-2 & magnitude of the upstream diffusion term \\
  \texttt{iflip}    &  0 & 1: Flip coordinate system handedness\\
  \texttt{iflip\_b} &  0 & 1: Flip sign of toroidal field\\
  \texttt{iflip\_j} &  0 & 1: Flip sign of toroidal current\\
  \texttt{iflip\_v} &  0 & 1: Flip sign of toroidal velocity\\
  \texttt{iflip\_z} &  0 & 1: Flip equilibrium across z=0 plane \\
\end{tabular}

\begin{tabular}{llp{4in}}
  \textbf{Option}&\textbf{Default}&\textbf{Description}\\
  \hline
  \texttt{icsym}    &  0 &  
    \begin{minipage}[t]{2.5in}
    Symmetry of random perturbations \\
    0: No symmetry\\
    1: Odd up-down symmetry (in $U$)\\
    2: Even up-down symmetry (in $U$)
  \end{minipage}\\

  \texttt{bzero} & 1      & $B_\tor$ at \texttt{rzero}\\
  \texttt{bx0}  & 0 & Initial field in x-direction for some test problems \\
  \texttt{vzero} & 0 & Initial toroidal velocity for some test problems \\
  \texttt{phizero} & 0 & Initial velocity stream function for some test problems \\
  \texttt{v0\_cyl} & 0 & Central toroidal velocity for some test problems \\
  \texttt{v1\_cyl} & 0 & VZ=v0\_cyl + v1\_cyl*psin**beta \\
 \texttt{idevice}    &  0 &
    \begin{minipage}[t]{2.5in}
    define coils for a particular device \\
    -1: reads coil.dat file \\
    0: generic dipole configuration \\
    1: CDX-U \\
    2: NSTX \\
    3: ITER \\
    4: DIII-D
  \end{minipage}\\
  \texttt{iwave} & 0 & defines what wave to initialize in wave propagation test \\ 
  \texttt{eps}      &  0.01 & Size of random perturbation\\
  \texttt{maxn}     &  200 & Maximum wavenumber of initial random noise\\
  \texttt{verzero}  & 0 & magnitude of initial vertical velocity \\
   \texttt{irmp}    &  0 &
    \begin{minipage}[t]{2.5in}
    1: apply nonaxisymmetric fields throughout plasma.  
       reads rmp\_coil.dat for (R,Z) of window pane coils.  
       reads rmp\_current.dat for (+-) currents in kA and phases in degrees.  
       toroidal mode number of current specified by ntor.
       requires \texttt{type\_ext\_field} = 0.  \\
    2: apply nonaxisymmetric fields only at boundaries. 
   \end{minipage}\\
   \texttt{type\_ext\_field}    &  -1 & External field type
    \begin{minipage}[t]{2.5in}
    0: RMP or error field for tokamak geometry.
    1: For free boundary stellarator only: FIELDLINES or MGRID.
   \end{minipage}\\

 \texttt{rmp\_atten}  &  0  & additional exponential decay of RMP field from r=1 for irmp=2 \\
 \texttt{iread\_ext\_field} & 0 & 1: read external field \\
 \texttt{beta}  & 0 & parameter used in some model equilibrium initializations \\
 \texttt{ln} & 0 & length scale parameter used in some model equilibrium \\
 \texttt{elongation} & 1 & elongation used in Solovev equilibrium

\end{tabular}

\begin{tabular}{llp{4in}}
  \textbf{Option}&\textbf{Default}&\textbf{Description}\\
  \hline

 \texttt{isample\_ext\_field} & 1 & factor to down sample external field data toroidally \\
 \texttt{isample\_ext\_field\_pol} & 1 & factor do down sample external field data poloidally \\
 \texttt{scale\_ext\_field} & 1 & factor to scale external field \\
 \texttt{shift\_ext\_field} & 0 & toroidal shift (in deg) of external fields \\
 \texttt{ibasicj\_solvep} & 0 & 0: uniform p, solve for F; 1: uniformF, solve for p \\
 \texttt{basicj\_nu} &1 & exponent in basicj equilibrium \\
 \texttt{basicj\_j0} & 1 & On-axis current density in basicj equilibrium \\
 \texttt{basicj\_voff} & 1 & Radial extent of flat toroidal rotation in basicj equilibrium \\
 \texttt{basicj\_vdelt} & 1 & Width of velocity drop-off, as fraction of ln, in basicj equilibrium \\
 \texttt{basicj\_dexp} & 1 & parameter for basicj equilibrium \\
 \texttt{basicj\_dvac} & 1 & parameter for basicj equilibrium \\
 \texttt{basicj\_q0} & 0 &   parameter for basicj equilibrium \\
 \texttt{basicj\_qa}  & 0 &   parameter for basicj equilibrium \\
 \texttt{pf\_shift} & 0 & (array) horizontal shift of PF coil \\
 \texttt{pf\_shift\_angle} & 0 & (array) direction of PF shift in degrees \\
 \texttt{pf\_tilt} & 0 & (array) Angle of PF from vertical in degrees \\
 \texttt{pf\_tilt\_angle} & 0 & (array) Axis of rotation for PF tilt in degrees \\
 \texttt{tf\_shift} & 0 & horizontal shift of TF coils \\
 \texttt{tf\_shift\_angle} & 0 & direction of TF shift in degrees \\
 \texttt{tf\_tilt} & 0 & angle of TF from vertical in degrees \\
 \texttt{tf\_tilt\_angle} & 0 & axis of rotation for TF tilt in degrees 

\end{tabular}

\subsection{Grad-Shafranov Solver Options}
\begin{tabular}{lcp{4in}}
  \textbf{Option}&\textbf{Default}&\textbf{Description}\\
  \hline
  \texttt{inumgs}& 0      & 1: Use numerical def. of p and g from profile-p and profile-g files\\
 \texttt{igs}   & 80     & Max number of Picard iterations\\
  \texttt{eta\_gs} & 1000.& factor for smoothing nonaxisymmetries in psi in 3D GS solve \\
  \texttt{igs\_pp\_ffp\_rescale} & 0 & 1: rescale p' and FF' to match p and F \\
  \texttt{nv1equ}& 0 & 1:use numvar =1 equilibrium for numvar .GT. 1 \\
  \texttt{tcuro} & 1	  & (scaled) plasma current in GS equilibrium\\
  \texttt{xmag}  & 1      & $R$-coordinate of magnetic axis\\
  \texttt{zmag}  & 0      & $Z$-coordinate of magnetic axis\\
  \texttt{xmag0} & 0      &  if nonzero, target magnetic axis $R$ for feedback\\
  \texttt{zmag0} & 0      &  if nonzero, target magnetic axis $Z$ for feecback\\
  \texttt{xlim}  & 0      & $R$-coordinate of limiter\\
  \texttt{zlim}  & 0      & $Z$-coordinate of limiter\\
  \texttt{xlim2}  & 0      & $R$-coordinate of limiter \#2\\
  \texttt{zlim2}  & 0	   & $Z$-coordinate of limiter \#2\\
  \texttt{rzero}  & 1      & nominal major radius of device for itor=1 \\
  \texttt{libetap}& 1.2    & approximate value of $l_i/2 + \beta_P$ for free-boundary equ \\
  \texttt{p0}    & 0.01   & Pressure at magnetic axis\\
  \texttt{pi0}   & 0.005  & Ion pressure at magnetic axis\\
  \texttt{p1}    & 0     & $p^{\prime}(\Psi)$ at magnetic axis\\
  \texttt{p2}    & 0     & $p^{\prime \prime}(\Psi)$ at magnetic axis\\
  \texttt{pedge} & 0	  & Pressure in vacuum region\\
  \texttt{tedge} & 0     & temperature in vacuum region (if .GT. 0).  Only
                           used in GS solve.   Boundary value of electron temp
                           is $twall = pedge \times pefac/den\_edge $ \\
 \texttt{tiedge} & 0     & ion temperature in vacuum region  \\
 \texttt{expn}  & 0 & \parbox[t]{4in}{Fraction of pressure gradient due to
    density gradient: $n = p^\mathtt{expn}$.}\\
 \texttt{q0}    & 1	  & Safety factor at magnetic axis\\
 \texttt{djdpsi}& 0	  & $J_\tor'(\Psi)$ at magnetic axis\\
 \texttt{th\_gs}& 0.8     & implicitness of GS Picard iterations\\
 \texttt{tol\_gs}& $10^{-8}$  & convergence criteria for GS iteration \\
  \texttt{pscale}  & 1.       & factor multiplying pressure profile \\
  \texttt{bscale}      &  1.0 & Factor multipying toroidal field\\
  \texttt{bpscale}     &  1.0 & Factor multiplying F' (keeping F0 constant) \\
  \texttt{vscale}      &  1.0 & Factor multiplying toroidal rotation profile \\
  \texttt{iread\_bscale}&  0   & 1: read profile\_bscale for factor to scale F \\
  \texttt{iread\_pscale} & 0   & 1: read profile\_pscale for factor to scale $p$ and $p^{\prime} $ \\

\end{tabular}

\begin{tabular}{llp{4in}}
  \textbf{Option}&\textbf{Default}&\textbf{Description}\\
  \hline
  \texttt{batemanscale} &  1   & Bateman scale the TF, keeping curent profile fixed \\
  \texttt{irot}         &  0   & 1: include toroidal rotation in equilibrium calculatin \\
  \texttt{iscale\_rot\_by\_p} & 1 & see below \\
  \texttt{alpha0}       &  0   & $\alpha_0$ in analytic rotation profile \\
  \texttt{alpha1}      	&  0   & $\alpha_1$ in analytic	rotation profile \\
  \texttt{alpha2}      	&  0   & $\alpha_2$ in analytic	rotation profile \\
  \texttt{alpha3}      	&  0   & $\alpha_3$ in analytic	rotation profile \\
\end{tabular}

For iread\_omega=0, the function $\alpha(\psi)$ is parameterized by:
 \[ \tilde{\alpha} = \alpha_{0} + \alpha_{1} s + \alpha_{2} s^{2} + \alpha_{3} s^{3} \]
For iscale\_rot\_by\_p = 0:  $\alpha = \tilde{\alpha} \times n(\psi) / p(\psi)$ . \\
For iscale\_rot\_by\_p = 1:  $\alpha = \tilde{\alpha} $. \\
For iscale\_rot\_by\_p = 2:  $\alpha = \left[ \alpha_{0} + \alpha_{1} e^{-\left[ \left( \psi - \alpha_{2} \right) / \alpha_{3} 
                                          \right]^{2} }       \right] \times n(\psi) /p(\psi) $ \\
In all cases, the angular velocity is then determined by:
\[      \omega = \left[  \frac{2 \alpha p(\psi)}{R_0^2 n(\psi)} \right]^{\frac{1}{2}} \]
\begin{tabular}{llp{4in}}
  \texttt{idenfunc}         & 0   &
  \begin{minipage}[t]{4.0in}
    0: $ n = \mbox{den0} \times (p/p0)^{\mbox{expn}} + \mbox{denedge} $ \\
    1: $ n = \mbox{den0} \times \frac{1}{2} \times 
       \left[1 + \tanh \left(\frac{\psi - (\psi_B + n_O (\psi_B - \psi_M))}
                                  {\Delta \times (\psi_B - \psi_M)        } \right)    \right] $ \\
    2: $ n = \mbox{den0} + \frac{1}{2}  \left( \mbox{den\_edge} - \mbox{den0} \right)
                  \times  \left[1 + \tanh \left( \frac{ \tilde{\psi} - n_O}
                                               {      \Delta            } \right)  \right] $\\
    3: if $\tilde{\psi}$ .LT. $n_O$ and $(\psi - \psi_M) \times \left[d \psi /dx (x - x_{MA}) + d \psi /dz (z - z_{MA} )           
                \right] $ .GT. 0, then $n$ = den0.     Else, $n$ = den\_edge.  \\
    ( $\psi_B = \mbox{psibound}, \psi_M = \mbox{psimin}, \tilde{\psi} = (\psi - \psi_M)/(\psi_B - \psi_M) $ )
  \end{minipage}    \\

  \texttt{den\_edge}        & 0.0 & edge density.  If 0, set to den0*(pedge/p0)**expn \\
  \texttt{den0}             & 1.0 & (scaled) central density\\
  \texttt{denoff}           & 1.0 & $n_O$: offset for idenfunc=1,2,3 \\
  \texttt{dendelt}          & 0.1 & $\Delta$: width of transition region for idenfunc=1,2 \\

  \texttt{divertors} & 0  & Number of divertors (0--2)\\
  \texttt{divcur}& 0.1    & Divertor current(s), as fraction of tcuro\\
  \texttt{xdiv}  & 0      & $r$-coordinate of divertor current(s)\\
  \texttt{zdiv}  & 0      & \parbox[t]{4in}{$z$-coordinate of
    divertor 
    current.  If $\mathtt{divertors} = 2$, the second divertor has 
    $z = -\mathtt{zdiv}$.}\\
  \texttt{xnull}     & 0 & Guess for $r$-coordinate of x-point\\
  \texttt{znull}     & 0 & Guess for $z$-coordinate of x-point\\
  \texttt{mod\_null\_rs} & 0 & if 1: you can reset xnull and znull from C1input \\
  \texttt{xnull0}  &  0  & Target R-Coordinate of x-point for feedback \\
  \texttt{znull0}  &  0  & Target Z-Coordinate of x-point for feedback \\
  \texttt{xnull2}     & 0 & Guess for $r$-coordinate of inactive x-point\\
  \texttt{znull2}     & 0 & Guess for $z$-coordinate of inactive x-point\\
  \texttt{mod\_null\_rs2} & 0 & if1: you can reset xnull2 and znull2 from C1input \\
  \end{tabular}

  \begin{tabular}{llp{4.0in}}
  \textbf{Option}&\textbf{Default}&\textbf{Description}\\
  \hline
 \texttt{gs\_pf\_psi\_width}            & 0 & width of psi smoothing into provate flux region \\
 \texttt{gs\_vertical\_feedback}        & 0 & proportional feedback of each coil to (zmag-zmag0) (array) \\ 
 \texttt{gs\_vertical\_feedback\_i}     & 0 & integral feedback of each coil to (zmag-zmag0) (array) \\
 \texttt{gs\_vertical\_feedback\_x}     & 0 & proportional feedback of each coil to (znull-znull0) (array) \\
 \texttt{gs\_vertical\_feedback\_x\_i} & 0 & integral feedback of each coil to (znull-znull0) (array) \\
 \texttt{gs\_radial\_feedback}          & 0 & proportional feedback of each coil to (xmag-xmag0) (array) \\
 \texttt{gs\_radial\_feedback\_i}       & 0 & integral feedback of each coil to (xmag-xmag0) (array) \\
 \texttt{gs\_radial\_feedback\_x}       & 0 & proportional feedback of each coil to (xnull-xnull0) (array) \\
 \texttt{gs\_radial\_feedback\_x\_i}    & 0 & integral feedback of each coil to (xnull-xnull0) (array) \\
 \texttt{igs\_extend\_p}                & 0 & 1: extend p past pls=1 using ne and Te profiles  \\
 \texttt{igs\_feedfac}                  & 1 & proportionality factor for external field feedback \\
 \texttt{igs\_forcefree\_lcfs}          & -1 & 1: ensure that GS solution is force free at LCFS \\
 \texttt{igs\_start\_xpoint\_search}    & 0 &  number of GS iterations before searching for x-point \\
 \texttt{sigma0}                        & 0 &  width of Gaussian for initial current distribution for GS iteration \\
 \texttt{igs\_extend\_diagmag}          & 1 &  1: extend diamagnetic rotation past psi=1 \\
 
\texttt{adapt\_qs} & 0 & Safety factor values to pack around (array) \\
\texttt{adapt\_zlow} & 0 & Z-coordinate below which SOL adaption is coarse \\
\texttt{adapt\_zup}  & 0 & Z-coordinate above which SOL adaptation is coarse


\end{tabular}




\subsection{Transport Coefficients}

\begin{tabular}{llp{4.0in}}
  \textbf{Option}&\textbf{Default}&\textbf{Description}\\
  \hline
  \texttt{ivisfunc} & 0 & select viscosity function \\
                    &   & 0: $ \mbox{visc} = \mbox{amu} $ \\
                    &   & 1: $ \mbox{visc} = \mbox{amu} + \frac{1}{2} \mbox{amu\_edge} \times
                 \left[ 1. + \tanh \left[   \frac{\psi - \left( \psi_l+\nu_0(\psi_l - \psi_0)\right)}
                                                 {\nu_{\Delta} (\psi_l - \psi_0)                   }\right] \right] $  \\
                    &   & 2:  $ \mbox{visc} = \mbox{amu} + \frac{1}{2} \mbox{amu\_edge} \times
                 \left[ 1. + \tanh \left[   \frac{\tilde{\psi} - \nu_0}
                                                 {\nu_{\Delta}}  \right] \right] $  \\
                    &   & or, if amuoff2 .ne. 0 and amudelt2.ne.0) \\ 
                    &   &    $ \mbox{visc} = \mbox{amu} + \frac{1}{4} \mbox{amu\_edge} \times
                     \left[ 2. + \tanh \left[   \frac{\tilde{\psi} - \nu_0}
                                                 {\nu_{\Delta}}  \right]                     
                               + \tanh \left[   \frac{\tilde{\psi} - \nu_{02} }
                                                 {\nu_{\Delta2}}  \right]    \right] $  \\
                    &   & 3:  visc = amu or amu\_edge depending on criteria in define\_fields \\

  \texttt{amu}       & 0 & core viscosity for ivisfunc =0,..,3 \\
  \texttt{amu\_edge} & 0 & edge viscosity for ivisfunc = 1,..,3 \\
  \texttt{amuoff}    & 0 & $\nu_0$ in ivisfunc = 1,2 \\
  \texttt{amuoff2}   &   & $\nu_{02}$ in ivisfunc = 1,2 \\
  \texttt{amudelt}   & 0 & $\nu_{\Delta}$ in ivisfunc = 1,2 \\
  \texttt{amudelt2}  & 0 & $\nu_{\Delta2}$ in ivisfunc = 1,2 \\
  \texttt{amuc}   & 0 & Compressional viscosity coefficient\\
  \texttt{amupar} & 0 & Parallel viscosity coefficient \\
  \texttt{amue}  & 0 & bootstrap viscosity coefficient \\
  \hline
  \texttt{iresfunc} & 0 & select resistivity function  \\
                    &   & 0: eta = etar + eta0/Te**(3/2) \\
                    &   & 1: $ \mbox{eta} = \mbox{etar} + \frac{1}{2} \mbox{eta0} \times
                 \left[ 1. + \tanh \left[   \frac{\psi - \left( \psi_l+\mbox{etaoff} \times (\psi_l - \psi_0)\right)}
                                                 {\mbox{etadelt}\times (\psi_l - \psi_0)                   }\right] \right] $  \\
                    &   & 2:  $ \mbox{eta} = \mbox{etar} + \frac{1}{2} \mbox{eta0} \times
                 \left[ 1. + \tanh \left[   \frac{\tilde{\psi} - \mbox{etaoff}}
                                                 {\mbox{etadelt}}  \right] \right] $  \\
                    &   & The following two options are applied in a way that they \\ 
                    &   & should not have negative values...even if the idl plots \\
                    &   & indicate otherwise \\
                    &   & 3: eta = etar for $\tilde{\psi} < \mbox{etaoff}$ othrwise eta0  \\
                    &   & 4:  Spitzer resistivity with offset. \\
                    &   &     Define $T_{wall}$ = pedge*pefac/den\_edge \\
                    &   & $\mbox{for} T_e > T_{wall} - T_e^{off}, \eta = (T_e-T_e^{off})^{-3/2} $ \\
                    &   & $\mbox{for} T_e < T_{wall} - T_e^{off}, \eta = (T_{wall}-T_e^{off})^{-3/2} $ \\
                    &   & can be increased by inputing eta\_fac $>$ 1.  \\
                    &   & 5:  Simple neoclassical model:  \\
                    &   & $\eta = \mbox{eta0} \times (n_e/p_e)^{3/2} / (1. - 1.46 (r/R)^{1/2})   $ \\
 \texttt{etar}    & 0 & see description of iresfunc \\
  \texttt{eta0}   & 0 & see description of iresfunc \\
  \texttt{etaoff} & 0 & see description of iresfunc \\
  \texttt{etadelt} & 0 & see description of iresfunc \\
  \texttt{eta\_te\_offset} & 0 & $T_e^{off}$  for iresfunc=4 \\
  \texttt{ikprad\_te\_offset} & 0 & if 1, $T_e^{off}$ also used in kprad and ablation \\
  \texttt{eta\_fac} & 0 & for iresfunc=4, resistivity multiplied by eta\_fac \\
  \texttt{eta\_mod} & 0 & if 1: remove d/dphi terms in resistivity \\
\end{tabular}

\begin{tabular}{llp{4.0in}}
  \textbf{Option}&\textbf{Default}&\textbf{Description}\\
  \hline
  \texttt{eta\_max} & 0 & maximum resistivity in plasma (defaults to etavac) \\
  \texttt{eta\_min} & 0 & minimum resistivity in plasma  \\
  \hline
  \texttt{ikappafunc} & 0 &  select thermal conductivity function \\
    & & 0: $\kappa = \mbox{kappat} + \mbox{kappa0} \times * (n^3/p)^{1/2}  $\\
    & & 1: $\kappa = \mbox{kappa0} \times \frac{1}{2}
                    \left[1 + \tanh \left[\frac{\psi -\left( \psi_l + \kappa^{0ff} \times (\psi_l - \psi_0)\right)}
                                               {\kappa_{\Delta} \times (\psi_l - \psi_0)} \right]  \right]$ \\
    & & 2: $\kappa = \mbox{kappa0} \times \frac{1}{2}
                    \left[  1 + \tanh \frac{\tilde{\psi} - \kappa^{off}}
                                           {  \kappa_{\Delta}}  \right] \mbox{for} \tilde{\psi} < 1 $  \\
    & & 2: $\kappa = \mbox{kappa0} \times \frac{1}{2}
                    \left[  1 + \tanh \frac{2 - \tilde{\psi} - \kappa^{off}}
                                           {  \kappa_{\Delta}}  \right] \mbox{for} \tilde{\psi} > 1 $  \\
    & & 3:  $\kappa = \mbox{kappat} + \mbox{kappa0} \times 1/(pn)^{1/2}$ \\
    & & 4:  $\kappa = \mbox{kappat} + \mbox{kappa0} \times ( 1. + \mbox{kappadelt} \times |\nabla T_e|^2 ) $  \\
    & & 5:  $\kappa = \mbox{kappat} + \mbox{kappa0}/T_e$ limited by kappa\_max  \\ 
    & & 10: read from profile\_kappa file in $m^2/\mbox{sec}$  \\ 
    & & 11: read from profile\_kappa file in normalized units  \\
    & & 12: option to go with itaylor=27   \\
 \texttt{kappa\_max}  & 0 & if .NE. 0, max $\kappa$ for ikappafunc=5 \\
 \texttt{kappai\_fac} & 1 & ion thermal conduction is kappai\_fac* kappa \\
 \texttt{ikapscale}   & 0 & if 1: kappar gets scaled by kappa  \\
 \texttt{ikappar\_ni} & 0 & 1: include 1/n terms in parallel heat flux \\
 \texttt{kappaoff}    & 0 & $\kappa^{off}$ see ikappafunc \\
 \texttt{kappadelt}   & 0 & $\kappa_{\Delta} $ see ikappafunc \\
 \texttt{kappat}      & 0 & isotropic thermal conductivity \\
 \texttt{kappa0}      & 0 & see ikappafunc \\
 \texttt{ikapparfunc} & 0 &  select parallel thermal conductivity (PTC) function \\
                      &   & 0: PTC = kappar  \\
                      &   & 1: PTC = $\mbox{kappar}/  \left[ (T_{crit}/T)^{5/2} + 1   \right] $ \\
 \texttt{kappar} & 0 & Parallel thermal conductivity\\
 \texttt{tcrit}  & 0 & $T_{crit}$ for ikapparfunc = 1  \\

 \texttt{kappari\_fac}& 1 & ion parallel thermal conductivity is kappari\_fac x electron value \\
 \texttt{kappax}      & 0 & coefficient of $B \times \nabla T$ temperature diffusion \\
 \texttt{kappah}      & 0 & if nonzero, $\mbox{kappa} = \mbox{kappah} \times \tanh^2 \left[ (\tilde{\psi}-1.)/2  \right] $ \\
  \texttt{kappaf} & 1 & Factor multiplying kappa when $\nabla p < \nabla p_{crit} $ \\
  \texttt{kappag} & 0 & Thermal diffusion proportional to pressure gradient \\
  \texttt{gradp\_crit} & 0 & $\nabla p_{crit} $ for kappaf,kappag model \\
  \texttt{k\_fac} & 1 & Factor by which TF is multiplied in denominator of kappa\_par \\
  \texttt{temin\_q0}  & 0 & Min temperature used in equipartition for ipres=1 \\
  \hline
  \texttt{idenmfunc}   & 0 & selects from of particle diffusion (PD) \\
                      &   &  0: PD = denm \\
                      &   &  1: PD = denm + denmt/Te \\
                      &   &  10: read from file profile\_denm in $m^2/\mbox{sec}$  \\
                      &   &  11: read from file profile\_denm in normalized units \\
  \texttt{denm}   & 0 &   see idenmfunc \\
  \texttt{denmt}  & 0 &  multiplier of 1/Te for idenmfunc=1 \\
  \texttt{denmmin} & 0 & minimum value of denm \\
  \texttt{denmmax} & 1.E6 & maximum value of denm \\
 
\end{tabular}

\subsection{Hyper-Diffusivity}

\begin{tabular}{lcp{4.0in}}
  \textbf{Option} & \textbf{Default} & \textbf{Description}\\
  \hline
  \texttt{imp\_hyper}  & 0 & switch for evaluating hyper resistivity \\
                       &   & 0: $\lambda_H \nabla^2 {\bf J} $ explicit for $\psi$ implicit for F  \\                 \\
                       &   & 1: $\lambda_H \nabla^2 {\bf J} $ implicit for $\psi$ implicit for F  \\                
                       &   & 2: $ ({\bf B}/B^2)\nabla \bullet \lambda_H \nabla \sigma $ implicit for $\psi$ and F    
                                   ($\sigma= {\bf J \bullet B}/B^2 $)  \\                \\
 \texttt{deex}         & 1 & scale length used in hyper \\
 \texttt{hyper}        & 0 & hyper coefficient for $\psi$ equation \\
 \texttt{hyperc}       & 0 & hyper coefficient for poloidal velocity   \\
 \texttt{hyperi}       & 0 & hyper coefficient for toroidal field   \\
 \texttt{hyperp}       & 0 & hyper coefficient for pressure   \\
 \texttt{hyperv}       & 0 & hyper coefficient for toroidal flow   \\
 \texttt{ihypdx}       & 2 & hyper terms multiplied by deex**ihypdx \\
 \texttt{ihypeta}      & 1 & swithc for multipliers of hyper terms   \\
                       & 1 & magnetic field hyper multiplied by eta  \\
                       & 2 & magnetic field hyper multiplied by p\\
 \texttt{ihypamu}            & 1 & 1: velocity hyper coefficient multiplied by amu \\
 \texttt{ihypkappa}          & 1 & 1: pressure hyper coefficient multiplied by kappa \\
\end{tabular}

\subsection{Unit Normalizations}
\begin{tabular}{lcp{4in}}
  \textbf{Option}&\textbf{Default}&\textbf{Description}\\
  \hline
  \texttt{n0\_norm} & $10^{14}$ & Density normalization (in cgs)\\
  \texttt{b0\_norm} & $10^4$    & Magnetic field normalization (in cgs)\\
  \texttt{l0\_norm} & $100$     & Length normalization (in cgs)
\end{tabular}

\subsection{Boundary Conditions}

\begin{tabular}{lcp{4.0in}}
  \textbf{Option} & \textbf{Default} & \textbf{Description}\\
  \hline
 
  \texttt{isurface} & 1 & include surface terms in Galerkin method \\
  \texttt{icurv}    & 2 & if $>$ 0, include curvature from mesh \\
  \texttt{nonrect}  & 0 & 1: non-rectangular boundary \\
  \texttt{ifixedb} & 0 & Set $\psi=0$ on boundary\\
  \texttt{inonormalflow}& 1 & 1: No-normal-flow boundary\\
  \texttt{inoslip\_pol} & 0 & 1: No-slip boundaries for poloidal velocity\\
  \texttt{inoslip\_tor} & 1 & 1: No-slip boundaries for toroidal velocity\\
  \texttt{inostress\_tor}&0 & 1: No-normal-stress boundary for toroidal 
                                 velocity\\
  \texttt{iconst\_bz} & 1 & 1: Toroidal field held constant on boundary\\
  \texttt{iconst\_bn} & 1 & 1: Hold normal field constant on boundary \\
  \texttt{iconst\_n}  & 0 & 1: Density held constant on boundary\\
  \texttt{iconst\_p}  & 1 & 1: Pressure held constant on boundary\\
  \texttt{iconst\_t}  & 0 & 1: Temperature held constant on boundary\\
  \texttt{inograd\_p} & 0 & 1: No normal pressure gradient\\
  \texttt{inograd\_t} & 0 & 1: No normal temperature gradient\\
  \texttt{inograd\_n} & 0 & 1: No normal density gradient \\
  \texttt{com\_bc}& 0 & 1: $\nabla^2 \chi = 0$\\
  \texttt{vor\_bc}& 0 & 1: $\Delta^* U = 0$\\
  \texttt{inocurrent\_pol} & 0 & 1: no poloidal current on boundary \\
  \texttt{inocurrent\_tor} & 0 & 1: no toroidal current on boundary \\
  \texttt{inocurrent\_norm}& 0 & 1: no normal current on boundary \\
  \texttt{ifbound}  & -1 & boundary condition on $f\prime$  \\
                    &    & 1: Dirichlet  \\
                    &    & 2: Neumann    \\
  \texttt{iconstflux}  & 0 & 1: conserves toroidal flux in nonlinear calculation \\
  \texttt{tebound}     & -1 & boundary condition for electron temperature \\
  \texttt{tibound}     & -1 & boundary condition for ion temperature \\
  \texttt{iper}   & 0 & 1: Left/right boundaries periodic\\
  \texttt{jper}   & 0 & 2: Top/bottom boundaries periodic\\

\end{tabular}



\subsection{Time Step}
\begin{tabular}{lcp{4in}}
  \textbf{Option}&\textbf{Default}&\textbf{Description}\\
  \hline
  \texttt{dt}         & 0.1 & Initial size of ime step\\
  \texttt{ntimemax}   & 20  & Total number of time steps\\
  \texttt{integrator} & 0   & 0: Crank-Nicholson (CN); 1: BDF2\\
  \texttt{imp\_mod}   & 0   & 
  \begin{minipage}[t]{4in}
    0: $\theta$-implicit\\
    1: Implicit leapfrof (\texttt{isplitstep} = 1 only)\\
  \end{minipage}\\
  \texttt{thimp}      & 0.5 & Implicitness parameter\\
  \texttt{thimpsm}    & 1   & Implicitness of the smoother functions\\
  \texttt{isplitstep} & 1   & 0: Fully implicit time step; 
                              1: split time step.\\
  \texttt{iteratephi} & 0   & 1: Calculate transport coefficients after
    field advance, then recalculate field advance. \\
  \texttt{idiff} & 0 & 1: solve for difference between n and n+1 in B,p \\
  \texttt{idifv} & 0 & 1: solve for difference between n and n+1 for V \\
  \texttt{irecalc\_eta} & 0 & 1:recalculate transport coefficients after density solve \\
  \texttt{iconst\_eta}  & 0 & 1" don't evolve resistivity \\
  \texttt{itime\_independent} & 0 & 1:exclude d/dt terms \\
  \texttt{harned\_mikic}  & 0 & coefficient of Harned-Mikic 2F stabilization term \\
  \texttt{isources}       & 0 & 1:include source terms in velocity advance \\
  \texttt{nskip}   & 1 & number of time steps per matrix recalculation \\
  \texttt{pskip}   & 1 & number of timesteps the preconditioner is reused \\
  \texttt{iskippc} & 1 & number of times preconditioner is reused \\
  \texttt{ddt}     & 0 & change in timestep per timestep \\
  \texttt{frequency} & 0 & frequency in time-independent calculation \\
  \texttt{dtmin}     & 4.0 & minimum timestep for variable timestep calculation \\
  \texttt{dtmax}     & 40. & maximum timestep for variable timestep calculation \\
  \texttt{dtkecrit}  & 0 &  lower timestep if ekin es above this (0.01 typical) \\
  \texttt{ dtfrac}   & .10 & max fractional change of timestep in 1 cycle \\
  \texttt{max\_repeat}  & 3 & max number time step is repeated for ksp\_max iterations exceeded \\
  \texttt{ksp\_max}     & 10000 & max number of ksp iterations before repeating time step \\
  \texttt{ksp\_min}     & 1200  & increase dt if ksp < ksp\_min  \\
  \texttt{ksp\_warn}    & 1600  & decrease dt if ksp > ksp\_warn \\
\end{tabular}

\subsection{Mesh}

\begin{tabular}{lcp{4.0in}}
  \textbf{Option} & \textbf{Default} & \textbf{Description}\\
  \hline
  \texttt{nplanes}& 1 & number of toroidal planes for 3D nonlinear \\
  \texttt{xzero}  & 0 & $R$-coordinate of bottom left corner of domain\\
  \texttt{zzero}  & 0 & $z$-coordinate of bottom left corder of domain\\
  \texttt{tiltangled} & 0 & angle a rectangular mesh is tilted \\
  \texttt{mesh\_model}&   & model file name from which the mesh is generated \\
  \texttt{mesh\_filename} & & mesh name of .smb files (without number) \\
  \texttt{imatassemble}  & 0 & 1: use petsc matrix parallel assembly instead of scorec \\
  \texttt{imulti\_region} & 0 & 1: Mesh has multiple regions that include resistive wall and
                               vacuum.   Wall resistivity is "eta\_wall", vacuum
                               resistivity is "eta\_vac"  \\
  \texttt{toroidal\_pack\_angle} & 0 & toroidal angle of maximum mesh packing \\
  \texttt{toroidal\_pack\_factor} & 1 & ratio of longest to shortest toroidal element \\

  \texttt{iread\_vmec}    & 0 & 1: read geometry from VMEC file \\
  \texttt{vmec\_filename} & geometry.nc& name of vmec data file \\
  \texttt{nperiods} & 1 & model 1/nperiods of a torus when ifull\_torus=0 \\
  \texttt{iread\_planes} & 0 & Read positions of toroidal planes from plane\_positions \\
  \texttt{bloat\_distance} & 0 & factor to expand VMEC domain \\
  \texttt{bloat\_factor}   & 0 & factor to expand VMEC domain \\
  \texttt{ifull\_torus}    & 0 & 0: one field period, 1: full torus \\
  \texttt{igeometry}       & 0 & 0: default, identity \\
  \texttt{xcenter}         & 0 & center of logical mesh (x) \\
  \texttt{zcenter}         & 0 & center of logical mesh (z) \\
  \texttt{bound\_type(i)}   & 0 & Boundary conditions to apply on mesh loop \texttt{i}.  0 = None, 1 = First wall, 2 = Domain boundary\\
  \texttt{zone\_type(i)}    & 0 & Physics model of mesh zone \texttt{i}.  1 = plasma, 2 = conductor, 3 = vacuum.\\
\end{tabular}

\subsection{Solver}
\begin{tabular}{lcp{4in}}
  \textbf{Option}&\textbf{Default}&\textbf{Description}\\
  \hline
  \texttt{solver\_type} & 0 &  for PETSc only,  0: direct solve, 1: iterative solver.   
                               for Trilinios, iterative solver is used \\
  \texttt{solver\_tol}  & 1.e-9 & solver tolerance \\
  \texttt{num\_iter}    & 100 &   maximum number of iterations 
\end{tabular}

\subsection{Mesh Adaptation (will be depricated soon)}
\begin{tabular}{llcp{4in}}
\textbf{Option}&\textbf{Default}&\textbf{Description}\\
  \hline
 \texttt{iadapt} & 0 & 0: no adaptation \\
                 &   & 1:adapt mesh from the flux field in equilibrium
\end{tabular}


\subsection{Numerical Options}
\begin{tabular}{llp{4in}}
  \textbf{Option}&\textbf{Default}&\textbf{Description}\\
  \hline
 \texttt{int\_pts\_main}  & 25 & Sampling points for integrations in
                                main time step matrices\\
  \texttt{int\_pts\_aux}   & 25 & Sampling points for integrations in
                                calculations of auxiliary variables\\
  \texttt{int\_pts\_diag}  & 25 & Sampling points for integrations in
                                diagnostic calculations\\
  \texttt{int\_pts\_tor}   & 5 & Max number of toroidal integration points \\
  \texttt{ivform} & 0   & 0: $\u = \grad{U}\times\grad{\tor} + V
   \grad{\tor} + \grad{\chi}$\\
   & & 1: $\u = \r^2 \grad{U}\times\grad{\tor} + \r^2 \omega
   \grad{\tor} + \r^{-2} \grad{\chi}$\\
   & & Now depricated to ivform=1 only \\
  \texttt{jadv}   & 0   & 1: Use toroidal current density equation
                          instead of poloidal flux equation.\\
  \texttt{max\_ke}& 1.0  & Maximum value of kinetic energy before solution is
                          rescaled in linear simulations. (0 = don't rescale)\\
  \texttt{equilibrate} & 0 & 1: scale trial function so L2 norm = 1 \\
  \texttt{regular}     & 0 & regularization constant in chi equation \\
  \texttt{iset\_pe\_floor} & 0 & 1: do not let pe drop below pe\_floor \\
  \texttt{pe\_floor}       & 0 & minimum value for pe when iset\_pe\_floor=1 \\

 \texttt{iset\_pi\_floor} & 0 & 1: do not let pi drop below pe\_floor \\
  \texttt{pi\_floor}	   & 0 & minimum value for pi when iset\_pi\_floor=1 \\

 \texttt{iset\_te\_floor} & 0 & 1: do not let te drop below te\_floor \\
  \texttt{te\_floor}	   & 0 & minimum value for te when iset\_te\_floor=1 \\

 \texttt{iset\_ti\_floor} & 0 & 1: do not let ti drop below ti\_floor \\
  \texttt{ti\_floor}	   & 0 & minimum value for ti when iset\_ti\_floor=1 \\

 \texttt{iset\_ne\_floor} & 0 & 1: do not let ne drop below ne\_floor \\
  \texttt{ne\_floor}	   & 0 & minimum value for ne when iset\_ne\_floor=1 \\

 \texttt{iset\_ni\_floor} & 0 & 1: do not let ni drop below ni\_floor \\
  \texttt{ni\_floor}	   & 0 & minimum value for ni when iset\_ni\_floor=1 \\




  \texttt{iprecompute\_metric} & 0 & 1: precompute metric temsor
\end{tabular}

\subsection{Input Options}

\begin{tabular}{lcp{4in}}
  \textbf{Option}&\textbf{Default}&\textbf{Description}\\
  \hline
  \texttt{iread\_eqdsk}   & 0 & 1: Read EFIT g-file 'geqdsk'\\
                          &   & 2: Read psi from geqdsk, but use analytic profiles for p and F \\
                          &   & 3: read profiles from geqdsk, but not the psi \\
  \texttt{iread\_dskbal}  & 0 & 1: Read BAL file 'dskbal'\\
  \texttt{iread\_jsolver} & 0 & 1: Read Jsolver file 'fixed' \\

 \texttt{iread\_omega}       & 0 & 1: reads in rotation profile \\
 \texttt{iread\_omega\_ExB}  & 0 & 1: read ExB rotation \\
 \texttt{iread\_omega\_e}    & 0 & 1: read electron rotation \\
 \texttt{iread\_ne}          & 0 & 1: read in electron  density profile \\
 \texttt{iread\_te}          & 0 & 1: read in temperature profile \\
 \texttt{iread\_p}           & 0 & 1: read pressure profile from profile\_p \\
 \texttt{iread\_neo}         & 0 & 1: read velocity profiles from NEO output \\
 \texttt{ineo\_subtract\_diamag} & 0 & 1: subtract diamagnetic term from input vel when reading neo velocity \\
 \texttt{iread\_heatsource}      & 0 & 1: read heat source profile (psi normalized) scaled by ghs\_rate \\
 \texttt{iread\_partilesource}   & 0 & 1: read particle source profile (psi normalized) scaled with pellet\_rate \\
 \texttt{iread\_f}               & 0 & 1: read R x BT from file \\
 \texttt{iread\_j}               & 0 & 1: read current density from a file





\end{tabular}

\subsection{Output Options}

\begin{tabular}{lcp{4in}}
  \textbf{Option}&\textbf{Default}&\textbf{Description}\\
  \hline
  \texttt{ntimepr}   & 5 & Number of timesteps per full field output\\
  \texttt{iprint}    & 0 & 1: Print detailed info to stdout\\
 \texttt{ifout}      & -1 & 1: ourput f field \\
 \texttt{idouble\_out} & 0 & 1: use double precision floating in putput hdf5 files \\
 \texttt{itemp\_plot}  & 0 & 1: output vdotgradt, deldotq\_perp, deldotq\_par, eta\_jsq \\
 \texttt{ibdgp}        & 0 & optons for plotting partial terms for bdgp plot \\
                       &   & (1) $[ \psi, \Phi] $, (2) $(f\prime,\Phi) $, (3) $-R^{-2} F \Phi \prime$ \\
 \texttt{iveldif}      & 0 & option for plotting partial terms for veldif plot \\
                       &   & (1) $|/psi,U|$, (2) $R^{-3}(\psi,chi) + R^{-2}|\chi,f\prime|$ \\
                       &   & (3)$ R^{-1} \Phi \prime$   (4) $R^{-1} \Phi \prime + R^{-1} F U\prime $ \\
                       &   & (5) $|\psi,U| + R(U,f\prime) - R^{-1}F U\prime $ (6) $R^{-2} |/chi,f\prime| $ \\
 \texttt{icalc\_scalars} & 1 & 1 : calculate scalar diagnostics \\
 \texttt{ike\_only}      & 0 & 1 " only calculate ke scalar diagnostic \\
 \texttt{ike\_harmonics} & 0 & number of toroidal harmonics of kinetic energy to be calculated for diagnostics \\
 \texttt{ibh\_harmonics} & 0 & number of toroidal harmonics of magnetic energy to be calculated for diagnostics \\
 \texttt{irestart}       & 0 & 0: start from timestep 0 \\
                         &   & 1: normal restart (can restart 3D from 2D) \\
                         &   & 3: start a 2D complex run from a 2D real restart \\
 \texttt{irestart\_slice}  & -1 & if set to an integer, restart from that time slice \\
 \texttt{itimer}           & 0 & 1: output internal timeing data \\
 \texttt{iwrite\_transport\_coefs}  & 1 & 1: output transport coefficients fields \\
 \texttt{iwrite\_aux\_vars}         & 1 & 1:output auxiliary variable fields \\

\end{tabular}


\subsection{Diagnostics}

\begin{tabular}{lcp{4in}}
  \textbf{Option}&\textbf{Default}&\textbf{Description}\\
  \hline
 \texttt{xray\_detector\_enabled} & 0 &  1:  enable xray detector \\
 \texttt{xray\_r0}     & 0 &  R coordinate of xray detector \\
 \texttt{xray\_phi0}   & 0 &  phi coordinate of xray detector \\
 \texttt{xray\_z0}     & 0 &  Z coordinate of xray detector \\
 \texttt{xray\_theta}  & 0 & angle of xray detector chord (degrees) \\
 \texttt{xray\_sigma} & 1 &  spread of xray detector chord (degrees) \\
 \texttt{imag\_probes}& 0 &  number of magnetic probes \\
 \texttt{mag\_probe\_x(i)} & 0 &    R coordinate of magnetic probe i \\
 \texttt{mag\_probe\_phi(i)} & 0 & phi coordinate of magnetic probe i \\
 \texttt{mag\_probe\_z(i)}   & 0 & Z coordinate of magnetic probe i \\
 \texttt{mag\_probe\_nx(i)}  & 0 & R component of normal vector of mag probe i \\
 \texttt{mag\_probe\_nphi(i)}& 0 & phi component of normal vector of mag probe i \\
 \texttt{mag\_probe\_nz(i)}  & 0 & Z component of normal vector of mag probe i \\
 \texttt{iflux\_loops}       & 0 & number of flux loops \\
 \texttt{flux\_loop\_x(i)}   & 0 & R coordinate of flux loop i \\
 \texttt{flux\_loop\_z(i)}   & 0 & Z coordinate of flux loop i \\
 \texttt{ifixed\_temax}      & 0 & 1: temax evaluated at (xmag0,0,zmag0)

\end{tabular}
\subsection{Sources/Sinks}

\begin{tabular}{llp{4.5in}}
  \textbf{Option}&\textbf{Default}&\textbf{Description}\\
  \hline
  \texttt{ibeam}    & 0 &  neutral beam source \\
                    &   &  1:include neutral beam particle, energy, and momentum source: \\
                    &   &  $S = \frac {nb_n  }{4 r \pi^2 n b_{dr}  }
                            \exp - \left[ (r - nb_r)^2 + (z - nb_z)^2 \right]/ 2 nb_{dr}^2$  \\
                    &   &  2:include only particle and energy source  \\
                    &   &  3:include only energy source  \\
                    &   &  4:include only momentum and energy source  \\
                    &   &  5:include only momentum source  \\
  \texttt{beam\_x}  & 0 & R coordinate of beam center (in m) \\
  \texttt{beam\_z}  & 0 & Z coordinate of beam center (in m) \\
  \texttt{beam\_v}  & 1.e4 & beam voltage (in volts) \\
  \texttt{beam\_rate}  & 0 & ions/second deposited by beam \\
  \texttt{beam\_dr}    & .1 & dispersion of beam deposition \\
  \texttt{beam\_dv}    & 100. & dispersion of beam voltage (in volts) \\                        
  \texttt{beam\_fracpar}  & 1 & cosine of beam angle relative to parallel (for momentum source)   \\
\hline                  
  \texttt{vloop}          & 0 & initial loop voltage,   NOTE: to change vloop at restart time, must have control\_type = -1 \\
  \texttt{tcur}           & 0 & target (scaled) plasma current for current control: $\mu_0 I_P$ \\
  \texttt{tcuri}          & 0 & if tcuri .ne. tcurf, the target current is a function of time \\
  \texttt{tcurf}          & 0 & tcur = tcuri + (tcurf-tcuri) x .5 x (1. + tanh((t-tcur\_t0)/tcur\_tw)) \\
  \texttt{tcur\_t0}       & 0 & see tcurf \\
  \texttt{tcur\_tw}       & 0 & see tcurf \\                                       
  \texttt{control\_type}  & -1. &  0 old current control algorithm (not recommended) \\
                          &     &  1: standard PID  control with the following control parameters \\                                                        
  \texttt{control\_p}     & 0 & proportional control coefficient \\
  \texttt{control\_i}     & 0 & integral control coefficient \\                 
  \texttt{control\_d}     & 0 & derivative control coefficient  \\

  \hline
  \texttt{ipellet}      & 0    & density source if non-zero (3D part equals 1 for 2D runs).  
                                 Double-digit values have volume integrals normalized to 1.  
                                 Make netative for initial perturbation only. \\
                        &   & Define:  $G_{2D} = \frac{1}{2 \pi R  V_P^2}\exp \left[ - \frac{ (R - R_P)^2 + (Z - Z_P)^2}{2 V_P^2}
                                                                                \right]$  \\
                        &   & 1: $ S = G_{2D} \times \frac{R}{\sqrt{2 \pi}V_t} 
                                  \exp \left[- \frac{RR_P\left(1 - \cos(\phi - \phi_P)\right)}{V_t^2} \right] $  \\
                        &   & 2: $ S = \mbox{den0} \times \left( \mbox{max} (p , p_{edge} )/p_0 \right)^{expn} $ 2D and 3D \\
                        &   & 3: Gaussian sourrce proportional to pressure \\
                        &   & $ S = p \times G_{2D} \times \frac{R}{\sqrt{2 \pi}V_p}
                                  \exp \left[- \frac{RR_P\left(1 - \cos(\phi - \phi_P)\right)}{V_p^2} \right] $  \\
                        &   & 4: Same distribution as ipellet=1 in 3D \\
                        &   & $ S = \sqrt{2 \pi} R V_p \times G_{2D} \times \frac{1}{2 \pi V_p V_t}
                                  \exp \left[- \frac{RR_P\left(1 - \cos(\phi - \phi_P)\right)}{V_t^2} \right] $ 
\end{tabular}

\begin{tabular}{llp{4.5in}}
  \textbf{Option}&\textbf{Default}&\textbf{Description}\\
  \hline
        &  & 11: Same as \#1 but numerically normalized \\
        &  & 12: Spherical, Cartesian Gaussian numerically normalized \\
        &  & 2D: $ S = R G_{2D} $ \\
        &  & 3D: $ S = \exp \left[ - \frac{(R\cos\phi - R_p\cos\phi_p)^2 
                                          +(R\sin\phi - R_p\sin\phi_p)^2
                                          +(Z - Z_p)^2}{2 V_P^2} \right] $ \\
        &  & 13: Axisymmetric, toroidal Gaussian, numerically normalized 2D and 3D:
                 $S = G_{2D} $ \\
        &  & 14: Toroidal distribution is a blend of a von Mises and Cauchy distribution \\
        &  &     \begin{equation}
                 \begin{aligned}  
              &  S = G_{2D}\times \frac{R}{\sqrt{2 \pi} V_p} \times \nonumber \\
              &  \left[ (1 - f_c) \exp \left[  - \frac {R R_p \left( 1 - \cos(\phi-\phi_p) \right)}{V_t^2} \right] 
                       + f_c \times \frac{\cosh(\frac{V_t}{\sqrt{RR_p}}) - cos(\phi_p)}
                                             {\cosh(\frac{V_t}{\sqrt{RR_p}}) - cos(\phi - \phi_p)} \right]
                   \nonumber \end{aligned} 
                   \end{equation} \\


        &  & 15: Toroidal von-Mises distribution with angular half-width \\
        &  & $ S = G_{2D} \times \exp \left[ \cos(\phi - \phi_p) / V_p^2                   
                                      \right] $ \\
        &  & Pellet\_var\_tor radians for ipellet=15, distance otherwise \\

  \texttt{ipellet\_z}  & 0  & Atomic number of pellet (0 for main-ion species) \\
  \texttt{ipellet\_abl} & 0 & Turn on pellet ablation.  (Recommend double-digit ipellet for particle conservation) \\
                        &   & 1: Include ablation model [Parks NF94] calibrated on DIII-D (Li) \\
                        &   & 2: Include new ablation model [Parks, 2015] for small pellets (Li) \\
                        &   & 3: Parks model developed 6/20/2017 (for Neon) \\
  \texttt{temin\_abl}   & 0 & Minimum temperature at which ablation turns on \\
  \texttt{iread\_pellet}& 0 & 0: Single pellet defined by scalar parameters below \\
                        &   & 1: Read pellet.dat, with one row per pellet.   The 13
                                 space-delimited columns are: \\
                        &   & pellet\_r, pellet\_phi, pellet\_z, pellet\_rate, pellet\_var,
                              pellet\_var\_tor, pellet\_velr, pellet\_vel\_phi,
                              pellet\_vel\_z, r\_p, cloud\_pel, pellet\_mix, cauchy\_fraction \\    
  \texttt{pellet\_rate} & 0 & Particle number injection rate\\
  \texttt{pellet\_var}  & 1    & Variance of injection profile\\
  \texttt{pellet\_var\_tor} & 0 & toroidal spatial dispersion of pellet source $(V_t)$.  
                                  If zero, pellet\_var\_tor = pellet\_var. \\
  \texttt{pellet\_R}    & 0 & $R$-coordinate of injection profile\\
  \texttt{pellet\_z}    & 0 & $Z$-coordinate of injection profile\\
  \texttt{pellet\_velr} & 0 & initial radial velocity of the pellet \\
  \texttt{pellet\_velphi} & 0 & initial toroidal velocity of pellet \\
  \texttt{pellet\_velz}   & 0 & initial vertical velocity of pellet \\
  \texttt{r\_p}           & 1.e-3 & initial pellet radius \\
  \texttt{cloud\_pel}     & 1     & Parameter used to change the width of the density source if
                                    ablating.   In this case, pellet\_var = cloud\_pel * r\_p \\
  \texttt{pellet\_mix}   & 0 & Molar fraction of diatomic main ion molecures in pellet (e.g., $D_2$ )

\end{tabular}
\begin{tabular}{llp{4.5in}}
  \textbf{Option}&\textbf{Default}&\textbf{Description}\\
  \hline

  \texttt{irestart\_pellet} & 0 & 1: will read the following from C1input at restart (the rest from *.h5)
                                     pellet\_rate, pellet\_rate\_D2, pellet\_var\_tor, 
                                     pellet\_var, cloud\_pel, pellet\_mix, cauchy\_fraction \\
  \texttt{abl\_fac} & 1.0 &  factor to multiply ablation rate from predefined formula \\ 

  \hline
  \texttt{n\_control\_type} & -1 & -1:  no density control \\
                            &    &  0:  old density control algorithm (not recommended) \\
                            &    &  1:  Standard PID control with the following parameters: \\
  \texttt{n\_control\_p}    & 0  &  proportional feedback constant \\
  \texttt{n\_control\_i}    & 0 &  integral feedback constant \\
  \texttt{n\_control\_d}    & 0 &  derivative feedback constant \\
  \hline
  \texttt{igaussian\_heat\_source} & 0 & 1: include Gaussian heat fource \\
  \texttt{ghs\_x}   & 0 & R coordinate of Gaussian heat source \\
  \texttt{ghs\_z}   & 0 & Z coordinate of Gaussian heat source \\
  \texttt{ghs\_rate}& 0 & amplitude of Gaussian heat source \\
  \texttt{ghs\_var} & 0 & variance of Gaussian heat source \\
  \texttt{ghs\_phi} & 0 & phi coordinate of Gaussian heat source \\
  \texttt{ghs\_var\_tor} & 0 & toroidal variance of GAussian heat source \\
  \hline

  \texttt{ionization}   & 0  & 1: include neutral ionization source \\
  \texttt{ionization\_rate} & 0 & Ionization rate coefficient\\
  \texttt{ionization\_temp} & 0.01 & Ionization energy\\
  \texttt{ionization\_depth}& 0.01 & Temperature scale-length of neutral burn-out
\end{tabular}


\begin{tabular}{llp{4.5in}}
  \hline
current drive source & & $\dot{\psi} = ... + \eta(\Delta^* \psi - J_{CD}) $ \\
                     & & $J_{CD} = J_0 \exp- \left[ (R-R_{0cd})^2 + (Z-Z_{0cd})^2   
                                             \right]/W_{cd} - \Delta_{cd}  $ \\
 \texttt{icd\_source} & 0 & 1: include current drive \\
 \texttt{J\_0cd}      & 0 & $J_0$: Magnitude of Gaussian \\
 \texttt{R\_0cd}      & 0 & $R_{0cd}$: R coordinate of maximum \\
 \texttt{Z\_0cd}      & 0 & $Z_{0cd}$: Z coordinate of maximum \\
 \texttt{W\_cd}       & 0 & $W_{cd}$: Width of maximum \\
 \texttt{delta\_cd}   & 0 & $\Delta_{cd}$: shift of Gaussian \\
\hline
 \texttt{isink}          & 0 & number of density sinks: 0,1,or 2 \\
 \texttt{sink1\_x}       & 0 & R coordinate of first density sink \\
 \texttt{sink1\_z}       & 0 & Z coordinate of first density sink \\
 \texttt{sink1\_rate}    & 0 & rate of first density sink \\
 \texttt{sink1\_var}     & 1 & variance of first density sink \\
 \texttt{sink2\_x}       & 0 & R coordinate of second density sink \\
 \texttt{sink2\_z}       & 0 & Z coordinate of second density sink \\
 \texttt{sink2\_rate}    & 0 & rate of second density sink \\
 \texttt{sink2\_var}     & 1 & variance of second density sink \\
\hline
 \texttt{idenfloor}      & 0 & 1: density in vacuum pegged to den\_edge \\
 \texttt{alphadenfloor}  & 0 & multiplier of (den\_edge - den).  Must be .lt. 1/DT \\ 
%\end{tabular}
%\begin{tabular}{llp{4.5in}}
\hline
 \texttt{ipforce} & 0 & 1: include poloidal momentum source \\
                  &   & $F = f(\tilde{\psi}) \nabla \psi \times \nabla \phi$  \\
                  &   & $f(\tilde{\psi}) = \alpha(1 - \tilde{\psi})^N \frac{ \delta^2    }
                                                                           { (\tilde{\psi} - \psi_0)^2    + \delta^2} $  \\
                  &   & 2:  Include Luca Guzzatto form of momentum source  


\end{tabular}

\begin{tabular}{llp{4.5in}}
  \textbf{Option}&\textbf{Default}&\textbf{Description}\\
  \hline
  \texttt{dforce}  & 0 &  $\delta$ \\
  \texttt{xforce}  & 0 &  $\psi_0$ \\
  \texttt{nforce}  & 0 &  $N$ \\
  \texttt{aforce}   & 0 &  $ \alpha $ \\
  \texttt{iheat\_sink}         & 0 & 1:  special heat sink for itaylor=27  \\
  \texttt{coolrate}            & 0 & S = coolrate*(pedge-p) for iheat\_sink = 1 \\
  \texttt{iarc\_source}        & 0 & 1: density source due to halo current \\
  \texttt{arc\_source\_alpha}  & 0 &   parameter for arc\_source \\
  \texttt{arc\_source\_eta}    & .01 & parameter for arc\_source

\end{tabular}
\subsection{Resistive Wall}

\begin{tabular}{llp{4.5in}}
  \textbf{Option}&\textbf{Default}&\textbf{Description}\\
  \hline
  \texttt{eta\_vac}    & 1 &  resistivity of vacuum region \\
  \texttt{eta\_wall}   & .001 & resistivity of conducting wall regions \\
  \texttt{eta\_wallRZ} & .001 & poloidal resistivity of wall region (if different from eta\_wall) \\
  \texttt{iwall\_break}& 0 & number of wall breaks \\
  \texttt{eta\_break} & 1 & resistivity of wall break (array) \\
  \texttt{wall\_break\_phimax} & 0 & max phi coordinate for break (array) \\
  \texttt{wall\_break\_phimin} & 0 & min phi coordinate for break (array) \\
  \texttt{wall\_break\_xmax}   & 0 & max R coordinate for break (array) \\
  \texttt{wall\_break\_xmin}   & 0 & min R coordinate for break (array) \\
  \texttt{wall\_break\_zmax}   & 0 & max Z coordinate for break (array) \\
  \texttt{wall\_break\_zmin}   & 0 & min Z coordinate for break	(array)	\\ 
  \texttt{iwall\_regions}      & 0 & number of resistive wall regions \\
  \texttt{wall\_region\_eta()}  & 1.e-3 & resistivity of each wall region \\
  \texttt{wall\_region\_etaRZ()}& 1.e-3 & poloidal restivity (if different from wall\_region\_eta) \\
  \texttt{wall\_region\_filename()} & &   file name will wall contour points \\
  \texttt{eta\_zone(i)} & eta\_wall & Resistivity of mesh region \texttt{i}.  Only applies if \texttt{zone\_type(i)} = 2 (conductor).\\
  \texttt{etaRZ\_zone(i)} & eta\_zone(i) & Poloidal resistivity of mesh region \texttt{i}.  Only applies if \texttt{zone\_type(i)} = 2 (conductor).\\ 
  \texttt{eta\_rekc}  & 0 & resistivity of runaway electron killer coil (REKC) \\
  \texttt{ntor\_rekc} & 0 & toroidal mode number of REKC  \\
  \texttt{mpol\_rekc} & 0 & poloidal mode number of REKC \\
  \texttt{phi\_rekc}  & 0 & toroidal angle of fixed point of REKC \\
  \texttt{theta\_rekc}& 0 & poloidal angle of fixed point of REKC \\
  \texttt{rzero\_rekc}& 0 & R0 for computing theta of REKC \\
  \texttt{zzero\_rekc}& 0 & Z0 for computing theta of REKC \\
  \texttt{isym\_rekc} & 0 & if non-zero, coil is double helix with (+,-) mpol\_rekc
\end{tabular}

\subsection{Miscellaneous}

\begin{tabular}{llp{4.5in}}
  \textbf{Option}&\textbf{Default}&\textbf{Description}\\
  \hline
 \texttt(
 \texttt{gam}                    & 5/3 & ratio of sepcific heatx \\
 \texttt{db}                     & 0 & ion skin depth (overrides db\_fac \\
 \texttt{db\_fac}                & 0 & factor multiplying physical value of ion skin depth \\
 \texttt{mass\_ratio}            & 0 & ratio of ion to electron mass \\
 \texttt{lambdae}                & 0 & lambdae \\
 \texttt{z\_ion}                 & 1 & Z-effective \\
 \texttt{ion\_mass}              & 1 & ion mass in units of m\_p \\
 \texttt{lambda\_coulomb}         & 17 & Couloumb logarithm \\
 \texttt{thermal\_force\_coeff}  & 0 &  coefficient of thermal force \\
 \texttt{ntor}                   & 0 &  toroidal mode number for 3D linear (complex) \\
 \texttt{mpol}                   & 0 &  poloidal mode number for certain test problem initializations 

\end{tabular}

\subsection{Deprecated}

\begin{tabular}{llp{4.5in}}
  \textbf{Option}&\textbf{Default}&\textbf{Description}\\
  \hline
 \texttt{ipartitioned}   & 0 &     \\
 \texttt{igs\_method}    & -1&     \\
 \texttt{ibform}         & -1&     \\
 \texttt{delta\_wall}    & 1.&
\end{tabular}

\subsection{Trilinos Options}

\begin{tabular}{llp{4.5in}}
  \textbf{Option}&\textbf{Default}&\textbf{Description}\\
  \hline
 \texttt{drop\_tolerance}    & 0 &   ILU drop tolerance \\
 \texttt{graph\_fill}        & 0 & graph fill level \\
 \texttt{ilu\_fill\_level}   & 1 & ILU fill level \\
 \texttt{ilu\_omega}         & 1 & relaxation parameter for rILU \\
 \texttt{krylov\_solver}     & gmres & Krylov solver \\
 \texttt{poly\_ord}          & 1     & polynomial order for certain perconditioners \\
 \texttt{preconditioner}     &dom\_decomp & preconditioner \\
 \texttt{sub\_dom\_solver}   & ilu & subdomain solver in preconditioner \\
 \texttt{subdomain\_overlap} & 1   & subdomain overlap 

\end{tabular}

\subsection{Simple Radiation Model}

\begin{tabular}{llp{4.5in}}
  \textbf{Option}&\textbf{Default}&\textbf{Description}\\
  \hline
 \texttt{iprad}    & 0 & 1: call Prad radiation module with one impurity species \\
                   &   & $P_{rad} = n_e n_D L_D (T_e) + n_e n_Z L_Z(T_e)   $  \\
                   &   & Cooling rate of deuterium is $L_D = 5.35 \times 10^{-37}T_e^{1/2}[keV] W-m $ \\
                   &   & $L_Z(T_e)$ taken from Post, et al, Atomic data and nuclear data tables,
                                      {\bf 20} pp. 397-439 (1977)  \\
 \texttt{prad\_fz}  & 1 & density of impurity species as fraction of $n_e$ \\
 \texttt{prad\_z}      & 1 & Z of impurity species: Z=6(C), 18(Argon), 26(Fe) are available \\
 \texttt{iread\_prad}  & 0 & 1: Read impurity density from profile\_nz (units of $10^{20}/m^3$ ) 
\end{tabular}

\subsection{KPRAD Radiation Model}

\begin{tabular}{llp{4.5in}}
  \textbf{Option}&\textbf{Default}&\textbf{Description}\\
  \hline
 \texttt{ikprad}        & 0 & 1: KPRAD module with one impurity species \\
 \texttt{ikprad\_z}     & 1 & Z of impurity species in KPRAD module.   Presently available:
                                  2(He), 4(Be), 6 (C),10(Ne), 18 (Ar) \\
 \texttt{kprad\_fz}      & 0 & density of neutrals as fraction of ne \\
 \texttt{kprad\_nz}      & 0 & Density of neutral impurities \\
 \texttt{kprad\_nemin}   & $10^{-12}$& Minimum normalized electron density for KPRAD evolution \\
 \texttt{kprad\_temin}   & $ 2 \times 10^{-7} $ & Minimum normalized electron temperature for KPRAD evolution \\
 \texttt{ikprad\_max\_dt}&  0 & Set max time step for KPRAD ionization \\
                         &    & 0: MHD time step dt \\
                         &    & 1: {\bf RECOMMENDED:} dt/( kprad\_z + 1 ) : ensures evolution through all charge states. \\
 \texttt{ikprad\_evolve\_internal}  & 0 & Update local temperature during KPRAD subcycling: \\
                                    &   & 0: Te fixed before subcycling \\
                                    &   & 1: {\bf RECOMMENDED} Local ne and Te used for KPRAD 
                                                  ionization/radiation updated during subcycling each 
                                                  KPRAD time step due to density and energy changes \\
\texttt{ikprad\_evolve\_neutrals}   & 0 & Determines how KPRAD  neutrals evolve spatially: \\
                                    &   & 0: Neither advect nor diffuse \\
                                    &   & 1: {\bf RECOMMENDED} Advect and diffuse like other charge states \\
                                    &   & 2: Diffuse but do not advect \\
 

\texttt{ikprad\_min\_option}        & 1 & Determine how KPRAD behaves below minimum density/temperature
                                          (kprad\_nemin, kprad\_temin) \\ 
                                    &   & 1: No radiation/ionization/recombination (based on ne/te before subcycling) \\
                                    &   & 2: {\bf RECOMMENDED} Recombination but no radiation/ionization (based on ne/Te during subcycling) \\
                                    &   & 3: No radiation/ionization/recombination (based on ne/Te during subcycling).  
                                             NOTE: 1 and 3 behave the same if ikprad\_evolve\_internal = 0) \\

\texttt{iread\_lp\_source} & 0 & Read impurity source from Lagrangian Particle code cloud.txt (UNDER DEVELOPMENT)
\end{tabular}

\subsection{Stellarator Geometry}

\begin{tabular}{llp{4.5in}}
  \textbf{Option}&\textbf{Default}&\textbf{Description}\\
  \hline
 \texttt{type\_ext\_field}        & -1& External fields to be read in.. \\
				  &   &	0: Axisymmetric only. For RMP and error fields. \\
				  &   &	1: Free-boundary stellarator (\texttt{itaylor=41}) data. \\
				  &   &	2: Free-boundary stellarator (\texttt{itaylor=41}) data with \texttt{extsubtract=1}. \\
 \texttt{file\_ext\_field}        &   & (string) Vacuum/external field to be subtracted for \texttt{itaylor=41}.\\
				  &   & Currently supported: FIELDLINES, MGRID. \\
				  &   & Must start with `fieldlines' or `mgrid'. \\
                                  &   & Must have \texttt{type\_ext\_field=2}.\\
 \texttt{file\_total\_field}      &   & (string) Total field for free-boundary stellarator (\texttt{itaylor=41}).\\
				  &   & Currently supported: FIELDLINES, MGRID. \\
				  &   & Must start with `fieldlines' or `mgrid'. \\
                                  &   & Must have \texttt{type\_ext\_field=1,2}.\\
 \texttt{iread\_vmec}             & 0 & 1: read VMEC file to determine geometry. Must be 1 for stellarator. \\
 \texttt{vmec\_filename}          &   & (string) VMEC output .nc file \\
 \texttt{bloat\_factor}           & 0 & Free boundary only:  Scale factor to bloat computational boundary using input geometry. \\
 \texttt{bloat\_distance}         & 0 & Free boundary only:  Distance to expand computational boundary from input geometry. \\
 \texttt{igeometry}               & 0 & 1: must be set to use stellarator version \\
 \texttt{nperiods}                & 1 & Number of field periods (stellarator geometry; VMEC nfp must be divisible by nperiods).    Note nplanes
                                        should be equal to at least 2 x (number of toroidal modes 
                                        per field period) x nperiods \\
 \texttt{ifull\_torus}            & 0 & 0:  Solve on one field period \\
                                  &   & 1:  Solve on full torus \\
 \texttt{nzer\_factor}            & -1& (integer) Scale factor for resolution of Zernike polynomial
                                                  (used for interpolation of VMEC) \\
                                  &   &-1: n\_zeri= 2 x mpol (fixed boundary) and 1 x mpol (free boundary). \\
                                  &   &Otherwise: n\_zer = nzer\_factor x mpol where mpol is poloidal resolution in VMEC \\
 \texttt{nzer\_manual}            & -1&(int) Resolution of Zernike polynomial (mainly for testing)

\end{tabular}

\include{app-paraview}
\end{document}

