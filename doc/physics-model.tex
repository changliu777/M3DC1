\section{Physics Model}
\label{sec:physics_model}

\section{Extended-MHD Equations}

The basic physics model for a single-species plasma in M3D-C1 is the following:
\begin{eqnarray}
  \frac{\partial n_i}{\partial t} + \nabla \cdot (n_i \vec{v})
  & = & \nabla \cdot (D \nabla n_i) + \sigma_i
\\
  \label{eq:force_balance}
  \rho \left( \frac{\partial \vec{v}}{\partial t}
  + \vec{v} \cdot \nabla \vec{v} \right)
  & = & 
  \vec{J} \times \vec{B} - \nabla p - \nabla \cdot \Pi
  - \varpi\, \vec{v}
  %+ \vec{F}_s
\end{eqnarray}
For a single-species plasma, $n_e = Z_i n_i$, $\rho = m_i n_i$, and
$\varpi = m_i \sigma_i$ (these are modified for;5D multi-species plasmas,
as described in section~\ref{sec:impurities}).  The final term in
equation~(\ref{eq:force_balance}) represents the slowing of the fluid
velocity to conserve momentum as new particles are added.

The magnetic field is advanced using Faraday's law
\begin{equation}
\label{eq:Faraday}
\frac{\partial \vec{B}}{\partial t} = -\nabla \times \vec{E}
\end{equation}
The electron momentum equation defines the generalized Ohm's law
\begin{equation}
\label{eq:ohm}
  \vec{E} = - \vec{v} \times \vec{B} + \eta\, \left(\vec{J} - \vec{J}_x \right)
  + \frac{d_i}{n_e}\left(\vec{J} \times \vec{B} - \nabla p_e \right)
\end{equation}
where $\vec{J}_x$ represents an external force on the electrons.

Several models for advancing the ion and electron temperatures, either
together or independently, are implemented in M3D-C1.  Each of these
models are built from the electron temperature equation:
\begin{multline}
  \label{eq:electron_temperature}
  n_e \left[ \frac{\partial T_e}{\partial t} + \vec{v} \cdot \nabla T_e
  + (\Gamma - 1) T_e \nabla \cdot \vec{v} \right] + \sigma_e T_e
  = \\
  (\Gamma - 1) \left[ \eta \vec{J} \cdot \left(\vec{J} - \vec{J}_x \right)
  - \nabla \cdot \vec{q}_e + Q_e
  -\Pi_e : \nabla \vec{v} \right]
\end{multline}
and the ion temperature equation is:
\begin{multline}
  \label{eq:ion_temperature}
  n_i \left[ \frac{\partial T_i}{\partial t} + \vec{v} \cdot \nabla T_i
  + (\Gamma - 1) T_i \nabla \cdot \vec{v} \right] + \sigma_* T_i
  = \\
  (\Gamma - 1) \left[ - \nabla \cdot \vec{q}_i + Q_i
  -\Pi_i : \nabla \vec{v}  + \frac{1}{2} \varpi\, v^2
  \right]
\end{multline}
The final term in equation~(\ref{eq:ion_temperature})
accounts for the net loss of kinetic energy caused by the slowing of
the fluid velocity as new particles are added.  Models implemented
include a single total pressure equation in which $p_e / p_i$ is
assumed fixed; a single total temperature equation in which $T_e /
T_i$ is assumed fixed; a two-pressure equation in which the total
pressure $p$ and electron pressure $p_e$ are evolved separately; and a
two-temperature model in which the electron temperature $T_e$ and ion
temperature $T_i$ are evolved independently using
equations~(\ref{eq:electron_temperature}) and
(\ref{eq:ion_temperature}).  Each of these models conserves total
energy, in the sense that all sources and sinks are are due to
physical processes that are included in the model.

The electron heat source density $Q_e$ is the sum of the heating from
radiation and ionization $Q_{rad}$ (this is negative for power lost
from the plasma) and the collisional transfer of energy from the ions
to the electrons $Q_\Delta$~\cite{Braginskii65}.  The radiated power
is calculated using the KPRAD module, and is the sum of the
bremsstrahlung, line radiation, and recombination losses.  We note
that KPRAD calculates both the kinetic energy and potential energy
released as radiation during recombination, but we only include the
kinetic contribution in $Q_e$, since the potential contribution does
not subtract from the kinetic energy of the electron fluid.

The equipartition term is the sum of the energy transfer from all ion
species to electrons:
\begin{equation}
  Q_\Delta = 3 m_e n_e (T_i - T_e)
  \left(\frac{\nu_{ei}}{m_i}
  + \sum_{j=1}^Z \frac{\nu_{eZ}^{(j)}}{m_Z} \right)
\end{equation}
where the collision rate between electrons and ion species $j$ is
\begin{equation}
  \nu_{ej} = \frac{4 \sqrt{2 \pi} e^4
    Z_j^2 n_j \ln \Lambda}{3 \sqrt{m_e} T_e^{3/2}}
\end{equation}
Therefore one can write
\begin{equation}
  Q_\Delta = 3 \mu \, \nu_{e H}\, n_e\, (T_i - T_e)
\end{equation}
where
\begin{equation}
  \mu = \frac{m_e}{n_e} \left(\frac{Z_i^2 n_i}{m_i} +
  \sum_{j=1}^Z \frac{j^2 n_Z^{(j)}}{m_Z} \right)
\end{equation}
and where $\nu_{e H}$ is the collision frequency for a pure Hydrogen
plasma ($Z_j=1$, $m_j = m_p$, $n_j = n_e$).  Similarly, summing the
collisional drag between the electrons and each ion species, the total
resistivity can be written in terms of the collision frequency for a
pure Hydrogen plasma scaled by an ``effective $Z$''~\cite{Wesson87}:
\begin{align}
  \label{eq:eta}
  \eta & = \frac{m_e \nu_{e H}}{n_e e^2} Z_{eff}
  \\
  Z_{eff} & = \frac{Z_i^2 n_i + \sum_{j=1}^Z j^2 n_Z^{(j)}}{n_e}
\end{align}
The heat flux densities are defined by
\begin{align}
  \vec{q}_e & =
  -\kappa^{e} \nabla T_e
  - \kappa^{e}_\parallel \frac{\vec{B} \vec{B}}{B^2} \cdot \nabla T_e
  \\
  \vec{q}_i & =
  -\kappa^{i} \nabla T_i
  - \kappa^{i}_\parallel \frac{\vec{B} \vec{B}}{B^2} \cdot \nabla T_i
\end{align}



\subsection{Impurity model}
\label{sec:impurities}

The impurity model evolves each individual
charge state of the impurity species, $n_Z^{(j)}$ for $1 \le j \le Z$,
by including the continuity equations
\begin{equation}
  \frac{\partial n_Z^{(j)}}{\partial t} + \nabla \cdot (n_Z^{(j)} \vec{v})
  = \nabla \cdot (D \nabla n_Z^{(j)}) + \sigma_Z^{(j)}
\end{equation}
The electron density is defined to satisfy quasi-neutrality
\begin{equation}
  n_e = Z_i n_i + \sum_{j=1}^Z j\, n_Z^{(j)}
\end{equation}
The particle source density for each impurity charge state due to
ionization and recombination, $\sigma_Z^{(j)}$, is calculated using
the KPRAD model~\cite{Whyte97}.

In addition, the mass density $\rho$ and mass density source rate
$\varpi$ are modified to account for the impurity density and sources:
\begin{eqnarray}
  \rho & = & m_i n_i + \sum_{j=1}^Z m_Z n_Z^{(j)}
  \\
  \varpi & = & m_i \sigma_i + \sum_{j=1}^Z m_Z \sigma_Z^{(j)}
\end{eqnarray}

It is assumed that all ionized impurities have the same temperature as
the main ion species, $T_i$.  The ion temperature equation is replaced
by the sum of the temperature equations for all ion species
(\textit{i.e.} main ions and all ionized impurities):
\begin{multline}
  \label{eq:ion_temperature_multispecies}
  n_* \left[ \frac{\partial T_i}{\partial t} + \vec{v} \cdot \nabla T_i
  + (\Gamma - 1) T_i \nabla \cdot \vec{v} \right] + \sigma_* T_i
  = \\
  (\Gamma - 1) \left[ - \nabla \cdot \vec{q}_* + Q_*
  -\Pi_* : \nabla \vec{v}  + \frac{1}{2} \varpi\, v^2
  \right]
\end{multline}
Here, $n_*$, $\sigma_*$,$\Pi_*$, $Q_*$, and $\vec{q}_*$ are sum over  
all ion species of the particle densities, particle source densities,
stresses, energy density sources, and energy density fluxes,                    
respectively.
